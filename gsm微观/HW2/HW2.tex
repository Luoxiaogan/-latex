\documentclass[10pt, a4paper, oneside]{ctexart}
\usepackage{amsmath, amsthm, amssymb, bm, color, xcolor, framed, graphicx, hyperref, mathrsfs, etoolbox, wrapfig, xurl,booktabs}
\usepackage[thicklines]{cancel}
\usepackage{enumitem} % 用于更灵活的列表环境
\usepackage{geometry} % 调整页面边距
\usepackage{fancyhdr} % 页眉页脚定制
\hypersetup{
    colorlinks=true,            %链接颜色
    linkcolor=black,             %内部链接
    filecolor=magenta,          %本地文档
    urlcolor=cyan,              %网址链接
    pdftitle={Overleaf Example},
    pdfpagemode=FullScreen,
    }

% 页边距设置
\geometry{left=3cm, right=3cm, top=2.5cm, bottom=2.5cm}

% 页眉页脚设置
\pagestyle{fancy}
\fancyhf{}
\fancyhead[L]{\leftmark} % 左页眉显示章节标题
\fancyhead[R]{\thepage} % 右页眉显示页码

% 标题设置
\title{\textbf{gsm 微观 HW2}}
\author{罗淦  2200013522}
\date{\today}

% 行距设置
\linespread{1.2}

% 定义黑色边框的 problem 环境
\newenvironment{problem}{\begin{framed}\par\noindent\textbf{\textit{题目. }}}{\end{framed}\par}
\newenvironment{solution}{%
  \par\noindent\textbf{\textit{解答. }}\ignorespaces
}{%
  \hfill\ensuremath{\square}\par % 在结尾添加正方形
}
\newenvironment{note}{\par\noindent\textbf{\textit{题目的注记. }}\ignorespaces}{\par}


% 允许公式在页面之间自动换行
\allowdisplaybreaks

\begin{document}

\maketitle

% 添加目录
%\tableofcontents
%\newpage
\section{HW 1}

\subsection{Rosa喝咖啡}

\begin{solution}
(1) Rosa的"至少一样好"的关系是不可传递的. A和B相差0.75茶勺, 因此无法分辨, A和B至少一样好: $A\succeq B$. B和C相差0.75茶勺, 因此无法分辨, B和C至少一样好:  $A\succeq C$. 但是A和C差了1.5茶勺, 因此可以分辨, C比A好, 因此$A\succeq C$不成立, 所以传递性不成立.\\
注: 在无法分辨的情况下,既可以说$A\succeq B$, 也可以说$B\succeq A$.

(2) Rosa的"无法区分"的关系是不可传递的. 已知$A\sim B$, $B\sim C$, 但是$A\sim C$. 因此不具有传递性.

(3) Rosa的"更好"的关系是可传递的. 考虑其他的情况, 如果$A\succ B, B\succ C$, 那么A至少比B多一茶勺, B至少比C多一茶勺. 那么A至少比C多两茶勺, 有$A\succ C$.
\end{solution}

\subsection{小明同学的效用函数}
\begin{solution}
(1) 无差异曲线$(x_A+1)(x_B+1)=u$, 此处的$u$是定值 
(2)和(3)
\begin{figure}[h]
    \centering
    \includegraphics[width=0.8\textwidth]{image/2.png}
\end{figure}   
\end{solution}

\subsection{宁的效用函数}
\begin{solution}
For simplicity, $x_A=x, x_B=y$

(a) 预算约束: $x+2y\leq 40$.

(b) 最优组合$(20,10)$

求解$\max_{x+2y\leq 40}xy, L=xy+\lambda(x+2y-40), \begin{cases}
    L_{x}&=y+\lambda=0\\
    L_{y}&=x+2\lambda=0\\
    L_{x}&=x+2y-40=0\\
\end{cases}\Rightarrow x^*=20,y^*=10$

(c) 最大效用$u_{\max}=x^*y^*=200$

(d) 此时, 新的预算约束$x+3y\leq 40$.

求解$\max_{x+3y\leq 40}xy, L=xy+\lambda(x+3y-40), \begin{cases}
    L_{x}&=y+\lambda=0\\
    L_{y}&=x+3\lambda=0\\
    L_{x}&=x+3y-40=0\\
\end{cases}\Rightarrow x^*=20,y^*=\frac{20}{3}$

此时的效用是$u=\frac{400}{3}$.

要计算替代效用和收入效应, 那么在新的价格下, 如果保持购买力不变, 此时的补偿预算约束是: $x+3y\leq 50$\\
求解$\max_{x+3y\leq 50}xy, L=xy+\lambda(x+3y-50), \begin{cases}
  L_{x}&=y+\lambda=0\\
  L_{y}&=x+3\lambda=0\\
  L_{x}&=x+3y-50=0\\
\end{cases}\Rightarrow x^*=25,y^*=\frac{25}{3}$\\
因此:\\
替代效应: $x_{\text{补偿}}^*-x_{\text{最初}}^*=5$, $y_{\text{补偿}}^*-y_{\text{最初}}^*=-\frac{5}{3}$\\
收入效应: $x_{\text{最终}}^*-x_{\text{补偿}}^*=-5$, $y_{\text{最终}}^*-y_{\text{补偿}}^*=-\frac{5}{3}$

\subsection{}

\end{solution}




\end{document}