\documentclass[10pt, a4paper, oneside]{ctexart}
\usepackage{amsmath, amsthm, amssymb, bm, color, xcolor, framed, graphicx, hyperref, mathrsfs, etoolbox, wrapfig, xurl, algorithm, algpseudocode}
\usepackage[thicklines]{cancel}
\usepackage{enumitem} % 用于更灵活的列表环境
\usepackage{geometry} % 调整页面边距
\usepackage{fancyhdr} % 页眉页脚定制
\hypersetup{
    colorlinks=true,            %链接颜色
    linkcolor=black,             %内部链接
    filecolor=magenta,          %本地文档
    urlcolor=cyan,              %网址链接
    pdftitle={Overleaf Example},
    pdfpagemode=FullScreen,
    }

% 页边距设置
\geometry{left=3cm, right=3cm, top=2.5cm, bottom=2.5cm}

% 页眉页脚设置
\pagestyle{fancy}
\fancyhf{}
\fancyhead[L]{\leftmark} % 左页眉显示章节标题
\fancyhead[R]{\thepage} % 右页眉显示页码

\newcommand{\norm}[1]{\| #1 \|}
\newcommand{\ip}[1]{\left\langle#1\right\rangle}

% 标题设置
\title{\textbf{HW 3}}
\author{罗淦  2200013522}
\date{\today}

% 行距设置
\linespread{1.2}

% 定义黑色边框的 problem 环境
\newenvironment{problem}{\begin{framed}\par\noindent\textbf{\textit{题目. }}}{\end{framed}\par}
\newenvironment{solution}{%
  \par\noindent\textbf{\textit{解答. }}\ignorespaces
}{%
  \hfill\ensuremath{\square}\par % 在结尾添加正方形
}
\newenvironment{note}{\par\noindent\textbf{\textit{题目的注记. }}\ignorespaces}{\par}


% 允许公式在页面之间自动换行
\allowdisplaybreaks

\begin{document}

\maketitle

% 添加目录
%\tableofcontents
%\newpage

\section{HW 3}

\begin{problem}
    1. 设$A=\begin{pmatrix}
            1&2\\3&4\\5&6
        \end{pmatrix}$, $b=\begin{pmatrix}
            1\\1\\1
        \end{pmatrix}$, 用正则化方法求$LS$问题的解
    \end{problem}
    \begin{solution}
    \textcolor{blue}{考虑正则化方法, 是考虑正则化方程$A^TAx=A^Tb$}\\
    求解得到:
    \begin{align*}
        \begin{pmatrix}
            35&44\\44&56
        \end{pmatrix}\begin{pmatrix}
            x_1\\x_2
        \end{pmatrix}=\begin{pmatrix}
            9\\12
        \end{pmatrix}\Rightarrow
        \begin{pmatrix}
            x_1\\x_2
        \end{pmatrix}=\begin{pmatrix}
            -1\\1
        \end{pmatrix}
    \end{align*}
    \end{solution}
    
    \begin{problem}
    2. 设$x=(1,0,4,6,3,4)^T$, 求一个Householder变换和一个正数$\alpha$, 使得$Hx=(1,\alpha,4,6,0,0)^T$.
    \end{problem}
    \begin{solution}
    因为Householder变换是正交变换, 因此向量长度不变, 根据$||x||_2^2=\norm{Hx}_2^2$得到正数$\alpha=5$.
    
    两向量相减, 得到镜面对称的法方向$v=(0,-5,0,0,3,4)^T$, 那么Householder变换应该是
    \begin{align*}
        H(v)=I-2\frac{vv^T}{v^Tv}
    \end{align*}
    \textcolor{red}{请问助教老师, 考试的时候需要把真正的矩阵算出来吗?}
    \end{solution}
    \begin{note}
    注意书上的结论: Householder变换可以把一个向量的"任何若干相邻"的元素化为零, 想要把$x\in \mathbb{R}^n$, 长度为$n$的向量的$k+1$到$j$变成$0$, 那么需要把向量
    $$v=(0,\cdots,0,x_{k}-\alpha,\underbrace{x_{k+1},\cdots,x_{j}}_{\text{这是想要变成0的部分}},0,\cdots,0)^T, \quad \textcolor{red}{\alpha=\sqrt{x_k^2+\cdots+x_{j}^2}}$$
    \end{note}
    
    \begin{problem}
    5. $x=(x_1,x_2)^T$是一个二维的复向量, 给出一种算法, 计算酉矩阵
    $$Q=\begin{pmatrix}
        c&\bar{s}\\-s&c
    \end{pmatrix}, c\in R,c^2+|s|^2=1$$
    使得$Q$的第二个分量是$0$.
    \end{problem}
    
    \begin{solution}
    若$x_1=0$, 那么$c=0, s=e^{i\theta}$\\
    若$x_2=0$, 那么$s=0, c=1\text{或}-1$\\
    若$x_1\neq 0,x_2\neq 0$, 那么考虑$\alpha=\arctan \frac{|x_1|}{|x_2|}$, 取$c=\sin \alpha, s=\cos \alpha \cdot \frac{x_1|x_2|}{|x_1|x_2}$, 满足$Qx$的第二个分量是$0$. 即(实际上核心是$cx_2=sx_1$):
    \begin{align*}
        c = \frac{|x_1|}{\sqrt{|x_1|^2+|x_2|^2}}, s=\frac{|x_2|}{\sqrt{|x_1|^2+|x_2|^2}}\cdot\frac{x_1|x_2|}{|x_1|x_2}
    \end{align*}
    \textcolor{red}{但是, 结合书上对Givens变换的分析, 为了给出一个robust的算法, 防止在计算过程中出现溢出, 可以这么做}
    \begin{figure}[h]
        \centering
        \includegraphics[width=0.44\textwidth]{../../image/1.jpg}
    \end{figure}
    \end{solution}
    \begin{problem}
    7. 设$x$和$y$是$\mathbb{R}^n$中的两个非零向量, 给出一种算法来确定一个Householder变换$H$, 使得$Hx=\alpha y, \alpha\in R$.
    \end{problem}
    \begin{solution}
    因为把$x$镜面反射到$y$, 因此法向量$v\in \text{span}(x,y)$, 设$v=x+\beta y$, 因为 
    \begin{align*}
    H(v)x=x-2\frac{v^Tx}{v^Tv}v=(1-2\frac{v^Tx}{v^Tv})x-2\frac{v^Tx}{v^Tv}\beta y
    \end{align*}
    因此$x$的系数$1-2\frac{v^Tx}{v^Tv}=0 \iff \beta=\pm \frac{\norm{x}_2}{\norm{y}_2}$, 此时有:
    \begin{align*}
        H(v)x=\pm \frac{\norm{x}_2}{\norm{y}_2}y, v=x\pm \frac{\norm{x}_2}{\norm{y}_2}y
    \end{align*}
    \end{solution}
    
    \begin{problem}
        设$A\in \mathbb{R}^{m\times n}$(\textcolor{red}{书上typo, 此处就是$\mathbb{R}^{n\times n}$})是一个对角加边矩阵, 即:
        \begin{align*}
            A = \begin{bmatrix}
                \alpha_1 & \rho_2 & \rho_3 & \dots & \dots & \rho_n & 0 \\
                \beta_2 & \alpha_2 & 0 & \dots & \dots & \dots & 0 \\
                \beta_3 & 0 & \alpha_3 & \dots & \dots & \dots & 0 \\
                \vdots & \vdots & \vdots & \ddots & & \vdots & \vdots \\
                \vdots & \vdots & \vdots & & \ddots & \vdots & \vdots \\
                \beta_n & 0 & \dots & \dots & \dots & \alpha_{n-1} & 0 \\
                0 & 0 & \dots & \dots & \dots & 0 & \alpha_n
                \end{bmatrix}
        \end{align*}
        试给出用Givens变换求$A$的$QR$分解的详细算法
    \end{problem}
    \begin{solution}
    使用Givens变换对$A$进行逐列操作, 将$A$变为一个上三角矩阵. 但是这样还不如用Householder变换.
    
    想要利用稀疏矩阵的特性, 可以想到Givens变换$G(i,j,\theta)$左乘作用于矩阵$A$之后, 实际上仅仅改变了矩阵$A$的$i,j$行列交错的四个元素.
    
    因此, 为了消去$\beta_2$, 可以使用$G(1,2,\theta_2)$; 为了消去$\beta_3$, 可以使用$G(1,3,\theta_n)$; 直到为了消去$\beta_n$, 可以使用$G(1,n,\theta_n)$. 因此使用$n$个Givens变换即可达到效果, 具体的算法如下:
    \begin{figure}[h]
        \centering
        \includegraphics[width=0.7\textwidth]{../../image/2.png}
    \end{figure}
    \end{solution}

\end{document}