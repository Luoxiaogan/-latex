\documentclass[10pt, a4paper, oneside]{ctexart}
\usepackage{amsmath, amsthm, amssymb, bm, color, xcolor, framed, graphicx, hyperref, mathrsfs, etoolbox, wrapfig, xurl, algorithm, algpseudocode}
\usepackage[thicklines]{cancel}
\usepackage{enumitem} % 用于更灵活的列表环境
\usepackage{geometry} % 调整页面边距
\usepackage{fancyhdr} % 页眉页脚定制
\hypersetup{
    colorlinks=true,            %链接颜色
    linkcolor=black,             %内部链接
    filecolor=magenta,          %本地文档
    urlcolor=cyan,              %网址链接
    pdftitle={Overleaf Example},
    pdfpagemode=FullScreen,
    }

% 页边距设置
\geometry{left=3cm, right=3cm, top=2.5cm, bottom=2.5cm}

% 页眉页脚设置
\pagestyle{fancy}
\fancyhf{}
\fancyhead[L]{\leftmark} % 左页眉显示章节标题
\fancyhead[R]{\thepage} % 右页眉显示页码

\newcommand{\norm}[1]{\| #1 \|}
\newcommand{\ip}[1]{\left\langle#1\right\rangle}

% 标题设置
\title{\textbf{计算方法B  HW 4}}
\author{罗淦  2200013522}
\date{\today}

% 行距设置
\linespread{1.2}

% 定义黑色边框的 problem 环境
\newenvironment{problem}{\begin{framed}\par\noindent\textbf{\textit{题目. }}}{\end{framed}\par}
\newenvironment{solution}{%
  \par\noindent\textbf{\textit{解答. }}\ignorespaces
}{%
  \hfill\ensuremath{\square}\par % 在结尾添加正方形
}
\newenvironment{note}{\par\noindent\textbf{\textit{题目的注记. }}\ignorespaces}{\par}


% 允许公式在页面之间自动换行
\allowdisplaybreaks

\begin{document}

\maketitle

% 添加目录
%\tableofcontents
%\newpage

\section{HW 4}
\begin{problem}
    1. 设方程组 $A x=b$ 的系数矩阵为

    $$
    A_1=\left[\begin{array}{rrr}
    2 & -1 & 1 \\
    1 & 1 & 1 \\
    1 & 1 & -2
    \end{array}\right], \quad A_2=\left[\begin{array}{rrr}
    1 & 2 & -2 \\
    1 & 1 & 1 \\
    2 & 2 & 1
    \end{array}\right]
    $$
    
    
    证明:对 $A_1$ 来说, Jacobi 迭代法不收敛,而 G-S 迭代法收敛;对 $A_2$ 来说, Jacobi 迭代法收敛, 而 G-S 迭代法不收敛。
\end{problem}
\begin{solution}
Jacobi迭代:
$$x_{k+1}=D^{-1}(L+U)x_k+D^{-1}b$$
G-S迭代:
$$x_{k+1}=(D-L)^{-1}Ux_k+D^{-1}b$$
对于$A_1$, \textcolor{red}{第一个容易出错的地方$A=D-L-U$, $L$和$U$是原来的$A$的不含对角的下三角部分和上三角部分的\textbf{负数值}}
\begin{align*}
    D^{-1}(L+U)&=\begin{pmatrix}
        0&1/2&-1/2\\
        -1&0&-1\\
        1/2&1/2&0\\
    \end{pmatrix}\Rightarrow |\lambda I-D^{-1}(L+U)|=\lambda(\lambda^2+5/4)\Rightarrow \rho(D^{-1}(L+U))=\sqrt{5/4}>1\\
   (D-L)^{-1}U&=\begin{pmatrix}
        0&1/2&-1/2\\
        0&-1/2&-1/2\\
        0&0&-1/2\\
    \end{pmatrix}\Rightarrow |\lambda I-(D-L)^{-1}U|=\lambda(\lambda+1/2)^2\Rightarrow \rho(D^{-1}(L+U))=\sqrt{1/2}<1
\end{align*}
所以对 $A_1$ 来说, Jacobi 迭代法不收敛,而 G-S 迭代法收敛\\
而对于$A_2$:
\begin{align*}
    D^{-1}(L+U)&=\begin{pmatrix}
        0&-2&2\\
        -1&0&-1\\
        -2&-2&0\\
    \end{pmatrix}\Rightarrow |\lambda I-D^{-1}(L+U)|=\lambda^3 \Rightarrow \rho(D^{-1}(L+U))=0<1\\
   (D-L)^{-1}U&=\begin{pmatrix}
        0&-2&2\\
        0&2&-3\\
        0&0&2\\
    \end{pmatrix}\Rightarrow |\lambda I-(D-L)^{-1}U|=\lambda(\lambda-2)^2\Rightarrow \rho(D^{-1}(L+U))=2>1
\end{align*}
所以对 $A_2$ 来说, Jacobi 迭代法收敛, 而 G-S 迭代法不收敛
\end{solution}

\begin{problem}
    2. 设 $B \in \mathbf{R}^{n \times n}$ 满足 $\rho(B)=0$. 证明: 对任意的 $g, x_0 \in \mathbf{R}^n$, 迭代格式

    $$
    x_{k+1}=B x_k+g, \quad k=0,1, \cdots
    $$
    
    
    最多迭代 $n$ 次就可得到方程组 $x=B x+g$ 的精确解.
\end{problem}
\begin{solution}
因为$\rho(B)=0$, 所以$B$的特征多项式是$f_B(\lambda)=\lambda^n$, 根据\textcolor{red}{Cayley-Hamilton定理, 矩阵$B$的特征多项式零化$B$, 或者说, 矩阵的极小多项式整除特征多项式, 因此存在$m\leq n, B^m=0$}.\\
所以有:
\begin{align*}
    x_{m}-x^*=B^m(x_0-x^*)=0, m\leq n
\end{align*}
因此最多$n$次就可以得到精确解.
\end{solution}

\begin{problem}
    3. 考虑线性方程组

    $$
    A x=b,
    $$
    
    
    这里
    
    $$
    A=\left[\begin{array}{lll}
    1 & 0 & a \\
    0 & 1 & 0 \\
    a & 0 & 1
    \end{array}\right]
    $$
    
    (1) $a$ 为何值时, $A$ 是正定的?
    (2) $a$ 为何值时, Jacobi 迭代法收敛?
    (3) $a$ 为何值时, G-S 迭代法收敛?
\end{problem}
\begin{solution}
(1) \textcolor{blue}{判断一个矩阵$A$是不是正定矩阵的方法: \\
1. 矩阵$A$的特征值全部都是正数\\
2. 矩阵$A$的各阶顺序主子式都是正数}\\
考虑矩阵$A$的各阶顺序主子式, 分别是$1,1,1-a^2$, 因此$a^2<1$的时候, $A$是正定矩阵.

(2) Jacobi迭代法, 分析矩阵$D^{-1}(L+U)$的谱半径:
\begin{align*}
    D^{-1}(L+U)=\begin{pmatrix}
        0&0&-a\\
        0&0&0\\
        -a&0&0\\
    \end{pmatrix}\Rightarrow |\lambda I-D^{-1}(L+U)|=\lambda(\lambda+a)(\lambda-a)
\end{align*}
因此$|a|<1$时Jacobi迭代收敛

(3) G-S迭代法, 分析矩阵$(D-L)^{-1}U$的谱半径:
\begin{align*}
    (D-L)^{-1}U=\begin{pmatrix}
        0&0&-a\\
        0&0&0\\
        0&0&a^2\\
    \end{pmatrix}\Rightarrow |\lambda I-(D-L)^{-1}U|=\lambda^2(\lambda-a^2)
\end{align*}
因此$|a|<1$时G-S迭代收敛
\end{solution}

\begin{problem}
    5. 若 $A$ 是严格对角占优的或不可约对角占优的, 则 G-S 迭代法收敛.
\end{problem}
\begin{solution}
\textcolor{red}{模仿书上的证明方法, 也就是, 假设有一个绝对值大于等于$1$的特征值, 考虑迭代矩阵的特征多项式, 可以证明这个特征多项式(行列式)不等于$0$, 从而与特征值的定义矛盾}.

考虑G-S迭代的迭代矩阵$(D-L)^{-1}U$存在特征值, 使得$|\lambda|\geq 1$, 那么考虑特征多项式$|\lambda I -(D-L)^{-1}U |$, 又因为:
\begin{align*}
    \lambda I-(D-L)^{-1}U=(D-L)^{-1}(\lambda D-\lambda L -U)
\end{align*}
且 
$$\lambda D-\lambda L -U=\lambda(D-L-U)+\underbrace{(\lambda-1)}_{|\lambda|\geq 1}U=\lambda A+(\lambda-1)U$$
因此$\lambda D-\lambda L -U$也是严格对角占优的或者不可约对角占优的, 因此$\lambda D-\lambda L -U$非奇异, 行列式不等于$0$. 从而得到 
\begin{align*}
    |\lambda I-(D-L)^{-1}U|
    =|(D-L)^{-1}|\cdot|\lambda D-\lambda L -U|\neq 0
\end{align*}
这与$\lambda$是$I-(D-L)^{-1}U$的特征值矛盾, 因此$I-(D-L)^{-1}U$的所有特征值的模长小于$1$, 因此G-S迭代收敛.
\end{solution}
\end{document}