\documentclass[10pt, a4paper, oneside]{ctexart}
\usepackage{amsmath, amsthm, amssymb, bm, color, xcolor, framed, graphicx, hyperref, mathrsfs, etoolbox, wrapfig, xurl, algorithm, algpseudocode}
\usepackage[thicklines]{cancel}
\usepackage{enumitem} % 用于更灵活的列表环境
\usepackage{geometry} % 调整页面边距
\usepackage{fancyhdr} % 页眉页脚定制
\hypersetup{
    colorlinks=true,            %链接颜色
    linkcolor=black,             %内部链接
    filecolor=magenta,          %本地文档
    urlcolor=cyan,              %网址链接
    pdftitle={Overleaf Example},
    pdfpagemode=FullScreen,
    }

% 页边距设置
\geometry{left=3cm, right=3cm, top=2.5cm, bottom=2.5cm}

% 页眉页脚设置
\pagestyle{fancy}
\fancyhf{}
\fancyhead[L]{\leftmark} % 左页眉显示章节标题
\fancyhead[R]{\thepage} % 右页眉显示页码

% 标题设置
\title{\textbf{HW 1}}
\author{罗淦  2200013522}
\date{\today}

% 行距设置
\linespread{1.2}

% 定义黑色边框的 problem 环境
\newenvironment{problem}{\begin{framed}\par\noindent\textbf{\textit{题目. }}}{\end{framed}\par}
\newenvironment{solution}{%
  \par\noindent\textbf{\textit{解答. }}\ignorespaces
}{%
  \hfill\ensuremath{\square}\par % 在结尾添加正方形
}
\newenvironment{note}{\par\noindent\textbf{\textit{题目的注记. }}\ignorespaces}{\par}


% 允许公式在页面之间自动换行
\allowdisplaybreaks

\begin{document}

\maketitle

% 添加目录
%\tableofcontents
%\newpage

\section{HW 1}
\begin{problem}
    1. 求下三角矩阵的逆矩阵的详细算法
\end{problem}
\begin{solution}
下三角矩阵$L$可逆, 因为行列式是对角元乘积, 因此所有对角元均不为零\\
考虑$LX=I$, 已知下三角矩阵的逆矩阵一定是下三角矩阵
\textcolor{red}{逐列进行计算}:\\
对$(j,j)$, 有$l_{jj}x_{jj}=1 \iff x_{jj}=\frac{1}{l_{jj}}$\\
读$(i,j)$, 是$L$的第$i$行的非零元是$(1:i)$, X的第$j$列的非零元是$(j:n)$, 不妨取$i>j$, 那么有:
\begin{align*}
    \sum_{k=j}^{i} l_{ik}x_{kj}=0 \iff x_{ij}=-\frac{\sum_{k=j}^{i-1}l_{ik}x_{kj}}{l_{ii}}
\end{align*}
我们已知$x_{jj}$, 取$i=j+1$, 得到$x_{j+1,j}=-\frac{l_{ij}x_{jj}}{l_{ii}}$\\
下面取$i=j+2$, 得到$x_{j+2,j}=-\frac{\sum_{k=j}^{j+1}l_{ik}x_{kj}}{l_{ii}}$, 依此下去, 得到$X$矩阵在第$j$列的$(j,n)$的元素, 注意$X$矩阵的$X(1:j-1, j)$都是$0$.\\
因此, 完整的算法如下:
\begin{algorithm}
    \caption{下三角矩阵求逆}
    \begin{algorithmic}
        \State \textbf{Input}: 满秩的下三角矩阵 \( L \)
        \State \textbf{Output}: 逆矩阵 \( L^{-1} \)
        \State 初始化 \( L^{-1} \), 全零矩阵
        \For{$j=1:n$}
        \State $x_{jj}=\frac{1}{l_{jj}}$
            \For{$i=j+1:n$}
            \State $x_{i,j}=-\frac{\sum_{k=j}^{i-1}l_{ik}x_{kj}}{l_{ii}}$
            \EndFor
        \EndFor
        \State \textbf{Return} \( L^{-1} \)
    \end{algorithmic}
    \end{algorithm} 
\end{solution}

\begin{problem}
    4. 确定一个$3\times 3$的高斯变换$L$, 使得
    $$L\begin{bmatrix}
        2\\3\\4
    \end{bmatrix}=\begin{bmatrix}
        2\\7\\8
    \end{bmatrix}$$
\end{problem}
\begin{solution}
第二行加上了第一行成二, 第三行加上了第一行成二, 因此
$$L=\begin{pmatrix}
1&0&0\\2&1&0\\2&0&1
\end{pmatrix} = I - l_1e_1^T, \quad l_1 = \begin{pmatrix}
    0\\-2\\-2
\end{pmatrix}$$
\end{solution}

\begin{problem}
    5. 如果$A\in \mathbb{R}^{n\times n}$有三角分解, 并且都是非奇异的, 那么$A=LU$分解得到的$L$和$U$是唯一的.
\end{problem}
\begin{solution}
不妨假设分解是不唯一的, 有非奇异单位下三角矩阵$L_1, L_2$, 非奇异上三角矩阵$U_1,U_2$\\
使得$L_1U_1=L_2U_2 \iff L_2^{-1}L_1=U_2U_1^{-1}$, $L_2^{-1}L_1$是单位下三角, $U_2U_1^{-1}$是上三角\\因此$L_2^{-1}L_1=U_2U_1^{-1}=I\iff L_1=L_2,U_1=U_2$
\end{solution}

\begin{problem}
    8.设$A=[a_{ij}]\in \mathbb{R}^{n\times n}$是严格对角占优矩阵, 即
    \begin{align*}
        |a_{kk}|>\sum_{j=1, j\neq k}^{n} |a_{kj}|,\quad k=1,\cdots,n
    \end{align*}
    又假设经过一步Gauss消去之后, $A$有如下形状:
    $$\begin{bmatrix}
        a_{11}& a_1^T\\0&A_2
    \end{bmatrix}$$
    证明: 矩阵$A_2$仍然是严格对角占优矩阵. 由此推断, 对于严格对角占优矩阵来说, 用Gauss消去法和列主元Gauss消去法可以得到同样的结果.
\end{problem}
\begin{solution}
设矩阵
$$A=\begin{pmatrix}
    a_{11}&\alpha^T\\\beta&A_{22}
\end{pmatrix}$$
那么做一步Gauss消元就是左乘$L = I - l_1e_1^T$
\begin{align*}
    (I - l_1e_1^T)A = \begin{pmatrix}
       1&0\\ -l_1 & I_{n-1} 
    \end{pmatrix} \begin{pmatrix}
        a_{11}&\alpha^T\\\beta&A_{22}
    \end{pmatrix}=\begin{pmatrix}
        a_{11}&\alpha^T\\ \beta-a_{11}l_1& A_{22}-l_1\alpha^T
    \end{pmatrix}
\end{align*}
通过$\beta-a_{11}l_1=0$, 得到$\beta=a_{11}l_1$, 因此
\begin{align*}
    \begin{pmatrix}
        a_{11}&\alpha^T\\ \beta-a_{11}l_1& A_{22}-l_1\alpha^T
    \end{pmatrix}=\begin{pmatrix}
        a_{11}&\alpha^T\\ 0& A_{22}-\frac{1}{a_{11}}\beta\alpha^T
    \end{pmatrix}:= A^{(1)}
\end{align*}
因此$A^{(1)}(i,j)=a_{ij}^{\prime}=a_{ij}-\frac{a_{i1}a_{1j}}{a_{11}}$, $\forall 1\leq i,j\leq n$.\\
因此对于矩阵$A_2$, 考虑$2\leq k\leq n$, 有
\begin{align*}
    \sum_{j=2, j\neq k}^n |a_{kj}^{\prime}|& = \sum_{j=2, j\neq k}^n|a_{kj}-\frac{a_{k1}a_{1j}}{a_{11}} |\leq \sum_{j=2, j\neq k}^n|a_{kj}| + |\frac{a_{k1}}{a_{11}}|\sum_{j=2, j\neq k}^n |a_{1j}|\\
    &< |a_{kk}| - a_{k1} +  |\frac{a_{k1}}{a_{11}}| ( |a_{11}| - |a_{1k} |  ) \\&= |a_{kk}| - |\frac{a_{k1}a_{1k}}{a_{11}}| \leq |a_{kk}-\frac{a_{k1}a_{1k}}{a_{11}}| = |a_{kk}^{\prime}|
\end{align*}
因此$A_2$是严格对角占优的.\\
因此, 在高斯消去法的第$k-1$步后, 因为右下角的矩阵是严格对角占优的, 所以$A^{(k-1)}$第$k$列的$k$之后绝对值最大元素就是$|a_{kk}^{(k-1)}|$, 因此列主元得到的结果是不交换, 即与正常Gauss消去法得到一样的结果.
\end{solution}

\begin{problem}
10. $A$是正定矩阵, 对$A$执行一步Gauss消去得到:
\begin{align*}
    \begin{pmatrix}
        a_{11}& a_1^T \\ 0 &A_2
    \end{pmatrix}
\end{align*}
证明: $A_2$是正定矩阵.
\end{problem}
\begin{solution}
\textcolor{blue}{$A$是正定矩阵, 但是$A$不一定是对称矩阵\\但由于$x^TAx = x^TA^Tx=0$(since 这是一个标量), 所以$x^T(\frac{A-A^T}{2})x=0$, 因此$x^T(\frac{A+A^T}{2})x=0$\\ 反过来, $x^T(\frac{A+A^T}{2})x=0$, 那么$x^T(\frac{A+A^T}{2})x=0$, 有$x^T(\frac{A+A^T}{2})x=x^TAx=0$\\ 因此我们不妨假设正定矩阵$A$是对称的.}\\
对于正定矩阵$A$, 经过一步Gauss消元之后得到:
\begin{align*}
    A=\begin{pmatrix}
        a_{11} & \alpha^T \\ \alpha & A_{22} 
    \end{pmatrix} \Rightarrow LA = \begin{pmatrix}
        a_{11}&\alpha^T \\ 0 & A_{22} - \frac{1}{a_{11}}\alpha \alpha^T
    \end{pmatrix}
\end{align*}
要证明$A_2$正定, 即:
\begin{align*}
    x^T(A_{22} - \frac{1}{a_{11}}\alpha \alpha^T)x=0,\quad \forall x\in \mathbb{R}^{n-1}
\end{align*}     
根据条件: (\textcolor{blue}{注意, 这里可以取$\sqrt{a_{11}}$是因为$A$是正定矩阵, 所有对角元都是正的})
\begin{align*}
    &\begin{pmatrix}
        y&x
    \end{pmatrix} \begin{pmatrix}
        a_{11} & \alpha^T \\ \alpha & A_{22} 
    \end{pmatrix} \begin{pmatrix}
        y\\x
    \end{pmatrix}=a_{11}y^2+ 2yx^T\alpha +x^TA_{22}x\\
    =& a_{11}y^2+ 2yx^T\alpha + \frac{1}{a_{11}}(x^T\alpha )^2 +x^T(A_{22} - \frac{1}{a_{11}}\alpha \alpha^T)x\\
    =& (\sqrt{a_{11}}y + \frac{x^T\alpha}{\sqrt{a_{11}}})^2 + x^T(A_{22} - \frac{1}{a_{11}}\alpha \alpha^T)x \geq 0
\end{align*}
取$y = - \frac{x^T\alpha}{a_{11}}$, 得到:
\begin{align*}
    &(\sqrt{a_{11}}y + \frac{x^T\alpha}{\sqrt{a_{11}}})^2 + x^T(A_{22} - \frac{1}{a_{11}}\alpha \alpha^T)x\\
    =&x^T(A_{22} - \frac{1}{a_{11}}\alpha \alpha^T)x\geq 0
\end{align*}
且取$0$当且仅当$x=0$, 得证.
\end{solution}

\begin{problem}
14. 假定已知$A\in \mathbb{R}^{n\times n}$的三角分解$A=LU$, 设计算法来计算$A^{-1}$.
\end{problem}
\begin{solution}
形式上: $A^{-1} = U^{-1} L^{-1}$\\
\textcolor{red}{\textbf{方法一}}: 根据第1题的算法, 我们有了下三角矩阵求逆的算法, 那么可以得到$L^{-1}$, 以及$(U^{T})^{-1}=(U^{-1})^T$\\自然就得到了$A^{-1} = ((U^{T})^{-1})^T L^{-1}$.\\\textcolor{blue}{这样的计算复杂度是$O(n^3)+O(n^3)+O(n^3)=O(n^3)$, 注意加法的最后一个是矩阵乘法}\\
\textcolor{red}{\textbf{方法二}}: 因为$AA^{-1}=I$, 那么$A^{-1}$的第$j$列是$Ax_j=e_j$的解, 即$LUx_j=e_j$的解, 那么先解$Ly=e_j$, 再解$Ux_j=y$就可以解出$x_j$.\\
\textcolor{blue}{这样的计算复杂度是$n\cdot O(n^2) = O(n^3)$, 因为前代法和回代法的复杂度都是$O(n^2)$}\\
\textcolor{blue}{绷:题目是求$A^{-1}(i:j)$}, 那么直接用方法二即可:\\
$Ax_j=e_j$的解向量的地$i$个元素, $O(n^2)$
\end{solution}

\begin{problem}
19. 若$A=LL^T$是$A$的Cholesky分解, 试证: $L$的$i$阶顺序主子阵$L_i$正好是$A$的$i$阶顺序主子阵$A_i$的Cholesky因子.
\end{problem}

\begin{solution}
根据Cholesky分解, 有:
\begin{align*}
    A = LL^T = \begin{pmatrix}
        L_i & 0\\ L_{21} & L_{22}
    \end{pmatrix} \begin{pmatrix}
        L_i^T & L_{21}^T\\ 0 & L_{22}^T
    \end{pmatrix}=\begin{pmatrix}
        L_iL_i^T & L_i L_{21}^T \\ L_{21}L_i^T & L_{22}L_{22}^T
    \end{pmatrix}
\end{align*}
因此自然有: $A_i = L_i L_i^T$
\end{solution}

\end{document}