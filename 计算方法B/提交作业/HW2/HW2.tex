\documentclass[10pt, a4paper, oneside]{ctexart}
\usepackage{amsmath, amsthm, amssymb, bm, color, xcolor, framed, graphicx, hyperref, mathrsfs, etoolbox, wrapfig, xurl, algorithm, algpseudocode}
\usepackage[thicklines]{cancel}
\usepackage{enumitem} % 用于更灵活的列表环境
\usepackage{geometry} % 调整页面边距
\usepackage{fancyhdr} % 页眉页脚定制
\hypersetup{
    colorlinks=true,            %链接颜色
    linkcolor=black,             %内部链接
    filecolor=magenta,          %本地文档
    urlcolor=cyan,              %网址链接
    pdftitle={Overleaf Example},
    pdfpagemode=FullScreen,
    }

% 页边距设置
\geometry{left=3cm, right=3cm, top=2.5cm, bottom=2.5cm}

% 页眉页脚设置
\pagestyle{fancy}
\fancyhf{}
\fancyhead[L]{\leftmark} % 左页眉显示章节标题
\fancyhead[R]{\thepage} % 右页眉显示页码

\newcommand{\norm}[1]{\| #1 \|}
\newcommand{\ip}[1]{\left\langle#1\right\rangle}

% 标题设置
\title{\textbf{HW 2}}
\author{罗淦  2200013522}
\date{\today}

% 行距设置
\linespread{1.2}

% 定义黑色边框的 problem 环境
\newenvironment{problem}{\begin{framed}\par\noindent\textbf{\textit{题目. }}}{\end{framed}\par}
\newenvironment{solution}{%
  \par\noindent\textbf{\textit{解答. }}\ignorespaces
}{%
  \hfill\ensuremath{\square}\par % 在结尾添加正方形
}
\newenvironment{note}{\par\noindent\textbf{\textit{题目的注记. }}\ignorespaces}{\par}


% 允许公式在页面之间自动换行
\allowdisplaybreaks

\begin{document}

\maketitle

% 添加目录
%\tableofcontents
%\newpage

\section{HW 2}
\begin{problem}
2. 证明: 当且仅当$x$和$y$线性相关且$x^Ty\geq 0$时, 才有:
\begin{align*}
    || x+y||_2 = ||x||_2+||y||_2
\end{align*}
\end{problem}

\begin{solution}
$(\Rightarrow)$: 已知$y=kx, x^Ty\geq 0$, 那么$y=kx, k\geq 0$, 那么等式成立:
\begin{align*}
    ||x+y||_2=(1+k)||x||_2=||x||_2+||y||_2
\end{align*}
$(\Leftarrow)$: 已知等式成立, \textcolor{red}{考虑$y$关于$x$的正交分解$y=kx+(y-kx), k=\frac{x^Ty}{x^Tx} $}, 又因为等式成立, 两边平方后代入:
\begin{align*}
    \norm{x+y}_2^2&=(1+k)^2\norm{x}_2^2+\norm{y-kx}_2^2\\
    \norm{x}_2^2+\norm{y}_2^2+2\norm{x}_2\norm{y}_2&=\norm{x}_2^2+k^2\norm{x}_2^2+\norm{y-kx}_2^2+2\norm{x}_2\norm{y}_2\\
    || x+y||_2 = ||x||_2+||y||_2&\Rightarrow k\norm{x}_2=\norm{kx+y-kx}_2\\
    &\Rightarrow \norm{y-kx}_2^2=0\iff y=kx
\end{align*}
因此$x$和$y$线性相关, 又因为:
\begin{align*}
    \norm{x+y}_2=|1+k|\norm{x}_2=(1+|k|)\norm{x}_2
\end{align*}
故$k\geq 0$, 即$x^Ty\geq 0$.
\end{solution}

\begin{problem}
    3. 证明: 如果$A=[a_1,\cdots,a_n]$是按列分块的, 那么 
    \begin{align*}
        \norm{A}_F^2=\norm{a_1}_2^2+\cdots+\norm{a_n}_2^2
    \end{align*}
\end{problem}

\begin{solution}
\begin{align*}
    \norm{A}_2=\sum_{j=1}^n(\sum_{i=1}^n a_{ij}^2 )=\sum_{j=1}^n \norm{a_j}_2^2
\end{align*}
得证
\end{solution}

\begin{problem}
4. 证明: 
\begin{align*}
    \norm{AB}_F&\leq \norm{A}_2\norm{B}_F\\
    \norm{AB}_F&\leq \norm{A}_F\norm{B}_2
\end{align*}
\end{problem}

\begin{solution}
\textcolor{red}{\textbf{方法一}}: 考虑$\norm{AB}^2_F=\text{tr}(B^*A^*AB^*)$\\
对于Hermite矩阵$A^*A$, 首先, 它的对角元都是非负实数(因为第$i$个对角元是$A$的第$i$列的模平方)\\
其次, 根据谱定理, 它可以酉对角化, 即存在酉矩阵$U$, 使得$A^*A=U^*DU$, $D$是对角矩阵, 且对角元都是非负实数.\\
因此有:
\begin{align*}
    \norm{AB}_F^2&=\text{tr}(B^*A^*AB)=\text{tr}(B^*U^*DUB)=\text{tr}(DB^*U^*UB)\\&\leq \max_{i}(d_{ii})\text{tr}(B^*U^*UB)=\max_{i}(d_{ii})\text{tr}(U^*B^*BU)\\&=\max_{i}(d_{ii})\text{tr}(B^*B)=\norm{A}_2^2\norm{B}_F^2
\end{align*}
同理有:
\begin{align*}
    \norm{AB}_F^2&=\text{tr}(B^*A^*AB)=\text{tr}(A^*B^*BA)=\text{tr}(A^*\tilde{U}^*\tilde{D}\tilde{U}A)=\text{tr}(\tilde{D}A^*\tilde{U}^*\tilde{U}A)\\&\leq \max_{i}(\tilde{d}_{ii})\text{tr}(A^*\tilde{U}^*\tilde{U}A)=\max_{i}(\tilde{d}_{ii})\text{tr}(\tilde{U}^*A^*A\tilde{U})\\&=\max_{i}(\tilde{d}_{ii})\text{tr}(A^*A)=\norm{B}_2^2\norm{A}_F^2=\norm{A}_F^2\norm{B}_2^2
\end{align*}
上面的方法用到的性质有:
\begin{itemize}
    \item 矩阵成绩的trace中, 乘积可以互换
    \item 矩阵的trace在相似变化下保持不变, 因为$\text{tr}(U^*CU)=\text{tr}(U^*UC)=\text{tr}(C)$.
\end{itemize}

\textcolor{red}{\textbf{方法二}}: 

考虑$\norm{AB}_F^2=\sum_{i=1}^n \norm{AB[i,:]}_2^2$, \textcolor{blue}{而$AB$的第$i$行就是$A$的第$i$行乘$B$: $AB[i,:]=\alpha_i^TB$}, 因此
\begin{align*}
    \norm{AB}_F^2=\sum_{i=1}^n \norm{AB[i,:]}_2^2=\sum_{i=1}^n \norm{\alpha_i^TB}_2^2\leq \sum_{i=1}^n \norm{\alpha_i^T}_2^2\norm{B}_2^2=\norm{A}_F^2\norm{B}_2^2
\end{align*}
换一个角度, \textcolor{blue}{$AB$的第$j$行就是$A$乘上$B$的第$j$列: $AB[:,j]=A\beta_j$}, 因此:
\begin{align*}
    \norm{AB}_F^2=\sum_{j=1}^n \norm{AB[:,j]}_2^2=\sum_{j=1}^n \norm{A\beta_j}_2^2\leq \sum_{j=1}^n \norm{A}_2^2\norm{\beta_j}_2^2=\norm{A}_2^2\norm{B}_F^2
\end{align*}
这里用到了:
\begin{itemize}
    \item 矩阵范数和向量范数的相容性: $\norm{Ax}_2\leq \norm{A}_2\norm{x}_2$
\end{itemize}
\end{solution}

\begin{note}
(1) $F$范数(平方)的两种表示: $\norm{A}_F^2=\sum_{i,j}|a_{ij}|^2=\textcolor{blue}{\text{tr}(A^TA) }$, 注意在复矩阵的情况下为$\textcolor{blue}{\text{tr}(A^*A)}$, 共轭转置.\\
(2) 共轭转置不变的复矩阵, 即Hermite矩阵, 是可以酉对角化的(即, 实矩阵意义下的正交对角化)\\
(3) $A^TA$的特征值都是非负实数
\end{note}

\begin{problem}
    8. 若$\norm{A}<1$, 且$\norm{I}=1$, 证明:
    \begin{align*}
        \norm{(I-A)^{-1}}\leq \frac{1}{1-\norm{A}}
    \end{align*}
\end{problem}

\begin{solution}
因为$\norm{A}<1$, 且任意的矩阵范数都大于等于谱半径, 因此$\rho(A)\leq \norm{A}<1$\\
因为$\rho(A)<1 \iff \sum_{k=0}^{\infty}A^k$收敛, 因此$ \sum_{k=0}^{\infty}A^k$存在\\
因为$(I-A)(I+A+\cdots+A^k)=I-A^{k+1}$, 且$\norm{A^{k+1}}\leq \norm{A}^{k+1}\to 0 \Rightarrow A^{k+1}\to \bf{0}$\\
因此可以证明: $(I-A)\sum_{k=0}^{\infty}A^k=I$, 即$I-A$可逆, $(I-A)^{-1}=\sum_{k=0}^{\infty}A^k$\\
注意题目条件\textcolor{red}{$\norm{I}=1$}, 我们有
\begin{align*}
    \norm{(I-A)^{-1}}=\norm{\sum_{k=0}^{\infty}A^k}\leq \sum_{k=0}^{\infty}\norm{A^k}\leq \sum_{k=0}^{\infty}\norm{A}^k=\frac{\textcolor{red}{\norm{I}}\cdot(1-\lim_{k\to \infty}\norm{A}^k)}{1-\norm{A}}=\frac{1}{1-\norm{A}}
\end{align*}
\end{solution}
\begin{note}
需要记住的结论:
\begin{itemize}
    \item $A$的任意矩阵范数大于等于谱半径: $\rho(A)\leq \norm{A}$, \textcolor{blue}{即, 这一条对任意矩阵成立}
    \item $A$给定, 谱半径是矩阵(算子)范数的下确界, $\forall \epsilon>0$, 存在算子范数, 使得$\norm{A}\leq \rho(A)+\epsilon$
    \item $\lim_{k\to \infty}A^k=0 \iff \sum_{k=0}^{\infty}A^k \;\text{收敛}\iff \rho(A)<1$, 若收敛, $(I-A)^{-1}=\sum_{k=0}^{\infty}A^k$
\end{itemize}
\end{note}

\begin{problem}
11. 设 
\begin{align*}
    A=\begin{bmatrix}
        375&374\\752&750
    \end{bmatrix}
\end{align*}
(1) 计算$A^{-1}$和$\kappa_{\infty}(A)$\\
(2) 选择$b,\delta b, x, \delta x$, 使得
\begin{align*}
    Ax=b, \quad A(x+\delta x)=b+\delta b
\end{align*}
而且$\frac{\norm{\delta b}_{\infty}}{\norm{ b}_{\infty}}$很小, 但$\frac{\norm{\delta x}_{\infty}}{\norm{x}_{\infty}}$很大\\
(2) 选择$b,\delta b, x, \delta x$, 使得
\begin{align*}
    Ax=b, \quad A(x+\delta x)=b+\delta b
\end{align*}
而且$\frac{\norm{\delta x}_{\infty}}{\norm{ x}_{\infty}}$很小, 但$\frac{\norm{\delta b}_{\infty}}{\norm{b}_{\infty}}$很大
\end{problem}

\begin{solution}
观察矩阵, 设$a=374$, 那么:
\begin{align*}
    A = \begin{bmatrix}
        a&a-1\\
        2(a+1)&2a
    \end{bmatrix}
\end{align*}
记住了\textcolor{blue}{二阶矩阵求逆的标准公式:
\begin{align*}
    \begin{pmatrix}
        a&b\\c&d
    \end{pmatrix}^{-1}=\frac{1}{|ad-bc|}\begin{pmatrix}
        d&-b \\-c &a
    \end{pmatrix}
\end{align*}}
因此有:
\begin{align*}
    A^{-1}=\frac{1}{2}\begin{pmatrix}
        2a&1-a\\-2a-2&a
    \end{pmatrix}=\begin{pmatrix}
        375&-187\\-376&187.5
    \end{pmatrix}
\end{align*}
计算$\kappa_{\infty}(A)=\norm{A}_{\infty}\norm{A^{-1}}_{\infty}$, 有(行范数计算)
\begin{align*}
    \kappa_{\infty}(A)=\norm{A}_{\infty}\norm{A^{-1}}_{\infty}=(752+750)\cdot(376+187.5)= 846377
\end{align*}

(2) 考虑先进行敏度分析, $A(x+\delta x)=b+\delta b$, 因为$Ax=b$, 所以$\delta x=A^{-1}\delta b$, 因此有$\norm{\delta x}_{\infty}\leq \norm{A^{-1}}_{\infty}\norm{\delta b}_{\infty}$, 两边同时除以$\norm{b}_{\infty}\leq \norm{A}_{\infty}\norm{x}_{\infty}$, 有:
\begin{align*}
    &\frac{\norm{\delta x}_{\infty}}{\norm{A}_{\infty}\norm{x}_{\infty}} \leq \frac{\norm{\delta x}_{\infty}}{\norm{b}_{\infty}}\leq \norm{A^{-1}}_{\infty}\frac{\norm{\delta b}_{\infty}}{\norm{b}_{\infty}}\\
    \Rightarrow& \frac{\norm{\delta x}_{\infty}}{\norm{x}_{\infty}} \leq \norm{A}_{\infty}\norm{A^{-1}}_{\infty}\frac{\norm{\delta b}_{\infty}}{\norm{b}_{\infty}}=\kappa(A)\frac{\norm{\delta b}_{\infty}}{\norm{b}_{\infty}}
\end{align*}
虽然我们已知$\kappa(A)$很大, 但这似乎对我们的构造没有什么帮助. 因为这个不等式是一个关于上下界的估计, 但实际值可以不在等号附近.

\textcolor{red}{好的思路: 因为矩阵$A$的条件数很大(实际上是$\norm{A}_{\infty}$和$\norm{A^{-1}}_{\infty}$都不小), 所以对于$A\delta x=\delta b$(已知$\delta x$, 算$\delta b$)和$\delta x=A^{-1}\delta b$(已知$\delta b$, 算$\delta x$), 即使已知的对象比较小, 被计算的部分也会被放得很大}.

取$x=\begin{pmatrix}
    1\\-1
\end{pmatrix}$, 那么$b=Ax=\begin{pmatrix}
    1\\2
\end{pmatrix}$, \textcolor{blue}{先取$\delta b = \begin{pmatrix}
\epsilon\\0
\end{pmatrix} $ }, 计算得到$\delta x =A^{-1}\delta b=\begin{pmatrix}
    375\epsilon\\-376\epsilon
\end{pmatrix} $, 那么($\epsilon>0$) 
\begin{align*}
    &\frac{\norm{\delta b}_{\infty}}{\norm{ b}_{\infty}}=\frac{\epsilon}{2}, \frac{\norm{\delta x}_{\infty}}{\norm{ x}_{\infty}}=\frac{376\epsilon}{1}=376\epsilon\\
    \epsilon=1\Rightarrow& \frac{\norm{\delta b}_{\infty}}{\norm{ b}_{\infty}}=\frac{1}{2}, \frac{\norm{\delta x}_{\infty}}{\norm{ x}_{\infty}}=376
\end{align*}

(3) 类似于(2), 取$x=\begin{pmatrix}
    1\\-1
\end{pmatrix}$, 那么$b=Ax=\begin{pmatrix}
    1\\2
\end{pmatrix}$, \textcolor{blue}{先取$\delta x = \begin{pmatrix}
    \epsilon\\0
    \end{pmatrix} $ }, 计算$\delta b = A\delta x=\begin{pmatrix}
        375\epsilon\\ 752\epsilon
    \end{pmatrix} $, 那么($\epsilon>0$) 
\begin{align*}
    &\frac{\norm{\delta b}_{\infty}}{\norm{ b}_{\infty}}=\frac{752\epsilon}{2}, \frac{\norm{\delta x}_{\infty}}{\norm{ x}_{\infty}}=\frac{\epsilon}{1}=\epsilon\\
    \epsilon=1\Rightarrow& \frac{\norm{\delta b}_{\infty}}{\norm{ b}_{\infty}}=376, \frac{\norm{\delta x}_{\infty}}{\norm{ x}_{\infty}}=1
\end{align*}
\end{solution}

\end{document}