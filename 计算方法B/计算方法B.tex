\documentclass[10pt, a4paper, oneside]{ctexart}
\usepackage{amsmath, amsthm, amssymb, bm, color, xcolor, framed, graphicx, hyperref, mathrsfs, etoolbox, wrapfig, xurl, algorithm, algpseudocode}
\usepackage[thicklines]{cancel}
\usepackage{enumitem} % 用于更灵活的列表环境
\usepackage{geometry} % 调整页面边距
\usepackage{fancyhdr} % 页眉页脚定制
\hypersetup{
    colorlinks=true,            %链接颜色
    linkcolor=black,             %内部链接
    filecolor=magenta,          %本地文档
    urlcolor=cyan,              %网址链接
    pdftitle={Overleaf Example},
    pdfpagemode=FullScreen,
    }

% 页边距设置
\geometry{left=3cm, right=3cm, top=2.5cm, bottom=2.5cm}

% 页眉页脚设置
\pagestyle{fancy}
\fancyhf{}
\fancyhead[L]{\leftmark} % 左页眉显示章节标题
\fancyhead[R]{\thepage} % 右页眉显示页码

\newcommand{\norm}[1]{\| #1 \|}
\newcommand{\ip}[1]{\left\langle#1\right\rangle}

% 标题设置
\title{\textbf{计算方法B}}
\author{Little Wolf}
\date{\today}

% 行距设置
\linespread{1.2}

% 定义黑色边框的 problem 环境
\newenvironment{problem}{\begin{framed}\par\noindent\textbf{\textit{题目. }}}{\end{framed}\par}
\newenvironment{solution}{%
  \par\noindent\textbf{\textit{解答. }}\ignorespaces
}{%
  \hfill\ensuremath{\square}\par % 在结尾添加正方形
}
\newenvironment{note}{\par\noindent\textbf{\textit{题目的注记. }}\ignorespaces}{\par}


% 允许公式在页面之间自动换行
\allowdisplaybreaks

\begin{document}

\maketitle

% 添加目录
\tableofcontents
\newpage

\section{线性方程组的直接解法}
\begin{problem}
    1. 求下三角矩阵的逆矩阵的详细算法
\end{problem}
\begin{solution}
下三角矩阵$L$可逆, 因为行列式是对角元乘积, 因此所有对角元均不为零\\
考虑$LX=I$, 已知下三角矩阵的逆矩阵一定是下三角矩阵
\textcolor{red}{逐列进行计算}:\\
对$(j,j)$, 有$l_{jj}x_{jj}=1 \iff x_{jj}=\frac{1}{l_{jj}}$\\
读$(i,j)$, 是$L$的第$i$行的非零元是$(1:i)$, X的第$j$列的非零元是$(j:n)$, 不妨取$i>j$, 那么有:
\begin{align*}
    \sum_{k=j}^{i} l_{ik}x_{kj}=0 \iff x_{ij}=-\frac{\sum_{k=j}^{i-1}l_{ik}x_{kj}}{l_{ii}}
\end{align*}
我们已知$x_{jj}$, 取$i=j+1$, 得到$x_{j+1,j}=-\frac{l_{ij}x_{jj}}{l_{ii}}$\\
下面取$i=j+2$, 得到$x_{j+2,j}=-\frac{\sum_{k=j}^{j+1}l_{ik}x_{kj}}{l_{ii}}$, 依此下去, 得到$X$矩阵在第$j$列的$(j,n)$的元素, 注意$X$矩阵的$X(1:j-1, j)$都是$0$.\\
因此, 完整的算法如下:
\begin{algorithm}
    \caption{下三角矩阵求逆}
    \begin{algorithmic}
        \State \textbf{Input}: 满秩的下三角矩阵 \( L \)
        \State \textbf{Output}: 逆矩阵 \( L^{-1} \)
        \State 初始化 \( L^{-1} \), 全零矩阵
        \For{$j=1:n$}
        \State $x_{jj}=\frac{1}{l_{jj}}$
            \For{$i=j+1:n$}
            \State $x_{i,j}=-\frac{\sum_{k=j}^{i-1}l_{ik}x_{kj}}{l_{ii}}$
            \EndFor
        \EndFor
        \State \textbf{Return} \( L^{-1} \)
    \end{algorithmic}
    \end{algorithm} 
\end{solution}

\begin{problem}
    4. 确定一个$3\times 3$的高斯变换$L$, 使得
    $$L\begin{bmatrix}
        2\\3\\4
    \end{bmatrix}=\begin{bmatrix}
        2\\7\\8
    \end{bmatrix}$$
\end{problem}
\begin{solution}
第二行加上了第一行成二, 第三行加上了第一行成二, 因此
$$L=\begin{pmatrix}
1&0&0\\2&1&0\\2&0&1
\end{pmatrix} = I - l_1e_1^T, \quad l_1 = \begin{pmatrix}
    0\\-2\\-2
\end{pmatrix}$$
\end{solution}

\begin{problem}
    5. 如果$A\in \mathbb{R}^{n\times n}$有三角分解, 并且都是非奇异的, 那么$A=LU$分解得到的$L$和$U$是唯一的.
\end{problem}
\begin{solution}
不妨假设分解是不唯一的, 有非奇异单位下三角矩阵$L_1, L_2$, 非奇异上三角矩阵$U_1,U_2$\\
使得$L_1U_1=L_2U_2 \iff L_2^{-1}L_1=U_2U_1^{-1}$, $L_2^{-1}L_1$是单位下三角, $U_2U_1^{-1}$是上三角\\因此$L_2^{-1}L_1=U_2U_1^{-1}=I\iff L_1=L_2,U_1=U_2$
\end{solution}

\begin{problem}
    8.设$A=[a_{ij}]\in \mathbb{R}^{n\times n}$是严格对角占优矩阵, 即
    \begin{align*}
        |a_{kk}|>\sum_{j=1, j\neq k}^{n} |a_{kj}|,\quad k=1,\cdots,n
    \end{align*}
    又假设经过一步Gauss消去之后, $A$有如下形状:
    $$\begin{bmatrix}
        a_{11}& a_1^T\\0&A_2
    \end{bmatrix}$$
    证明: 矩阵$A_2$仍然是严格对角占优矩阵. 由此推断, 对于严格对角占优矩阵来说, 用Gauss消去法和列主元Gauss消去法可以得到同样的结果.
\end{problem}
\begin{solution}
设矩阵
$$A=\begin{pmatrix}
    a_{11}&\alpha^T\\\beta&A_{22}
\end{pmatrix}$$
那么做一步Gauss消元就是左乘$L = I - l_1e_1^T$
\begin{align*}
    (I - l_1e_1^T)A = \begin{pmatrix}
       1&0\\ -l_1 & I_{n-1} 
    \end{pmatrix} \begin{pmatrix}
        a_{11}&\alpha^T\\\beta&A_{22}
    \end{pmatrix}=\begin{pmatrix}
        a_{11}&\alpha^T\\ \beta-a_{11}l_1& A_{22}-l_1\alpha^T
    \end{pmatrix}
\end{align*}
通过$\beta-a_{11}l_1=0$, 得到$\beta=a_{11}l_1$, 因此
\begin{align*}
    \begin{pmatrix}
        a_{11}&\alpha^T\\ \beta-a_{11}l_1& A_{22}-l_1\alpha^T
    \end{pmatrix}=\begin{pmatrix}
        a_{11}&\alpha^T\\ 0& A_{22}-\frac{1}{a_{11}}\beta\alpha^T
    \end{pmatrix}:= A^{(1)}
\end{align*}
因此$A^{(1)}(i,j)=a_{ij}^{\prime}=a_{ij}-\frac{a_{i1}a_{1j}}{a_{11}}$, $\forall 1\leq i,j\leq n$.\\
因此对于矩阵$A_2$, 考虑$2\leq k\leq n$, 有
\begin{align*}
    \sum_{j=2, j\neq k}^n |a_{kj}^{\prime}|& = \sum_{j=2, j\neq k}^n|a_{kj}-\frac{a_{k1}a_{1j}}{a_{11}} |\leq \sum_{j=2, j\neq k}^n|a_{kj}| + |\frac{a_{k1}}{a_{11}}|\sum_{j=2, j\neq k}^n |a_{1j}|\\
    &< |a_{kk}| - a_{k1} +  |\frac{a_{k1}}{a_{11}}| ( |a_{11}| - |a_{1k} |  ) \\&= |a_{kk}| - |\frac{a_{k1}a_{1k}}{a_{11}}| \leq |a_{kk}-\frac{a_{k1}a_{1k}}{a_{11}}| = |a_{kk}^{\prime}|
\end{align*}
因此$A_2$是严格对角占优的.\\
因此, 在高斯消去法的第$k-1$步后, 因为右下角的矩阵是严格对角占优的, 所以$A^{(k-1)}$第$k$列的$k$之后绝对值最大元素就是$|a_{kk}^{(k-1)}|$, 因此列主元得到的结果是不交换, 即与正常Gauss消去法得到一样的结果.
\end{solution}

\begin{problem}
10. $A$是正定矩阵, 对$A$执行一步Gauss消去得到:
\begin{align*}
    \begin{pmatrix}
        a_{11}& a_1^T \\ 0 &A_2
    \end{pmatrix}
\end{align*}
证明: $A_2$是正定矩阵.
\end{problem}
\begin{solution}
\textcolor{blue}{$A$是正定矩阵, 但是$A$不一定是对称矩阵\\但由于$x^TAx = x^TA^Tx=0$(since 这是一个标量), 所以$x^T(\frac{A-A^T}{2})x=0$, 因此$x^T(\frac{A+A^T}{2})x=0$\\ 反过来, $x^T(\frac{A+A^T}{2})x=0$, 那么$x^T(\frac{A+A^T}{2})x=0$, 有$x^T(\frac{A+A^T}{2})x=x^TAx=0$\\ 因此我们不妨假设正定矩阵$A$是对称的.}\\
对于正定矩阵$A$, 经过一步Gauss消元之后得到:
\begin{align*}
    A=\begin{pmatrix}
        a_{11} & \alpha^T \\ \alpha & A_{22} 
    \end{pmatrix} \Rightarrow LA = \begin{pmatrix}
        a_{11}&\alpha^T \\ 0 & A_{22} - \frac{1}{a_{11}}\alpha \alpha^T
    \end{pmatrix}
\end{align*}
要证明$A_2$正定, 即:
\begin{align*}
    x^T(A_{22} - \frac{1}{a_{11}}\alpha \alpha^T)x=0,\quad \forall x\in \mathbb{R}^{n-1}
\end{align*}     
根据条件: (\textcolor{blue}{注意, 这里可以取$\sqrt{a_{11}}$是因为$A$是正定矩阵, 所有对角元都是正的})
\begin{align*}
    &\begin{pmatrix}
        y&x^T
    \end{pmatrix} \begin{pmatrix}
        a_{11} & \alpha^T \\ \alpha & A_{22} 
    \end{pmatrix} \begin{pmatrix}
        y\\x
    \end{pmatrix}=a_{11}y^2+ 2yx^T\alpha +x^TA_{22}x\\
    =& a_{11}y^2+ 2yx^T\alpha + \frac{1}{a_{11}}(x^T\alpha )^2 +x^T(A_{22} - \frac{1}{a_{11}}\alpha \alpha^T)x\\
    =& (\sqrt{a_{11}}y + \frac{x^T\alpha}{\sqrt{a_{11}}})^2 + x^T(A_{22} - \frac{1}{a_{11}}\alpha \alpha^T)x \geq 0
\end{align*}
取$y = - \frac{x^T\alpha}{a_{11}}$, 得到:
\begin{align*}
    &(\sqrt{a_{11}}y + \frac{x^T\alpha}{\sqrt{a_{11}}})^2 + x^T(A_{22} - \frac{1}{a_{11}}\alpha \alpha^T)x\\
    =&x^T(A_{22} - \frac{1}{a_{11}}\alpha \alpha^T)x\geq 0
\end{align*}
且取$0$当且仅当$x=0$, 得证.
\end{solution}
\begin{solution}
\textcolor{red}{\textbf{订正的解法: 因为正定矩阵不一定是对称矩阵}}\\
考虑:
\begin{align*}
    &\begin{pmatrix}
        x&y^T
    \end{pmatrix} \begin{pmatrix}
        a_{11} & \alpha^T \\ \beta & A_{22} 
    \end{pmatrix} \begin{pmatrix}
        x\\y
    \end{pmatrix}=a_{11}y^2+ x\alpha^Ty+ xy^T\beta +x^TA_{22}x\\
    =& (\sqrt{a_{11}}x + \frac{1}{\sqrt{a_{11}}} y^T\beta )(\sqrt{a_{11}}x + \frac{1}{\sqrt{a_{11}}} \alpha^Ty ) + y^T(A_{22}- \frac{1}{a_{11}} \beta \alpha^T)y
\end{align*} 
取$x = - \frac{y^T\alpha}{a_{11}}$, 有:
\begin{align*}
    &\begin{pmatrix}
        x&y^T
    \end{pmatrix} \begin{pmatrix}
        a_{11} & \alpha^T \\ \beta & A_{22} 
    \end{pmatrix} \begin{pmatrix}
        x\\y
    \end{pmatrix} = y^T\underbrace{(A_{22}- \frac{1}{a_{11}} \beta \alpha^T)}_{A_2}y
\end{align*}
因此取任意的向量$y\in \mathbb{R}^{n-1}$, 有$y^TA_{2}y\geq 0$\\
且$y^TA_{2}y=0\iff y=0$\\
因此$A$正定可以推出$A_2$正定.
\end{solution}

\begin{problem}
14. 假定已知$A\in \mathbb{R}^{n\times n}$的三角分解$A=LU$, 设计算法来计算$A^{-1}$.
\end{problem}
\begin{solution}
形式上: $A^{-1} = U^{-1} L^{-1}$\\
\textcolor{red}{\textbf{方法一}}: 根据第1题的算法, 我们有了下三角矩阵求逆的算法, 那么可以得到$L^{-1}$, 以及$(U^{T})^{-1}=(U^{-1})^T$\\自然就得到了$A^{-1} = ((U^{T})^{-1})^T L^{-1}$.\\\textcolor{blue}{这样的计算复杂度是$O(n^3)+O(n^3)+O(n^3)=O(n^3)$, 注意加法的最后一个是矩阵乘法}\\
\textcolor{red}{\textbf{方法二}}: 因为$AA^{-1}=I$, 那么$A^{-1}$的第$j$列是$Ax_j=e_j$的解, 即$LUx_j=e_j$的解, 那么先解$Ly=e_j$, 再解$Ux_j=y$就可以解出$x_j$.\\
\textcolor{blue}{这样的计算复杂度是$n\cdot O(n^2) = O(n^3)$, 因为前代法和回代法的复杂度都是$O(n^2)$}\\
\textcolor{blue}{绷:题目是求$A^{-1}(i:j)$}, 那么直接用方法二即可:\\
$Ax_j=e_j$的解向量的地$i$个元素, $O(n^2)$
\end{solution}

\begin{problem}
19. 若$A=LL^T$是$A$的Cholesky分解, 试证: $L$的$i$阶顺序主子阵$L_i$正好是$A$的$i$阶顺序主子阵$A_i$的Cholesky因子.
\end{problem}

\begin{solution}
根据Cholesky分解, 有:
\begin{align*}
    A = LL^T = \begin{pmatrix}
        L_i & 0\\ L_{21} & L_{22}
    \end{pmatrix} \begin{pmatrix}
        L_i^T & L_{21}^T\\ 0 & L_{22}^T
    \end{pmatrix}=\begin{pmatrix}
        L_iL_i^T & L_i L_{21}^T \\ L_{21}L_i^T & L_{22}L_{22}^T
    \end{pmatrix}
\end{align*}
因此自然有: $A_i = L_i L_i^T$
\end{solution}

\section{线性方程组的敏度分析}

\begin{problem}
2. 证明: 当且仅当$x$和$y$线性相关且$x^Ty\geq 0$时, 才有:
\begin{align*}
    || x+y||_2 = ||x||_2+||y||_2
\end{align*}
\end{problem}

\begin{solution}
$(\Rightarrow)$: 已知$y=kx, x^Ty\geq 0$, 那么$y=kx, k\geq 0$, 那么等式成立:
\begin{align*}
    ||x+y||_2=(1+k)||x||_2=||x||_2+||y||_2
\end{align*}
$(\Leftarrow)$: 已知等式成立, \textcolor{red}{考虑$y$关于$x$的正交分解$y=kx+(y-kx), k=\frac{x^Ty}{x^Tx} $}, 又因为等式成立, 两边平方后代入:
\begin{align*}
    \norm{x+y}_2^2&=(1+k)^2\norm{x}_2^2+\norm{y-kx}_2^2\\
    \norm{x}_2^2+\norm{y}_2^2+2\norm{x}_2\norm{y}_2&=\norm{x}_2^2+k^2\norm{x}_2^2+\norm{y-kx}_2^2+2\norm{x}_2\norm{y}_2\\
    || x+y||_2 = ||x||_2+||y||_2&\Rightarrow k\norm{x}_2=\norm{kx+y-kx}_2\\
    &\Rightarrow \norm{y-kx}_2^2=0\iff y=kx
\end{align*}
因此$x$和$y$线性相关, 又因为:
\begin{align*}
    \norm{x+y}_2=|1+k|\norm{x}_2=(1+|k|)\norm{x}_2
\end{align*}
故$k\geq 0$, 即$x^Ty\geq 0$.
\end{solution}

\begin{problem}
    3. 证明: 如果$A=[a_1,\cdots,a_n]$是按列分块的, 那么 
    \begin{align*}
        \norm{A}_F^2=\norm{a_1}_2^2+\cdots+\norm{a_n}_2^2
    \end{align*}
\end{problem}

\begin{solution}
\begin{align*}
    \norm{A}_2=\sum_{j=1}^n(\sum_{i=1}^n a_{ij}^2 )=\sum_{j=1}^n \norm{a_j}_2^2
\end{align*}
得证
\end{solution}

\begin{problem}
4. 证明: 
\begin{align*}
    \norm{AB}_F&\leq \norm{A}_2\norm{B}_F\\
    \norm{AB}_F&\leq \norm{A}_F\norm{B}_2
\end{align*}
\end{problem}

\begin{solution}
\textcolor{red}{\textbf{方法一}}: 考虑$\norm{AB}^2_F=\text{tr}(B^*A^*AB^*)$\\
对于Hermite矩阵$A^*A$, 首先, 它的对角元都是非负实数(因为第$i$个对角元是$A$的第$i$列的模平方)\\
其次, 根据谱定理, 它可以酉对角化, 即存在酉矩阵$U$, 使得$A^*A=U^*DU$, $D$是对角矩阵, 且对角元都是非负实数.\\
因此有:
\begin{align*}
    \norm{AB}_F^2&=\text{tr}(B^*A^*AB)=\text{tr}(B^*U^*DUB)=\text{tr}(DB^*U^*UB)\\&\leq \max_{i}(d_{ii})\text{tr}(B^*U^*UB)=\max_{i}(d_{ii})\text{tr}(U^*B^*BU)\\&=\max_{i}(d_{ii})\text{tr}(B^*B)=\norm{A}_2^2\norm{B}_F^2
\end{align*}
同理有:
\begin{align*}
    \norm{AB}_F^2&=\text{tr}(B^*A^*AB)=\text{tr}(A^*B^*BA)=\text{tr}(A^*\tilde{U}^*\tilde{D}\tilde{U}A)=\text{tr}(\tilde{D}A^*\tilde{U}^*\tilde{U}A)\\&\leq \max_{i}(\tilde{d}_{ii})\text{tr}(A^*\tilde{U}^*\tilde{U}A)=\max_{i}(\tilde{d}_{ii})\text{tr}(\tilde{U}^*A^*A\tilde{U})\\&=\max_{i}(\tilde{d}_{ii})\text{tr}(A^*A)=\norm{B}_2^2\norm{A}_F^2=\norm{A}_F^2\norm{B}_2^2
\end{align*}
上面的方法用到的性质有:
\begin{itemize}
    \item 矩阵成绩的trace中, 乘积可以互换
    \item 矩阵的trace在相似变化下保持不变, 因为$\text{tr}(U^*CU)=\text{tr}(U^*UC)=\text{tr}(C)$.
\end{itemize}

\textcolor{red}{\textbf{方法二}}: 

考虑$\norm{AB}_F^2=\sum_{i=1}^n \norm{AB[i,:]}_2^2$, \textcolor{blue}{而$AB$的第$i$行就是$A$的第$i$行乘$B$: $AB[i,:]=\alpha_i^TB$}, 因此
\begin{align*}
    \norm{AB}_F^2=\sum_{i=1}^n \norm{AB[i,:]}_2^2=\sum_{i=1}^n \norm{\alpha_i^TB}_2^2\leq \sum_{i=1}^n \norm{\alpha_i^T}_2^2\norm{B}_2^2=\norm{A}_F^2\norm{B}_2^2
\end{align*}
换一个角度, \textcolor{blue}{$AB$的第$j$行就是$A$乘上$B$的第$j$列: $AB[:,j]=A\beta_j$}, 因此:
\begin{align*}
    \norm{AB}_F^2=\sum_{j=1}^n \norm{AB[:,j]}_2^2=\sum_{j=1}^n \norm{A\beta_j}_2^2\leq \sum_{j=1}^n \norm{A}_2^2\norm{\beta_j}_2^2=\norm{A}_2^2\norm{B}_F^2
\end{align*}
这里用到了:
\begin{itemize}
    \item 矩阵范数和向量范数的相容性: $\norm{Ax}_2\leq \norm{A}_2\norm{x}_2$
\end{itemize}
\end{solution}

\begin{note}
(1) $F$范数(平方)的两种表示: $\norm{A}_F^2=\sum_{i,j}|a_{ij}|^2=\textcolor{blue}{\text{tr}(A^TA) }$, 注意在复矩阵的情况下为$\textcolor{blue}{\text{tr}(A^*A)}$, 共轭转置.\\
(2) 共轭转置不变的复矩阵, 即Hermite矩阵, 是可以酉对角化的(即, 实矩阵意义下的正交对角化)\\
(3) $A^TA$的特征值都是非负实数
\end{note}

\begin{problem}
    8. 若$\norm{A}<1$, 且$\norm{I}=1$, 证明:
    \begin{align*}
        \norm{(I-A)^{-1}}\leq \frac{1}{1-\norm{A}}
    \end{align*}
\end{problem}

\begin{solution}
因为$\norm{A}<1$, 且任意的矩阵范数都大于等于谱半径, 因此$\rho(A)\leq \norm{A}<1$\\
因为$\rho(A)<1 \iff \sum_{k=0}^{\infty}A^k$收敛, 因此$ \sum_{k=0}^{\infty}A^k$存在\\
因为$(I-A)(I+A+\cdots+A^k)=I-A^{k+1}$, 且$\norm{A^{k+1}}\leq \norm{A}^{k+1}\to 0 \Rightarrow A^{k+1}\to \bf{0}$\\
因此可以证明: $(I-A)\sum_{k=0}^{\infty}A^k=I$, 即$I-A$可逆, $(I-A)^{-1}=\sum_{k=0}^{\infty}A^k$\\
注意题目条件\textcolor{red}{$\norm{I}=1$}, 我们有
\begin{align*}
    \norm{(I-A)^{-1}}=\norm{\sum_{k=0}^{\infty}A^k}\leq \sum_{k=0}^{\infty}\norm{A^k}\leq \sum_{k=0}^{\infty}\norm{A}^k=\frac{\textcolor{red}{\norm{I}}\cdot(1-\lim_{k\to \infty}\norm{A}^k)}{1-\norm{A}}=\frac{1}{1-\norm{A}}
\end{align*}
\end{solution}
\begin{note}
需要记住的结论:
\begin{itemize}
    \item $A$的任意矩阵范数大于等于谱半径: $\rho(A)\leq \norm{A}$, \textcolor{blue}{即, 这一条对任意矩阵成立}
    \item $A$给定, 谱半径是矩阵(算子)范数的下确界, $\forall \epsilon>0$, 存在算子范数, 使得$\norm{A}\leq \rho(A)+\epsilon$
    \item $\lim_{k\to \infty}A^k=0 \iff \sum_{k=0}^{\infty}A^k \;\text{收敛}\iff \rho(A)<1$, 若收敛, $(I-A)^{-1}=\sum_{k=0}^{\infty}A^k$
\end{itemize}
\end{note}

\begin{problem}
11. 设 
\begin{align*}
    A=\begin{bmatrix}
        375&374\\752&750
    \end{bmatrix}
\end{align*}
(1) 计算$A^{-1}$和$\kappa_{\infty}(A)$\\
(2) 选择$b,\delta b, x, \delta x$, 使得
\begin{align*}
    Ax=b, \quad A(x+\delta x)=b+\delta b
\end{align*}
而且$\frac{\norm{\delta b}_{\infty}}{\norm{ b}_{\infty}}$很小, 但$\frac{\norm{\delta x}_{\infty}}{\norm{x}_{\infty}}$很大\\
(2) 选择$b,\delta b, x, \delta x$, 使得
\begin{align*}
    Ax=b, \quad A(x+\delta x)=b+\delta b
\end{align*}
而且$\frac{\norm{\delta x}_{\infty}}{\norm{ x}_{\infty}}$很小, 但$\frac{\norm{\delta b}_{\infty}}{\norm{b}_{\infty}}$很大
\end{problem}

\begin{solution}
观察矩阵, 设$a=374$, 那么:
\begin{align*}
    A = \begin{bmatrix}
        a&a-1\\
        2(a+1)&2a
    \end{bmatrix}
\end{align*}
记住了\textcolor{blue}{二阶矩阵求逆的标准公式:
\begin{align*}
    \begin{pmatrix}
        a&b\\c&d
    \end{pmatrix}^{-1}=\frac{1}{|ad-bc|}\begin{pmatrix}
        d&-b \\-c &a
    \end{pmatrix}
\end{align*}}
因此有:
\begin{align*}
    A^{-1}=\frac{1}{2}\begin{pmatrix}
        2a&1-a\\-2a-2&a
    \end{pmatrix}=\begin{pmatrix}
        375&-187\\-376&187.5
    \end{pmatrix}
\end{align*}
计算$\kappa_{\infty}(A)=\norm{A}_{\infty}\norm{A^{-1}}_{\infty}$, 有(行范数计算)
\begin{align*}
    \kappa_{\infty}(A)=\norm{A}_{\infty}\norm{A^{-1}}_{\infty}=(752+750)\cdot(376+187.5)= 846377
\end{align*}

(2) 考虑先进行敏度分析, $A(x+\delta x)=b+\delta b$, 因为$Ax=b$, 所以$\delta x=A^{-1}\delta b$, 因此有$\norm{\delta x}_{\infty}\leq \norm{A^{-1}}_{\infty}\norm{\delta b}_{\infty}$, 两边同时除以$\norm{b}_{\infty}\leq \norm{A}_{\infty}\norm{x}_{\infty}$, 有:
\begin{align*}
    &\frac{\norm{\delta x}_{\infty}}{\norm{A}_{\infty}\norm{x}_{\infty}} \leq \frac{\norm{\delta x}_{\infty}}{\norm{b}_{\infty}}\leq \norm{A^{-1}}_{\infty}\frac{\norm{\delta b}_{\infty}}{\norm{b}_{\infty}}\\
    \Rightarrow& \frac{\norm{\delta x}_{\infty}}{\norm{x}_{\infty}} \leq \norm{A}_{\infty}\norm{A^{-1}}_{\infty}\frac{\norm{\delta b}_{\infty}}{\norm{b}_{\infty}}=\kappa(A)\frac{\norm{\delta b}_{\infty}}{\norm{b}_{\infty}}
\end{align*}
虽然我们已知$\kappa(A)$很大, 但这似乎对我们的构造没有什么帮助. 因为这个不等式是一个关于上下界的估计, 但实际值可以不在等号附近.

\textcolor{red}{好的思路: 因为矩阵$A$的条件数很大(实际上是$\norm{A}_{\infty}$和$\norm{A^{-1}}_{\infty}$都不小), 所以对于$A\delta x=\delta b$(已知$\delta x$, 算$\delta b$)和$\delta x=A^{-1}\delta b$(已知$\delta b$, 算$\delta x$), 即使已知的对象比较小, 被计算的部分也会被放得很大}.

取$x=\begin{pmatrix}
    1\\-1
\end{pmatrix}$, 那么$b=Ax=\begin{pmatrix}
    1\\2
\end{pmatrix}$, \textcolor{blue}{先取$\delta b = \begin{pmatrix}
\epsilon\\0
\end{pmatrix} $ }, 计算得到$\delta x =A^{-1}\delta b=\begin{pmatrix}
    375\epsilon\\-376\epsilon
\end{pmatrix} $, 那么($\epsilon>0$) 
\begin{align*}
    &\frac{\norm{\delta b}_{\infty}}{\norm{ b}_{\infty}}=\frac{\epsilon}{2}, \frac{\norm{\delta x}_{\infty}}{\norm{ x}_{\infty}}=\frac{376\epsilon}{1}=376\epsilon\\
    \epsilon=1\Rightarrow& \frac{\norm{\delta b}_{\infty}}{\norm{ b}_{\infty}}=\frac{1}{2}, \frac{\norm{\delta x}_{\infty}}{\norm{ x}_{\infty}}=376
\end{align*}

(3) 类似于(2), 取$x=\begin{pmatrix}
    1\\-1
\end{pmatrix}$, 那么$b=Ax=\begin{pmatrix}
    1\\2
\end{pmatrix}$, \textcolor{blue}{先取$\delta x = \begin{pmatrix}
    \epsilon\\0
    \end{pmatrix} $ }, 计算$\delta b = A\delta x=\begin{pmatrix}
        375\epsilon\\ 752\epsilon
    \end{pmatrix} $, 那么($\epsilon>0$) 
\begin{align*}
    &\frac{\norm{\delta b}_{\infty}}{\norm{ b}_{\infty}}=\frac{752\epsilon}{2}, \frac{\norm{\delta x}_{\infty}}{\norm{ x}_{\infty}}=\frac{\epsilon}{1}=\epsilon\\
    \epsilon=1\Rightarrow& \frac{\norm{\delta b}_{\infty}}{\norm{ b}_{\infty}}=376, \frac{\norm{\delta x}_{\infty}}{\norm{ x}_{\infty}}=1
\end{align*}
\end{solution}

\section{最小二乘问题的解法}
\begin{problem}
1. 设$A=\begin{pmatrix}
        1&2\\3&4\\5&6
    \end{pmatrix}$, $b=\begin{pmatrix}
        1\\1\\1
    \end{pmatrix}$, 用正则化方法求$LS$问题的解
\end{problem}
\begin{solution}
\textcolor{blue}{考虑正则化方法, 是考虑正则化方程$A^TAx=A^Tb$}\\
求解得到:
\begin{align*}
    \begin{pmatrix}
        35&44\\44&56
    \end{pmatrix}\begin{pmatrix}
        x_1\\x_2
    \end{pmatrix}=\begin{pmatrix}
        9\\12
    \end{pmatrix}\Rightarrow
    \begin{pmatrix}
        x_1\\x_2
    \end{pmatrix}=\begin{pmatrix}
        -1\\1
    \end{pmatrix}
\end{align*}
\end{solution}

\begin{problem}
2. 设$x=(1,0,4,6,3,4)^T$, 求一个Householder变换和一个正数$\alpha$, 使得$Hx=(1,\alpha,4,6,0,0)^T$.
\end{problem}
\begin{solution}
因为Householder变换是正交变换, 因此向量长度不变, 根据$||x||_2^2=\norm{Hx}_2^2$得到正数$\alpha=5$.

两向量相减, 得到镜面对称的法方向$v=(0,-5,0,0,3,4)^T$, 那么Householder变换应该是
\begin{align*}
    H(v)=I-2\frac{vv^T}{v^Tv}
\end{align*}
\textcolor{red}{请问助教老师, 考试的时候需要把真正的矩阵算出来吗?}
\end{solution}
\begin{note}
注意书上的结论: Householder变换可以把一个向量的"任何若干相邻"的元素化为零, 想要把$x\in \mathbb{R}^n$, 长度为$n$的向量的$k+1$到$j$变成$0$, 那么需要把向量
$$v=(0,\cdots,0,x_{k}-\alpha,\underbrace{x_{k+1},\cdots,x_{j}}_{\text{这是想要变成0的部分}},0,\cdots,0)^T, \quad \textcolor{red}{\alpha=\sqrt{x_k^2+\cdots+x_{j}^2}}$$
\end{note}

\begin{problem}
5. $x=(x_1,x_2)^T$是一个二维的复向量, 给出一种算法, 计算酉矩阵
$$Q=\begin{pmatrix}
    c&\bar{s}\\-s&c
\end{pmatrix}, c\in R,c^2+|s|^2=1$$
使得$Q$的第二个分量是$0$.
\end{problem}

\begin{solution}
若$x_1=0$, 那么$c=0, s=e^{i\theta}$\\
若$x_2=0$, 那么$s=0, c=1\text{或}-1$\\
若$x_1\neq 0,x_2\neq 0$, 那么考虑$\alpha=\arctan \frac{|x_1|}{|x_2|}$, 取$c=\sin \alpha, s=\cos \alpha \cdot \frac{x_1|x_2|}{|x_1|x_2}$, 满足$Qx$的第二个分量是$0$. 即(实际上核心是$cx_2=sx_1$):
\begin{align*}
    c = \frac{|x_1|}{\sqrt{|x_1|^2+|x_2|^2}}, s=\frac{|x_2|}{\sqrt{|x_1|^2+|x_2|^2}}\cdot\frac{x_1|x_2|}{|x_1|x_2}
\end{align*}
\textcolor{red}{但是, 结合书上对Givens变换的分析, 为了给出一个robust的算法, 防止在计算过程中出现溢出, 可以这么做}
\begin{figure}[h]
    \centering
    \includegraphics[width=0.44\textwidth]{image/1.jpg}
\end{figure}
\end{solution}
\newpage
\begin{problem}
7. 设$x$和$y$是$\mathbb{R}^n$中的两个非零向量, 给出一种算法来确定一个Householder变换$H$, 使得$Hx=\alpha y, \alpha\in R$.
\end{problem}
\begin{solution}
因为把$x$镜面反射到$y$, 因此法向量$v\in \text{span}(x,y)$, 设$v=x+\beta y$, 因为 
\begin{align*}
H(v)x=x-2\frac{v^Tx}{v^Tv}v=(1-2\frac{v^Tx}{v^Tv})x-2\frac{v^Tx}{v^Tv}\beta y
\end{align*}
因此$x$的系数$1-2\frac{v^Tx}{v^Tv}=0 \iff \beta=\pm \frac{\norm{x}_2}{\norm{y}_2}$, 此时有:
\begin{align*}
    H(v)x=\pm \frac{\norm{x}_2}{\norm{y}_2}y, v=x\pm \frac{\norm{x}_2}{\norm{y}_2}y
\end{align*}
\end{solution}

\begin{problem}
    设$A\in \mathbb{R}^{m\times n}$(\textcolor{red}{书上typo, 此处就是$\mathbb{R}^{n\times n}$})是一个对角加边矩阵, 即:
    \begin{align*}
        A = \begin{bmatrix}
            \alpha_1 & \rho_2 & \rho_3 & \dots & \dots & \rho_n & 0 \\
            \beta_2 & \alpha_2 & 0 & \dots & \dots & \dots & 0 \\
            \beta_3 & 0 & \alpha_3 & \dots & \dots & \dots & 0 \\
            \vdots & \vdots & \vdots & \ddots & & \vdots & \vdots \\
            \vdots & \vdots & \vdots & & \ddots & \vdots & \vdots \\
            \beta_n & 0 & \dots & \dots & \dots & \alpha_{n-1} & 0 \\
            0 & 0 & \dots & \dots & \dots & 0 & \alpha_n
            \end{bmatrix}
    \end{align*}
    试给出用Givens变换求$A$的$QR$分解的详细算法
\end{problem}
\begin{solution}
使用Givens变换对$A$进行逐列操作, 将$A$变为一个上三角矩阵. 但是这样还不如用Householder变换.

想要利用稀疏矩阵的特性, 可以想到Givens变换$G(i,j,\theta)$左乘作用于矩阵$A$之后, 实际上仅仅改变了矩阵$A$的$i,j$行列交错的四个元素.

因此, 为了消去$\beta_2$, 可以使用$G(1,2,\theta_2)$; 为了消去$\beta_3$, 可以使用$G(1,3,\theta_n)$; 直到为了消去$\beta_n$, 可以使用$G(1,n,\theta_n)$. 因此使用$n$个Givens变换即可达到效果, 具体的算法如下:
\begin{figure}[h]
    \centering
    \includegraphics[width=0.7\textwidth]{image/2.png}
\end{figure}
\end{solution}

\section{线性方程组的古典迭代解法}

\begin{problem}
    1. 设方程组 $A x=b$ 的系数矩阵为

    $$
    A_1=\left[\begin{array}{rrr}
    2 & -1 & 1 \\
    1 & 1 & 1 \\
    1 & 1 & -2
    \end{array}\right], \quad A_2=\left[\begin{array}{rrr}
    1 & 2 & -2 \\
    1 & 1 & 1 \\
    2 & 2 & 1
    \end{array}\right]
    $$
    
    
    证明:对 $A_1$ 来说, Jacobi 迭代法不收敛,而 G-S 迭代法收敛;对 $A_2$ 来说, Jacobi 迭代法收敛, 而 G-S 迭代法不收敛。
\end{problem}
\begin{solution}
Jacobi迭代:
$$x_{k+1}=D^{-1}(L+U)x_k+D^{-1}b$$
G-S迭代:
$$x_{k+1}=(D-L)^{-1}Ux_k+D^{-1}b$$
对于$A_1$, \textcolor{red}{第一个容易出错的地方$A=D-L-U$, $L$和$U$是原来的$A$的不含对角的下三角部分和上三角部分的\textbf{负数值}}
\begin{align*}
    D^{-1}(L+U)&=\begin{pmatrix}
        0&1/2&-1/2\\
        -1&0&-1\\
        1/2&1/2&0\\
    \end{pmatrix}\Rightarrow |\lambda I-D^{-1}(L+U)|=\lambda(\lambda^2+5/4)\Rightarrow \rho(D^{-1}(L+U))=\sqrt{5/4}>1\\
   (D-L)^{-1}U&=\begin{pmatrix}
        0&1/2&-1/2\\
        0&-1/2&-1/2\\
        0&0&-1/2\\
    \end{pmatrix}\Rightarrow |\lambda I-(D-L)^{-1}U|=\lambda(\lambda+1/2)^2\Rightarrow \rho(D^{-1}(L+U))=\sqrt{1/2}<1
\end{align*}
所以对 $A_1$ 来说, Jacobi 迭代法不收敛,而 G-S 迭代法收敛\\
而对于$A_2$:
\begin{align*}
    D^{-1}(L+U)&=\begin{pmatrix}
        0&-2&2\\
        -1&0&-1\\
        -2&-2&0\\
    \end{pmatrix}\Rightarrow |\lambda I-D^{-1}(L+U)|=\lambda^3 \Rightarrow \rho(D^{-1}(L+U))=0<1\\
   (D-L)^{-1}U&=\begin{pmatrix}
        0&-2&2\\
        0&2&-3\\
        0&0&2\\
    \end{pmatrix}\Rightarrow |\lambda I-(D-L)^{-1}U|=\lambda(\lambda-2)^2\Rightarrow \rho(D^{-1}(L+U))=2>1
\end{align*}
所以对 $A_2$ 来说, Jacobi 迭代法收敛, 而 G-S 迭代法不收敛
\end{solution}

\begin{problem}
    2. 设 $B \in \mathbf{R}^{n \times n}$ 满足 $\rho(B)=0$. 证明: 对任意的 $g, x_0 \in \mathbf{R}^n$, 迭代格式

    $$
    x_{k+1}=B x_k+g, \quad k=0,1, \cdots
    $$
    
    
    最多迭代 $n$ 次就可得到方程组 $x=B x+g$ 的精确解.
\end{problem}
\begin{solution}
因为$\rho(B)=0$, 所以$B$的特征多项式是$f_B(\lambda)=\lambda^n$, 根据\textcolor{red}{Cayley-Hamilton定理, 矩阵$B$的特征多项式零化$B$, 或者说, 矩阵的极小多项式整除特征多项式, 因此存在$m\leq n, B^m=0$}.\\
所以有:
\begin{align*}
    x_{m}-x^*=B^m(x_0-x^*)=0, m\leq n
\end{align*}
因此最多$n$次就可以得到精确解.
\end{solution}

\begin{problem}
    3. 考虑线性方程组

    $$
    A x=b,
    $$
    
    
    这里
    
    $$
    A=\left[\begin{array}{lll}
    1 & 0 & a \\
    0 & 1 & 0 \\
    a & 0 & 1
    \end{array}\right]
    $$
    
    (1) $a$ 为何值时, $A$ 是正定的?
    (2) $a$ 为何值时, Jacobi 迭代法收敛?
    (3) $a$ 为何值时, G-S 迭代法收敛?
\end{problem}
\begin{solution}
(1) \textcolor{blue}{判断一个矩阵$A$是不是正定矩阵的方法: \\
1. 矩阵$A$的特征值全部都是正数\\
2. 矩阵$A$的各阶顺序主子式都是正数}\\
考虑矩阵$A$的各阶顺序主子式, 分别是$1,1,1-a^2$, 因此$a^2<1$的时候, $A$是正定矩阵.

(2) Jacobi迭代法, 分析矩阵$D^{-1}(L+U)$的谱半径:
\begin{align*}
    D^{-1}(L+U)=\begin{pmatrix}
        0&0&-a\\
        0&0&0\\
        -a&0&0\\
    \end{pmatrix}\Rightarrow |\lambda I-D^{-1}(L+U)|=\lambda(\lambda+a)(\lambda-a)
\end{align*}
因此$|a|<1$时Jacobi迭代收敛

(3) G-S迭代法, 分析矩阵$(D-L)^{-1}U$的谱半径:
\begin{align*}
    (D-L)^{-1}U=\begin{pmatrix}
        0&0&-a\\
        0&0&0\\
        0&0&a^2\\
    \end{pmatrix}\Rightarrow |\lambda I-(D-L)^{-1}U|=\lambda^2(\lambda-a^2)
\end{align*}
因此$|a|<1$时G-S迭代收敛
\end{solution}

\begin{problem}
    5. 若 $A$ 是严格对角占优的或不可约对角占优的, 则 G-S 迭代法收敛.
\end{problem}
\begin{solution}
\textcolor{red}{模仿书上的证明方法, 也就是, 假设有一个绝对值大于等于$1$的特征值, 考虑迭代矩阵的特征多项式, 可以证明这个特征多项式(行列式)不等于$0$, 从而与特征值的定义矛盾}.

考虑G-S迭代的迭代矩阵$(D-L)^{-1}U$存在特征值, 使得$|\lambda|\geq 1$, 那么考虑特征多项式$|\lambda I -(D-L)^{-1}U |$, 又因为:
\begin{align*}
    \lambda I-(D-L)^{-1}U=(D-L)^{-1}(\lambda D-\lambda L -U)
\end{align*}
且 
$$\lambda D-\lambda L -U=\lambda(D-L-U)+\underbrace{(\lambda-1)}_{|\lambda|\geq 1}U=\lambda A+(\lambda-1)U$$
因此$\lambda D-\lambda L -U$也是严格对角占优的或者不可约对角占优的, 因此$\lambda D-\lambda L -U$非奇异, 行列式不等于$0$. 从而得到 
\begin{align*}
    |\lambda I-(D-L)^{-1}U|
    =|(D-L)^{-1}|\cdot|\lambda D-\lambda L -U|\neq 0
\end{align*}
这与$\lambda$是$I-(D-L)^{-1}U$的特征值矛盾, 因此$I-(D-L)^{-1}U$的所有特征值的模长小于$1$, 因此G-S迭代收敛.
\end{solution}

\section{非对称特征值问题的计算方法}

\begin{problem}
7. 分别应用幂法于矩阵
$$A=\begin{bmatrix}
    \lambda&1\\
    0&\lambda
\end{bmatrix}, B=\begin{bmatrix}
    \lambda&1\\
    0&-\lambda
\end{bmatrix} ,\lambda \neq 0$$
并考察所得序列的特性
\end{problem}
\begin{solution}
(1) $A$的两个特征值都是$\lambda$, 且特征子空间只有一维, $v=(1,0)$\\
取$u_0=(1,1)$, 有(不妨假设$\lambda>0$) 
\begin{align*}
    y_1=Au_0=(\lambda+1,\lambda),& \mu_1=\lambda+1, u_1=(\lambda+1,\lambda)/(\lambda+1)\\
    y_2=Au_1=(\lambda^2+2\lambda,\lambda^2),& \mu_2=(\lambda^2+2\lambda)/(\lambda+1), u_2=(\lambda^2+2\lambda)/(\lambda^2+2\lambda)\\
\end{align*}
因此, 实际有: 
$$y_k=A^ku_0/\norm{A^k}_{\infty}$$
而 
$A=\lambda I +H, H^k=0, \forall k\geq 2$
因此 
$$A^k=\lambda^k I +k\lambda^{k-1}H=\begin{bmatrix}
    \lambda^k&k\lambda^{k-1}\\
    0&\lambda^{k}\\
\end{bmatrix}\Rightarrow y_k=A^ku_0/\norm{A^k}_{\infty}=\frac{1}{\lambda^k+k\lambda^{k-1}}\begin{pmatrix}
    \lambda^k+k\lambda^{k-1}\\
    \lambda^k
\end{pmatrix}$$
而 
$$\frac{1}{\lambda^k+k\lambda^{k-1}}\begin{pmatrix}
    \lambda^k+k\lambda^{k-1}\\
    \lambda^k
\end{pmatrix}\to \begin{pmatrix}
    1\\0
\end{pmatrix}$$
\textcolor{red}{而因为有两个相同的特征值, 收敛速度由
$$\frac{\lambda^k}{\lambda^k+k\lambda^{k-1}}=\frac{1}{1+\frac{k}{\lambda}}\sim O(\frac{\lambda}{k})\to 0$$
决定, 收敛速度较慢}

(2) $B$的特征值是$\lambda, -\lambda$\\
取$u_0=(1,1)$, 那么:
\begin{align*}
    y_1=Bu_0=(\lambda+1,-\lambda), &\mu_1=\lambda+1, u_1=(\lambda+1,-\lambda)/(\lambda+1)\\
    y_2=Bu_1=(\lambda^2,\lambda^2)/(\lambda+1), &\mu_2=(\lambda^2)/(\lambda+1), u_2=(1,1)
\end{align*}
因此是一个震荡的, 不收敛的序列\\
如果取$u_0=(1,0)$, 那么就是一步收敛(因为是特征向量).
\end{solution}

\begin{problem}
    9. $A\in C^{n\times n}$有实特征值, 满足$\lambda_1>\lambda_2\geq \cdots \geq \lambda_n$. 对$A-\mu I$用幂法, 证明: 选择$\mu=\frac{1}{2}(\lambda_2+\lambda_n)$, 产生的迭代序列收敛到属于$\lambda_1$的特征向量最快.
\end{problem}
\begin{solution}
\textcolor{red}{这等价于求
$$\min_{\mu} \max_{2\leq i\leq n} |\frac{\lambda_i-\mu}{\lambda_1-\mu}|$$}
要证明:
$$\min_{\mu} \max_{2\leq i\leq n} |\frac{\lambda_i-\mu}{\lambda_1-\mu}|=\frac{1}{2}(\lambda_2+\lambda_n)$$
因为:
$$\frac{\lambda_2-\mu}{\lambda_1-\mu}=\frac{\lambda_n-\mu}{\lambda_1-\mu}\Rightarrow \mu^*=\frac{1}{2}(\lambda_2+\lambda_n)$$
\end{solution}

\begin{problem}
    11. 用反幂法计算 
    $$\begin{bmatrix}
        2&1&0\\
        1&3&1\\
        0&1&4\\
    \end{bmatrix}$$
    对应于近似特征值$\tilde{\lambda}=1.2679$的近似特征向量
\end{problem}
\begin{solution}
原理:
\begin{align*}
    &A(X_1,X_2,X_3)=(X_1,X_2,X_3)\begin{pmatrix}
        \lambda_1\\
        \lambda_2\\
        \lambda_3\\
    \end{pmatrix}\iff X^{-1}AX=\begin{pmatrix}
        \lambda_1\\
        \lambda_2\\
        \lambda_3\\
    \end{pmatrix}\\
    \Rightarrow& (X^{-1}AX)^{-1}=X^{-1}A^{-1}X=D^{-1}\iff A^{-1}X=XD^{-1}
\end{align*}
因此对应于近似特征值$\tilde{\lambda}=1.2679$的近似特征向量是$A^{-1}$的对应于$1/\tilde{\lambda}=0.7887$的近似特征向量

根据迭代格式, 首先要计算$A$的逆矩阵:
$$A^{-1}=\frac{1}{18}\begin{bmatrix}
    11&-4&1\\
    -4&8&-2\\
    1&-2&5
\end{bmatrix}$$
取初始的向量$u_0=(1,1,1)$, 那么:
\begin{align*}
    y_1=A^{-1}u_0=\frac{1}{9}\begin{pmatrix}
        4\\1\\2
    \end{pmatrix}, & \mu_1=\norm{y_1}_{\infty}=\frac{4}{9}, u_1=\frac{1}{4}\begin{pmatrix}
        4\\1\\2
    \end{pmatrix}\\
    y_2=A^{-1}u_1=\frac{1}{12}\begin{pmatrix}
        7\\-2\\2
    \end{pmatrix}, & \mu_2=\norm{y_2}_{\infty}=\frac{7}{12}, u_2=\frac{1}{7}\begin{pmatrix}
        7\\-2\\2
    \end{pmatrix}\\
    y_3=A^{-1}u_2=\frac{1}{42}\begin{pmatrix}
        29\\-16\\7
    \end{pmatrix}, & \mu_3=\norm{y_3}_{\infty}=\frac{29}{42}, u_3=\frac{1}{29}\begin{pmatrix}
        29\\-16\\7
    \end{pmatrix}\\
    y_4=A^{-1}u_3=\frac{1}{261}\begin{pmatrix}
        390\\-1\\48
    \end{pmatrix}, & \mu_4=\norm{y_4}_{\infty}=\frac{65}{87}\approx 0.747, u_4=\frac{1}{390}\begin{pmatrix}
        390\\-1\\48
    \end{pmatrix}
\end{align*}
因此近似的特征向量是$\frac{1}{390}\begin{pmatrix}
    390\\-1\\48
\end{pmatrix}$
\end{solution}

\begin{problem}
14. 应用QR算法的基本迭代格式于矩阵
$$A=\begin{bmatrix}
    1&0\\1&-1
\end{bmatrix}$$ 
并考察所得矩阵序列的特点, 它是否收敛?  
\end{problem}
\begin{solution}
对于$2\times 2$的矩阵, 最方便的方法是使用Givens变换
\begin{align*}
    &\begin{pmatrix}
        c&s\\
        -s&c
    \end{pmatrix}\begin{pmatrix}
        1\\1
    \end{pmatrix}=\begin{pmatrix}
        \sqrt{2}\\0
    \end{pmatrix}\Rightarrow c=s=\frac{1}{\sqrt{2}}\\
    &A=A_1=Q_1R_1=\begin{pmatrix}
        \frac{1}{\sqrt{2}}&-\frac{1}{\sqrt{2}}\\
        \frac{1}{\sqrt{2}}&\frac{1}{\sqrt{2}}
    \end{pmatrix}\begin{pmatrix}
        \sqrt{2}&-\frac{1}{\sqrt{2}}\\
        0&-\frac{1}{\sqrt{2}}
    \end{pmatrix}\Rightarrow A_2=R_1Q_1=\begin{pmatrix}
        \frac{1}{2}&-\frac{3}{2}\\
        -\frac{1}{2}&-\frac{1}{2}
    \end{pmatrix}
\end{align*}
再进行QR分解:
\begin{align*}
    &\begin{pmatrix}
        c&s\\
        -s&c
    \end{pmatrix}\begin{pmatrix}
        \frac{1}{2}\\-\frac{1}{2}
    \end{pmatrix}=\begin{pmatrix}
        \frac{1}{\sqrt{2}}\\0
    \end{pmatrix}\Rightarrow c=s-=\frac{1}{\sqrt{2}}\\
    &A_2=Q_2R_2=\begin{pmatrix}
        \frac{1}{\sqrt{2}}&\frac{1}{\sqrt{2}}\\
        -\frac{1}{\sqrt{2}}&\frac{1}{\sqrt{2}}
    \end{pmatrix}\begin{pmatrix}
        \frac{1}{\sqrt{2}}&-\frac{1}{\sqrt{2}}\\
        0&-\sqrt{2}
    \end{pmatrix}\Rightarrow A_2=R_1Q_1=\begin{pmatrix}
        1&0\\1&-1
    \end{pmatrix} 
\end{align*}
因此所得的序列是周期二摇摆的, 不收敛.
\end{solution}

\begin{problem}
    23. $A\in R^{n\times n}$是一个具有互不相同对角元的上三角阵, 给出计算$A$全部特征向量的详细算法 
\end{problem}
\begin{solution}
    上三角矩阵的对角元就是特征值, 因此有$n$个互不相同的特征值.

    算法描述: 对每一个特征值, $i=1,\cdots,n$, 计算$A-\lambda_i I$, \textcolor{red}{利用反幂法, 在已知特征值的情况下, 来求矩阵$A$对应于特征值的特征向量.}

    对于$A-\lambda_i I$可逆的情况, 就直接利用反幂法的迭代即可得到对应的特征向量

    对于$A-\lambda_i I$不可逆的情况, 使用微小扰动, $A-(\lambda_i+\epsilon) I$, 再使用反幂法.
\end{solution}


\end{document}