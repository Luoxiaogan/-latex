\documentclass[10pt, a4paper, oneside]{ctexart}
\usepackage{amsmath, amsthm, amssymb, bm, color, xcolor, framed, graphicx, hyperref, mathrsfs, etoolbox, wrapfig}
\usepackage[thicklines]{cancel}
\usepackage{enumitem} % 用于更灵活的列表环境
\usepackage{geometry} % 调整页面边距
\usepackage{fancyhdr} % 页眉页脚定制
\hypersetup{
    colorlinks=true,            %链接颜色
    linkcolor=black,             %内部链接
    filecolor=magenta,          %本地文档
    urlcolor=cyan,              %网址链接
    pdftitle={Overleaf Example},
    pdfpagemode=FullScreen,
    }

% 页边距设置
\geometry{left=3cm, right=3cm, top=2.5cm, bottom=2.5cm}

% 页眉页脚设置
\pagestyle{fancy}
\fancyhf{}
\fancyhead[L]{\leftmark} % 左页眉显示章节标题
\fancyhead[R]{\thepage} % 右页眉显示页码

% 标题设置
\title{\textbf{数分三}}
\author{Little Wolf}
\date{\today}

% 行距设置
\linespread{1.2}

% 定义黑色边框的 problem 环境
\newenvironment{problem}{\begin{framed}\par\noindent\textbf{\textit{题目. }}}{\end{framed}\par}
\newenvironment{solution}{%
  \par\noindent\textbf{\textit{解答. }}\ignorespaces
}{%
  \hfill\ensuremath{\square}\par % 在结尾添加正方形
}
\newenvironment{note}{\par\noindent\textbf{\textit{题目的注记. }}\ignorespaces}{\par}


% 允许公式在页面之间自动换行
\allowdisplaybreaks

\begin{document}

\maketitle

% 添加目录
\tableofcontents
\newpage

\section{王冠香补充题目}
\begin{problem}
    设$A,B$是$\mathbb{R}^n$的互不相交的闭集, 证明: 存在开集$O_1,O_2, \quad s.t.\quad A\subset O_1, B\subset O_2, O_1\cap O_2 = \varnothing$.
\end{problem}

\begin{solution}
    $d(x,A)=\inf\{|x-a|:a\in A\}$
\begin{align*}
    O_1&=\{x|\frac{d(x,A)}{d(x,A)+d(x,B)}<\frac{1}{2}\}=\{x|d(x,A)<d(x,B)\}\\
    O_2&=\{x|\frac{d(x,B)}{d(x,A)+d(x,B)}<\frac{1}{2}\}=\{x|d(x,B)<d(x,A)\}
\end{align*}
对任意的$x^*\in X_1$, $d(x^*,A)<d(x^*,B)$, 那么取$\delta_0=\frac{d(x^*,B)-d(x^*,A)}{4}$, $\forall x\in U(x^*,\delta_0)$, 有$d(x,A)\leq d(x^*,A)+\delta < d(x^*,B)-\delta\leq d(x,B)$, 即$U(x^*,\delta_0)\subset O_1$, 因此$O_1$是开集. 同理, $O_2$是开集.\\
根据定义(因为两个严格的不等式不能同时成立), $O_1\cap O_2=\varnothing$\\
因为$\forall x\in A, d(x,A)=0$, 而$A,B$是不相交的闭集, \textcolor{blue}{所以$B\subset A^c$, 且$A^c$是开集, 因此$\forall x\in A, \forall y\in B, \exists  \delta>0$, 使得$d(y,A)>\delta, \forall y\in B$, 从而有$d(x,B)>\delta$, 因此$x\in O_1\Rightarrow A\subset O_1$, 同理$B\subset O_2$}.
\end{solution}
\begin{note}
两个互不相交的闭集$A,B$, 因为$B\subset A^c$, $A^c$闭集, 所以$\forall b\in B, \exists \delta_b>0, U(b,\delta_b)\subset A^c$, 因此$\forall x\in A$取定, $|x-b|>\delta_b>0$.\\
考虑下确界$\inf\{|x-b|:b\in B\}$, 如果下确界等于$0$, 那么显然$x\in \partial A$(否则如果是内点, 上述下确界必然大于$0$); 但如果下确界等于$0$, 那么必然有一个$B$中的子列趋于$x$, 但$B$是闭集, 包含自身的极限点, 得到$x\in B$, 矛盾. 因此下确界一定大于$0$.\\ 
两个互不相交的闭集$A,B$, 单点到另一个集合的距离的下确界是正的.\\
两个互不相交的闭集$A,B$, 集合中任意一点到另一个集合的距离的下确界不一定是正的.\\
\textcolor{blue}{实际上, 闭集的性质本身保证了上述定义的点到集合的距离, 即下确界, 是可以被取到的.}
\end{note}

\section{多元函数的极限和连续}

\begin{problem}
1. 证明$\mathbb{R}^n$中两点距离满足三角不等式:对于$\forall \bm{x},\bm{y},\bm{z}\in \mathbb{R}^n$,有$|\bm{x}-\bm{z}|\leq |\bm{x}-\bm{y}|+|\bm{y}-\bm{z}|$
\end{problem}

\begin{solution}
设$a_i=x_i-y_i, b_i=y_i-z_i$, 要证: $|\bm{x}-\bm{z}|\leq |\bm{x}-\bm{y}|+|\bm{y}-\bm{z}|$, 即
\begin{align*}
    &|\bm{x}-\bm{z}|\leq |\bm{x}-\bm{y}|+|\bm{y}-\bm{z}| \iff \sqrt{\sum_{i=1}^n (a_i+b_i)^2}\leq \sqrt{\sum_{i=1}^n a_i^2}+\sqrt{\sum_{i=1}^n b_i^2}\\\iff& \sum_{i=1}^n a_ib_i\leq \sqrt{\sum_{i=1}^n a_i^2}\sqrt{\sum_{i=1}^n b_i^2}\iff \vec{a}\cdot \vec{b}\leq |\vec{a}|\cdot|\vec{b}|
\end{align*}
\end{solution}
\begin{note}
    直接硬证有点困难,尝试对要证明的结论做等价变形.
\end{note}

\begin{problem}
2. 若 $\lim _{k \rightarrow \infty}\left|\boldsymbol{x}_k\right|=+\infty$, 则称 $\mathbb{R}^n$ 中的点列 $\left\{\boldsymbol{x}_k\right\}$ 趋于 $\infty$. 现在设点列 $\left\{\boldsymbol{x}_k=\left(x_1^k, x_2^k, \cdots, x_n^k\right)\right\}$ 趋于 $\infty$, 试判断下列命题是否正确:

(1) 对于 $\forall i(1 \leqslant i \leqslant n)$, 序列 $\left\{x_i^k\right\}$ 趋于 $\infty$;

(2) $\exists i_0\left(1 \leqslant i_0 \leqslant n\right)$, 序列 $\left\{x_{i_0}^k\right\}$ 趋于 $\infty$.
\end{problem}
\begin{solution}
    (1) 不正确, 反例: $\bm{x}^k=(k,0,0,\cdots,0)$, 那么对$2\leq i\leq n$, 有$x_i^k\equiv 0$.
    
    (2) 不正确, 反例: 记$t\equiv k (\mod n)$, 设$\bm{x}^k$的第$t$个元素是$k$其余为$0$, 那么满足条件, 但$\forall i, 1\leq i \leq n$, 都有$x_{i}^k$在充分大的$K$后无限次取$0$,因此不可能趋于$\infty$.
\end{solution}

\begin{problem}
3. 求下列集合的聚点集:

(1) $E=\left\{\left(\frac{q}{p}, \frac{q}{p}, 1\right) \in \mathbb{R}^3: p, q \in \mathbb{N}\right.$ 互素, 且 $\left.q<p\right\}$;

(2) $E=\left\{\left(\ln \left(1+\frac{1}{k}\right)^k, \sin \frac{k \pi}{2}\right): k=1,2, \cdots\right\}$;

(3) $E=\left\{\left(r \cos \left(\tan \frac{\pi}{2} r\right), r \sin \left(\tan \frac{\pi}{2} r\right)\right) \in \mathbb{R}^2: 0 \leqslant r<1\right\}$.
\end{problem}
\begin{solution}
    (1)$E^{\prime}=\{(x,x,1)|x\in[0,1]\}$;

    (2) \textcolor{blue}{$\ln(1+\frac{1}{k})^k \sim (\frac{1}{k}-\frac{1}{2k^2}+o(\frac{1}{k^2}))^k \to 1(k\to \infty)$}. $\sin \frac{k\pi}{2}$的聚点集是$\{-1,0,1\}$. 因此$E^{\prime}=\{(1,-1),(1,0),(1,1)\}$;

    (3)\textcolor{red}{$E^{\prime}=\{(x,y)|x^2+y^2=1\}\textcolor{blue}{\cup E}$}. 因为$\lim_{r\to 1}r\cos(\tan\frac{\pi}{2}r)$极限并不存在, 但分析渐进性质可以知道, $\tan\frac{\pi}{2}r\to \infty$, 将$\tan\frac{\pi}{2}r$看成一个以半径$r$为自变量的角度参数, 那么当半径$r\to 1$的时候, 角度会转无数圈, 单位圆周成为聚点集. \textcolor{blue}{又因为$E$本身是连续曲线, 所以$\forall x\in E$, $x$当然是$E$的聚点.}
\end{solution}

\begin{problem}
4. 求下列集合的内部、外部、边界及闭包:

(1) $E=\left\{(x, y, z) \in \mathbb{R}^3: x>0, y>0, z=1\right\}$ ;

(2) $E=\left\{(x, y) \in \mathbb{R}^2: x>0, x^2+y^2-2 x>1\right\}$.
\end{problem}
\begin{solution}
    (1) "一张纸".\\内部$E^{o}=\varnothing$\\外部$(E^{c})^{o}=\mathbb{R}^n \backslash \{(x,y,1)|\textcolor{red}{x\geq 0 , y\geq 0}\}$(注意要把包含$0$的部分也去掉)\\边界$\partial E= \overline{E} = \{(x,y,1)|x\geq 0, y\geq 0\}$.

    (2) $x^2+y^2-2x>1\iff (x-1)^2+y^2>(\sqrt{2})^2$, 即扣去一个开圆盘留下的区域. 又$x>0$, 只看$x$正半轴的部分.\\内部$E^{o}=E=\{(x,y)|x>0, x^2+y^2-2x\textcolor{red}{>} 1\}$\\外部$(E^{c})^{o}=\mathbb{R}^3\backslash \{(x,y)|x\textcolor{red}{\geq}0,x^2+y^2-2x\textcolor{red}{\geq}1\}$(补集的内部,把$E$补成闭集之后扣掉)\\边界$\partial E=\{(x,y)|x^2+y^2-2x=1\}\cup \{(0,y)|y^2\geq 1\}$\\闭包$\overline{E}=\{(x,y)|x\geq 0, x^2+y^2-2x\geq 1\}$.
\end{solution}

\begin{problem}
5. 设 $\left\{\left(x_k, y_k\right)\right\} \subset \mathbb{R}^2$ 是一个点列, 判断如下命题是否为真:点列 $\left\{\left(x_k, y_k\right)\right\}$ 在 $\mathbb{R}^2$ 中有聚点的充分必要条件是 $\left\{x_k y_k\right\}$ 在 $\mathbb{R}$ 中有聚点.
\end{problem}
\begin{solution}\textcolor{red}{下面是错误的分析}: \\$\{(x_k,y_k)\}$有聚点$\iff$存在子列收敛$\{(x_{n_k},y_{n_k})\}\to (a,b)\Rightarrow \{x_{n_k}y_{n_k}\}\to ab \iff \{x_ky_k\}$有聚点.\\
\textcolor{red}{反例, 既不充分也不必要}: \\
(1) $\{(0,\frac{1}{k})\}$有极限(当然有聚点)$(0,0)$, 但$0\cdot \frac{1}{k}=0$是单点集, \textcolor{blue}{单点集没有聚点(这是我没有想到的)}
$$\{(x_n,y_n)\}\text{有聚点}\quad \text{不能推出}\quad \{x_ny_n\}\text{有聚点}$$
(2) $\{(k+1, \frac{1}{k})\}$没有聚点(因为$x$之间至少差了$1$!), 而$\{\frac{k+1}{k}\}$有极限(有聚点)$1$.
$$\{x_ny_n\}\text{有聚点}\quad \text{不能推出}\quad \{(x_n,y_n)\}\text{有聚点}$$
\end{solution}
\begin{note}
    极限点不一定是聚点, 因为极限点可以是整个序列取单点集: $1\to 1$\\
    而聚点的要求是: 一定要有无穷多个点(这是定义的区别)
\end{note}

\begin{problem}
6. 设 $E \subset \mathbb{R}^n$, 证明:

(1) $\bar{E}=E^{\circ} \cup \partial E$;

(2) $E^{\prime}=\bar{E}^{\prime}$
\end{problem}
\begin{solution}
    \textcolor{blue}{证明等号, 左边属于右边, 右边属于左边.}\\
(1) 方法一:$(\overline{E})^{c}=(E^{c})^{o}=(E^{o}\cup \partial E)^c \Rightarrow \overline{E}=E^{o}\cup \partial E$.\\
方法二:先证明$\overline{E}\subset E^{o}\cup \partial E$. 任取$x\in \overline{E}$, 如果$x\in E^{o}$, 当然有$x\in E^{o}\cup \partial E$; 如果$x\notin E^{o}$, 那么$x\in E\backslash E^{o} \textcolor{red}{\text{就是}}\partial E$, 因此有$\overline{E}\subset E^{o}\cup \partial E$. \textcolor{blue}{再证明$E^{o}\cup \partial E\subset \overline{E}$.}\\
(2) $E^{\prime}\subset \overline{E}^{\prime}$很好证明, 因为$E\subset \overline{E}$, 所以$E^{\prime}$中任取一点$x\in E^{\prime}$, 一定是$E$中子列的极限点, 当然也就是$\overline{E}$中子列的极限点, 因此$x\in \overline{E}^{\prime}$, 因此$E^{\prime}\subset \overline{E}^{\prime}$.\\
另一方面, 来证明$\overline{E}^{\prime}\subset E^{\prime}$.\textcolor{blue}{根据书上对闭包的定义,$\overline{E}=E\cup E^{\prime}$, 因此$\overline{E}^{\prime}=E^{\prime}\cup (E^{\prime})^{\prime}$, 因此只需要证明$(E^{\prime})^{\prime}\subset E^{\prime}$}.\\
\textbf{\textcolor{red}{方法一}}: 根据极限点的定义, $\forall x\in (E^{\prime})^{\prime}, \textcolor{red}{\forall}  \delta>0,\quad s.t. \quad U_0(x,\frac{\delta}{2})\cap E^{\prime} \neq \varnothing$; $\forall x^{\prime} \in U_0(x,\frac{\delta}{2})\cap E^{\prime} $(注意,取自上面的交集), 因为$x^{\prime}\in E^{\prime}$, 所以$\textcolor{red}{\forall}  \delta>0, \quad s.t. \quad U_0(x^{\prime},\frac{\delta}{2}) \cap E \neq \varnothing$. 即$|x-x^{\prime}|<\frac{\delta}{2}$, 且$\exists x^{\prime \prime}\in  U_0(x^{\prime},\frac{\delta}{2}) \cap E,\quad s.t. \quad |x^{\prime}-x^{\prime \prime}|<\frac{\delta}{2}$, 从而根据三角不等式, $|x-x^{\prime \prime}|<\delta$, 即$U_0(x,\delta)\cap E \neq \varnothing$. 由$\delta$的任意推出$x\in E^{\prime} \Rightarrow (E^{\prime})^{\prime}\subset E^{\prime}$.\\
\textbf{\textcolor{red}{方法二}}: 根据极限点的定义, $\forall x\in (E^{\prime})^{\prime}, \exists \{x_n\}\in E^{\prime}, \quad s.t. \quad x_n\to x$. 即$\forall \delta>0, \exists N_1>0, \quad s.t. \forall n>N_1, \quad |x-x_n|<\frac{\delta}{2}$. 任取一个满足$|x-x_n|<\frac{\delta}{2}$的$x_{n_0}$, 因为$x_{n_0}\in E^{\prime}$, $\exists \{y_n\}\in E,\quad s.t. \quad y_n\to x_{n_0}$, 即对上面相同的$\delta>0$, $\exists N_2>0, \forall n>N_2$, $s.t. \quad |x_{n_0}-y_n|<\frac{\delta}{2}$. 任取上述满足条件的一个$y_{n_1}$, 通过三角不等式得到$|x-y_{n_1}|\leq |x-x_{n_0}|+|x_{n_0}-y_{n_1} |<\delta$, $\forall n\geq N_1+N_2$, 得证. 
\end{solution}
\begin{note}
    (1) 书中的定义是:$\overline{E}=E\cup E^{\prime}$, 另一种定义: $\partial E=\overline{E}\backslash E^{o}$, 即$\overline{E}=E^{o}\cup \partial E$\\
    (2) 导集的理解:
    \begin{itemize}
        \item $\forall x\in E^{\prime}, \exists \{x_n\}\in E, \quad s.t. \quad x_n \to x$. (作为一个子列的极限点, \textcolor{red}{可以从这个角度得到\textbf{\textcolor{red}{方法二}}})
        \item $\forall x\in E^{\prime}, \textcolor{red}{\forall} \delta>0, \quad s.t. \quad U_0(x,\delta)\cup E\neq \varnothing$. (从邻域的角度)
    \end{itemize}
\end{note}

\begin{problem}
7. 设 $\left\{A_\lambda\right\}_{\lambda \in \Lambda}$ 为 $\mathbb{R}^n$ 的一族集合, 证明:

(1) 当 $\Lambda$ 为有限指标集时, 成立 $\overline{\bigcup_{\lambda \in \Lambda} A_\lambda} \subseteq \bigcup_{\lambda \in \Lambda} \overline{A_\lambda}, \bigcap_{\lambda \in \Lambda} A_\lambda^{\circ} \subseteq\left(\bigcap_{\lambda \in \Lambda} A_\lambda\right)^{\circ}$;

(2) 对任意的指标集, 成立 $\bigcup_{\lambda \in \Lambda} A_\lambda^{\circ} \subseteq\left(\bigcup_{\lambda \in \Lambda} A_\lambda\right)^{\circ}, \overline{\bigcap_{\lambda \in \Lambda} A_\lambda} \subseteq \bigcap_{\lambda \in \Lambda} \overline{A_\lambda}$.
\end{problem}

\begin{solution}
(1) $A_{\lambda}\subset \overline{A_{\lambda}}$, 故$\bigcup_{\lambda\in \Lambda} A_{\lambda}\subset \bigcup_{\lambda\in \Lambda}\overline{A_{\lambda}}$, 所以$\overline{\bigcup_{\lambda\in \Lambda} A_{\lambda}}\subset \overline{\bigcup_{\lambda\in \Lambda}\overline{A_{\lambda}}}$, 又因为指标集有限, 因此$\overline{\bigcup_{\lambda\in \Lambda}\overline{A_{\lambda}}}=\bigcup_{\lambda\in \Lambda}\overline{A_{\lambda}}$, 第一部分得证.\\
而$\bigcap_{\lambda \in \Lambda}A_{\lambda}^{o}=(\bigcup_{\lambda\in \Lambda} \overline{A_{\lambda}^{c}})^{c}\textcolor{red}{\subset}(\overline{\bigcup_{\lambda\in \Lambda} A_{\lambda}^{c}})^{c}=(\bigcap_{\lambda\in \Lambda}A_{\lambda})^{o}$.\\
(2) $\overline{A_{\lambda}}$闭集, 无穷闭集的交还是闭集, $\bigcap_{\lambda\in \Lambda}\overline{A_{\lambda}}$是闭集, 因此有$\overline{\bigcap_{\lambda\in \Lambda}A_{\lambda}}\subset \overline{\bigcap_{\lambda\in \Lambda}\overline{A_{\lambda}}}=\bigcap_{\lambda\in \Lambda}\overline{A_{\lambda}}$.\\
而$\bigcup_{\lambda\in \Lambda}A_{\lambda}^{o}=(\bigcup_{\lambda\in \Lambda}\overline{A^c_{\lambda}})^{c}\subset (\overline{\bigcup_{\lambda\in \Lambda}A_{\lambda}^c})^c=(\bigcap_{\lambda\in \Lambda}A_{\lambda})^{o}$
\end{solution}

\begin{problem}
8. 设 $E \subset \mathbb{R}^n$ ,证明:

(1) $E^{\prime}$ 是闭集;

(2) $\partial E$ 是闭集.
\end{problem}

\begin{solution}
(1) 即证明: $E^{\prime}=\overline{E^{\prime}}$, 而$\overline{E^{\prime}}=E^{\prime}\cup (E^{\prime})^{\prime}$, 显然$E^{\prime}\subset \overline{E^{\prime}}$, 又根据6题的结论, $(E^{\prime})^{\prime}\subset E^{\prime}$, 得证.\\
(2) 即证明: $\partial E = \overline{\partial E}=\partial E\cap (\partial E)^{\prime}$, 即证明$(\partial E)^{\prime}\subset \partial E$.\\
\textbf{\textcolor{red}{方法一}}: $E^{o}$是开集, $(E^{c})^{o}$是开集, 那么$E^{o}\cup (E^{c})^{o}$是开集, 那么$\mathbb{R}^n \backslash (E^{o}\cup (E^{c})^{o}) = \partial E$是闭集 (边界$E$理解成, 既不属于$E$的内部$E^{o}$, 也不属于补集的内部$(E^{c})^{o}$的部分).\\
\textbf{\textcolor{red}{方法二}}: (\textcolor{blue}{直接证明$(\partial E)^{\prime}\subset \partial E$.}) 考虑$\partial E=\overline{E}\backslash E^{o}$, 那么$(\overline{E})^{\prime}=(\partial E)^{\prime}\cup (E^{o})^{\prime}$, 根据第六题的结论, $(\overline{E})^{\prime}=\overline{E}$, 因此$\overline{E}=\partial E \cup E^{o}= (\partial E)^{\prime}\cup (E^{o})^{\prime}$, 因此$\forall x\in (\partial E)^{\prime}$, 只可能属于$\partial E$或者$E^{o}$. 采用反证法, 若$x\in E^{o}$, 根据极限点定义, $\textcolor{red}{\forall} \delta>0, U_0(x,\delta)\cap \partial E\neq \varnothing$, 但根据$E^{o}$是开集的定义, 充分小的$\delta$可以使$U_0(x,\delta)\subset E^{o}\Rightarrow U_0(x,\delta)\cap \partial E=\varnothing$, 矛盾. 
\end{solution}
\begin{note}
    (1) $(E^{\prime})^{\prime}\subset E^{\prime}$, $(\partial E)^{\prime}\subset \partial E$.\\
    (2) $\overline{E}=\partial E\cup E^{o} = E\cup E^{\prime}$\\
    (3) \textcolor{red}{问题: $\overline{E}=\partial E\cup E^{o} $的两边取导集, 还是可以得到等式$(\overline{E})^{\prime}=(\partial E)^{\prime}\cup (E^{o})^{\prime}$. 但是如果写成$\partial E=\overline{E}\backslash E^{o}$, 还可以两边取导集吗?}
\end{note}

\begin{problem}
9. 设 $E \subset \mathbb{R}^2$, 记 $E_1=\{x \in \mathbb{R}: \exists(x, y) \in E\}, E_2=\{y \in \mathbb{R}$ : $\exists(x, y) \in E\}$ ,判断下列命题是否为真 (说明理由):

(1) $E$ 为 $\mathbb{R}^2$ 中的开 (闭) 集时, $E_1$ 和 $E_2$ 均为 $\mathbb{R}$ 中的开 (闭) 集;

(2) $E_1$ 和 $E_2$ 均为 $\mathbb{R}$ 中的开 (闭) 集时, $E$ 为 $\mathbb{R}^2$ 中的开 (闭) 集。
\end{problem}

\begin{problem}
10. 构造 $\mathbb{R}^2$ 中单位圆盘 $\Delta=\left\{(x, y): x^2+y^2<1\right\}$ 内的一个点列 $\left\{\left(x_k, y_k\right)\right\}$, 使得它的点构成的集合的聚点集恰为单位圆周 $\partial \Delta$.
\end{problem}

\begin{solution}
考虑$\{(r_k\cos \theta_k, r_k \sin \theta_k)\}$, 当$r_k\to 1$时, 趋于$(\cos\theta, \sin\theta)$, 借鉴3(3)的思想, 构造$\theta$序列作为$r$的函数, 使得$r\to 1$的过程中, $\theta\to \infty$. 例如: $\{(r_k\cos(\tan \frac{\pi}{2}r_k),r_k\sin(\tan \frac{\pi}{2}r_k))\}$, 其中$r_k=\frac{k}{k+1}$, i.e., $\{(\frac{k}{k+1}\cos(\tan \frac{k\pi}{2(k+1)}),\frac{k}{k+1}\sin(\tan \frac{k\pi}{2(k+1)}))\}$.\\
\textcolor{blue}{和前面的3的区别是, 因为我这里构造的是离散点列而不是连续的线, 所以不用担心$E$本身也是导集的子集.}
\end{solution}
\begin{note}
    \textcolor{red}{问题: 除了构造$r_k\to 1$的同时, $\theta_k$可以与$r_k$独立地定义, 如果$\theta_k$的定义只是保证趋于有限$(\cos \theta, \sin \theta)$, 那么只能保证聚点是$\partial \Delta$的有限点, 即使以可列方式组合之后成大序列, 还是不能遍历不可数集, 那么$\theta_k$的定义必须保证趋于$(\infty, \infty)$吗?}
\end{note}

\begin{problem}
11. 设 $E_1, E_2 \subset \mathbb{R}^n$ 为两个非空集合, 定义 $E_1, E_2$ 间的距离如下:
$$
d\left(E_1, E_2\right)=\inf _{\boldsymbol{x} \in E_1, \boldsymbol{y} \in E_2}|\boldsymbol{x}-\boldsymbol{y}|
$$
(1) 举例说明存在开集 $E_1, E_2$, 使得 $E_1 \cap E_2=\varnothing$, 但 $d\left(E_1, E_2\right)=0$;

(2) 举例说明存在闭集 $E_1, E_2$, 使得 $E_1 \cap E_2=\varnothing$, 但 $d\left(E_1, E_2\right)=0$;

(3) 证明: 若紧集 $E_1, E_2$ 满足 $d\left(E_1, E_2\right)=0$, 则必有 $E_1 \cap E_2 \neq \varnothing$.
\end{problem}

\begin{problem}
12. 设 $F \subset \mathbb{R}^n$ 是紧集, $E \subset \mathbb{R}^n$ 是开集, 且 $F \subset E$ 。证明:存在开集 $O$ ,使得 $F \subset O \subset \bar{O} \subset E$ 。
\end{problem}

\begin{problem}
13. 求下列函数的定义域:

(1) $f(x, y, z)=\ln \left(y-x^2-z^2\right)$;

(2) $f(x, y, z)=\sqrt{x^2+y^2-z^2}$;

(3) $f(x, y, z)=\frac{\ln \left(x^2+y^2-z\right)}{\sqrt{z}}$.
\end{problem}
\begin{solution}
(1) $\{(x,y,z)| y\> x^2+z^2\}$\\
(2) $\{(x,y,z)| z^2\geq x^2+y^2\}$\\
(3) $\{(x,y,z)| x^2+y^2>z>0\}$
\end{solution}

\begin{problem}
14. 确定下列函数极限是否存在, 若存在则求出极限:

(1) $\lim _{E \ni(x, y) \rightarrow(0,0)} \frac{\sin \left(x^3+y^3\right)}{x^2+y}$, 其中 $E=\left\{(x, y): y>x^2\right\}$;

(2) $\lim _{(x, y) \rightarrow(0,0)} x \ln \left(x^2+y^2\right)$;

(3) $\lim _{|(x, y)| \rightarrow+\infty}\left(x^2+y^2\right) \mathrm{e}^{-(|x|+|y|)}$;

(4) $\lim _{|(x, y)| \rightarrow+\infty}\left(1+\frac{1}{|x|+|y|}\right)^{\frac{x^2}{|x|+|y|}}$;

(5) $\lim _{(x, y, z) \rightarrow(0,0,0)}\left(\frac{x y z}{x^2+y^2+z^2}\right)^{x+y}$;

(6) $\lim _{E \ni(x, y, z) \rightarrow(0,0,0)} x^{y z}$, 其中 $E=\{(x, y, z): x, y, z>0\}$;

(7) $\lim _{(x, y, z) \rightarrow(0,1,0)} \frac{\sin (x y z)}{x^2+z^2}$

(8) $\lim _{(x, y, z) \rightarrow(0,0,0)} \frac{\sin x y z}{\sqrt{x^2+y^2+z^2}}$

(9) $\lim _{\boldsymbol{x} \rightarrow \mathbf{0}} \frac{\left(\sum_{i=1}^n x_i\right)^2}{|\boldsymbol{x}|^2}$.
\end{problem}
\begin{solution}
(1) $\frac{\sin(x^3+y^3)}{x^2+y}=\frac{x^3+y^3+o(x^3+y^3)}{x^2+y}$. \\如果$\lim_{(x,y)\to (0,0)}\frac{x^3+y^3}{x^2+y}=0$, 那么当然有$\lim_{(x,y)\to (0,0)}\frac{o(x^3+y^3)}{x^2+y}=0$
首先考虑对分子配方, 使得最后留在分子的只有$x$.
\begin{align*}
    y^3=(x^2+y)y^2-x^2y^2=(x^2+y)(y^2+x^2y)-x^4y=(x^2+y)(y^2+x^2y+x^4)-x^6
\end{align*}
因此有
\begin{align*}
    |\frac{x^3+y^3}{x^2+y}|\leq |y^2+x^2y+x^4|+|\frac{x^3(1-x^3)}{x^2+y}|
\end{align*}
对$|x^2+y|\geq |x^2-|y||$, \textcolor{blue}{即使$y\to 0$, 我也不能取$|y|\leq \frac{x^2}{2}$, 因为这样就不是从各个方向来趋近于$(0,0)$了. 当然, 如果$x$是趋于一个非零的数, 我是可以这么做的.}\\
或许可以这样做: \textcolor{red}{如果$|y|>2x^2$, 那么$|x^2+y|\geq x^2$; 如果$|y|\leq \frac{x^2}{2}\leq 2x^2$, 那么$|x^2+y|\geq \frac{x^2}{2}$. 总之, $|x^2+y|\geq 2x^2$.}\\
因此有
\begin{align*}
    |\frac{x^3(1-x^3)}{x^2+y}|=\frac{|x^3(1-x^3)|}{|x^2+y|}\leq \frac{|x^3(1-x^3)|}{2x^2}=\frac{|x(1-x^3)|}{2}\to 0
\end{align*}
\textcolor{red}{这是在没有考虑题目给出的$y>x^2$的条件下做的, 如果有这个条件, 当然好做了}:
\begin{align*}
    |\frac{x^3(1-x^3)}{x^2+y}|\leq |\frac{x^3(1-x^3)}{2x^2}|=|x(1-x^3)|\to 0
\end{align*}
\end{solution}
\begin{note}
主要是因为分母是$x^2+y$, 非齐次导致不好操作. 否则可以极坐标换元\\
之所以对分子配方把分子上的$y$全部移除是为了后面对分母做完操作之后全部都是$x$就好办了.(之所以不去消去$x$是因为多出来的$xy$配方消不掉)\\
分类讨论来给出分母的下界这一点很有意思.
\end{note}
\begin{solution}
(2) \textcolor{blue}{看见$x^2+y^2$, 比较trivial地可以想到极坐标换元}. 
\begin{align*}
    x\ln(x^2+y^2)=2r\ln(r)\cdot \cos\theta\to 0
\end{align*}
(3) \textcolor{red}{考虑放缩之后\textbf{整体换元}, 这样就可以使用洛必达了(虽然换元之后就显然了)}
\begin{align*}
    \frac{x^2+y^2}{e^{|x|+|y|}}\leq \frac{(|x|+|y|)^2}{e^{|x|+|y|}} = \frac{t^2}{e^t}\to 0 , \quad t=|x|+|y|\to 0
\end{align*}
(4) 极限不存在, 首先取$x\equiv 0, y\to +\infty$的路径, 有极限为$1$(实际上恒等于$1$). 如果取$x=y\to +\infty$的路径, 那么
\begin{align*}
    (1+\frac{1}{2|x|})^{2|x|}=(1+\frac{1}{2|x|})^{2|x| \cdot \frac{1}{4}}\to e^{\frac{1}{4}}
\end{align*}
因此, 极限不存在\\
(5) 考虑点列$(\frac{1}{t},0,0), t\in \mathbb{N}^*$, 那么$\lim_{t\to\infty}o^{t}=0$; 点列$(\frac{1}{t},\frac{1}{t},\frac{1}{t})$\\那么$\lim_{t\to\infty}(\frac{3}{t})^{\frac{2}{t}}=\lim_{t\to\infty} e^{\frac{2}{t}\ln(\frac{3}{t})} \lim_{k\to 0^+}e^{2k\ln(3k)}=1$, 极限不存在.\\
(6) 点列$(0,\frac{1}{t},\frac{1}{t})$, 极限$\lim_{t\to\infty} 0^{\frac{1}{t^2}}=0$; 点列$(\frac{1}{t},\frac{1}{t},\frac{1}{t})$, 极限$\lim_{t\to\infty} (\frac{1}{t})^{\frac{1}{t^2}}=1$, 极限不存在\\
(7) 点列$(\frac{1}{t},\frac{t}{t+1},\frac{1}{t})$, 极限$\lim_{t\to\infty}\frac{\sin(\frac{1}{t(t+1)})}{\frac{2}{t^2}}=\frac{1}{2}$. 点列$(\frac{1}{t},\frac{t}{t+1},\frac{2}{t})$, 极限$\lim_{t\to\infty}\frac{\sin(\frac{2}{t(t+1)})}{\frac{5}{t^2}}=\frac{2}{5}$, 极限不存在.\\
(8) \textcolor{red}{极限存在, 注意和前几问的重大区别, 从渐进角度来看, 大概是$\frac{xyz}{\sqrt{x^2+y^2+z^2}}$, 分子的次数更大, 因此会趋于$0$!}\\
\textcolor{blue}{注意$|\sin t|\leq |t|$恒成立}
\begin{align*}
    |\frac{\sin(xyz)}{\sqrt{x^2+y^2+z^2}}|\leq |\frac{xyz}{\sqrt{x^2+y^2+z^2}}|
\end{align*}
考虑三维的球坐标换元$\begin{cases}
    x&=r\sin \theta\cos \phi\\
    y&=r \sin \theta \sin \phi\\
    z&= r\cos \phi
\end{cases}$, 那么
\begin{align*}
    |\frac{xyz}{\sqrt{x^2+y^2+z^2}}|=r^2\cdot|\sin\theta\cos \phi \sin \theta \sin \phi\cos \phi|\leq r^2 \to 0
\end{align*}
\textcolor{blue}{或者,使用基本不等式:
\begin{align*}
\frac{n}{\sum_{i=1}^n \frac{1}{x_i}}\leq \sqrt[n]{\Pi_{i=1}^n x_i}\leq \frac{\sum_{i=1}^n x_i}{n} \leq \sqrt[n]{\frac{\sum_{i=1}^n x_i}{n}}
\end{align*} }
那么有$\sqrt{x^2+y^2+z^2}\geq \sqrt{3}\cdot \sqrt[3]{xyz}$, 所以有
\begin{align*}
    |\frac{xyz}{\sqrt{x^2+y^2+z^2}}|\leq |\frac{xyz}{\sqrt{3}\cdot \sqrt[3]{xyz}}|=\frac{1}{\sqrt{3}}\cdot (xyz)^{\frac{2}{3}}\to 0
\end{align*}
(8) 点列$(\frac{1}{t},0,\cdots,0)$, 极限$\lim_{t\to \infty}\frac{\frac{1}{t^2}}{\frac{1}{t^2}}=1$\\点列$(\frac{1}{t},\frac{1}{t}, 0,\cdots,0)$, 极限$\lim_{t\to \infty}\frac{\frac{4}{t^2}}{\frac{2}{t^2}}=2$, 极限不存在.
\end{solution}


\begin{problem}
15. 试给出三元函数 $f(x, y, z)$ 累次极限 $\lim _{x \rightarrow x_0} \lim _{y \rightarrow y_0} \lim _{z \rightarrow z_0} f(x, y, z)$ 的定义, 并构造一个三元函数 $f(x, y, z)$, 使得它满足: $\lim _{(x, y, z) \rightarrow(0,0,0)} f(x, y, z)$存在, 但 $\lim _{x \rightarrow 0} \lim _{y \rightarrow 0} \lim _{z \rightarrow 0} f(x, y, z)$ 不存在.
\end{problem}
\begin{solution}
三元函数累次极限的定义: 设函数$w = f(x,y,z)$在$E\subset \mathbb{R}^3$上有定义\\
且邻域$U_0( (x_0,y_0,z_0), \delta  )\subset E$.\\
若在$U_0( (x_0,y_0,z_0), \delta )$内, 对每一个固定的$x\neq x_0, y\neq y_0$, 有$\lim_{z\to z_0}f(x,y,z)= \varphi(x,y)$存在\\
且(二元函数的累次极限已经定义了, 直接调用)$\lim_{x\to x_0}\lim_{y\to y_0}\varphi(x,y)=A$\\
则有$\lim_{x\to x_0}\lim_{y\to y_0}\lim_{z\to z_0}f(x,y,z)=A$.\\
\textcolor{blue}{构造:
$$f(x,y,z)=x+z+y\sin \frac{1}{z}$$
重极限$\lim_{(x,y,z)\to (0,0,0) }f(x,y,z)=0$存在, 但是累次极限不存在, 因为$z\to 0 $时就已经无穷了.}
\end{solution}

\begin{problem}
16. 设 $y=f(x)$ 在 $U_0\left(0, \delta_0\right) \subset \mathbb{R}$ 中有定义, 满足 $\lim _{x \rightarrow 0} f(x)=0$, 且对于 $\forall x \in U_0\left(0, \delta_0\right)$, 有 $f(x) \neq 0$. 记 $E=\{(x, y): x y \neq 0\}$, 证明:

(1) $\lim _{E \ni(x, y) \rightarrow(0,0)} \frac{f(x) f(y)}{f^2(x)+f^2(y)}$ 不存在;
\end{problem}

\end{document}

