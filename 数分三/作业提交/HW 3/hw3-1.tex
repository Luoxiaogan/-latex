\documentclass[10pt, a4paper, oneside]{ctexart}
\usepackage{amsmath, amsthm, amssymb, bm, color, xcolor, framed, graphicx, hyperref, mathrsfs, etoolbox, wrapfig}
\usepackage[thicklines]{cancel}
\usepackage{enumitem} % 用于更灵活的列表环境
\usepackage{geometry} % 调整页面边距
\usepackage{fancyhdr} % 页眉页脚定制
\hypersetup{
    colorlinks=true,            %链接颜色
    linkcolor=black,             %内部链接
    filecolor=magenta,          %本地文档
    urlcolor=cyan,              %网址链接
    pdftitle={Overleaf Example},
    pdfpagemode=FullScreen,
    }

% 页边距设置
\geometry{left=3cm, right=3cm, top=2.5cm, bottom=2.5cm}

% 页眉页脚设置
\pagestyle{fancy}
\fancyhf{}
\fancyhead[L]{\leftmark} % 左页眉显示章节标题
\fancyhead[R]{\thepage} % 右页眉显示页码

\newcommand{\norm}[1]{\| #1 \|}
\newcommand{\ip}[1]{\left\langle#1\right\rangle}

% 标题设置
\title{\textbf{数分三 HW 3}}
\author{罗淦 2200013522}
\date{\today}

% 行距设置
\linespread{1.2}

% 定义黑色边框的 problem 环境
\newenvironment{problem}{\begin{framed}\par\noindent\textbf{\textit{题目. }}}{\end{framed}\par}
\newenvironment{solution}{%
  \par\noindent\textbf{\textit{解答. }}\ignorespaces
}{%
  \hfill\ensuremath{\square}\par % 在结尾添加正方形
}
\newenvironment{note}{\par\noindent\textbf{\textit{题目的注记. }}\ignorespaces}{\par}


% 允许公式在页面之间自动换行
\allowdisplaybreaks

\begin{document}

\maketitle

% 添加目录

\section{HW 3}
\begin{problem}
    22. 求下列复合函数的偏导数, 其中 $f$ 是可微函数:
    
    (1) $z=f\left(x \mathrm{e}^y, x \mathrm{e}^{-y}\right)$;
    
    (2) $u=f\left(\sum_{i=1}^n x_i^2, \prod_{i=1}^n x_i^2, x_3, \cdots, x_n\right)$.
    \end{problem}
    
    \begin{solution}
    (1)
    \begin{align*}
        \frac{\partial z}{\partial x}&=f_1(xe^y,xe^{-y})e^y+f_2(xe^y,xe^{-y})e^{-y}\\
        \frac{\partial z}{\partial y}&=f_1(xe^y,xe^{-y})xe^y-f_2(xe^y,xe^{-y})xe^{-y}
    \end{align*}
    (2)
    \begin{align*}
        \frac{\partial u}{\partial x_i}&=f_1(\sum_{i=1}^n x_i^2, \prod_{i=1}^n x_i^2, x_3, \cdots, x_n)2x_i+f_2(\sum_{i=1}^n x_i^2, \prod_{i=1}^n x_i^2, x_3, \cdots, x_n)2x_i\Pi_{j\neq i}x_j^2\\&+\sum_{j=3}^{n}\delta_{ij}f_j(\sum_{i=1}^n x_i^2, \prod_{i=1}^n x_i^2, x_3, \cdots, x_n)
    \end{align*}
    \end{solution}
    
    \begin{problem}
        25. 若 $f(\boldsymbol{x})$ 是定义在区域 $D \subset \mathbb{R}^n(n \geqslant 2)$ 内的函数并且存在正整数 $K$, 使得 $f(t \boldsymbol{x})=t^K f(\boldsymbol{x})$ 对于 $\forall t>0, \forall \boldsymbol{x} \in D$ 成立, 则称 $f(\boldsymbol{x})$ 是 $K$次齐次函数. 设 $K$ 次齐次函数 $f(\boldsymbol{x})$ 在 $D$ 内具有各个 $k(1 \leqslant k \leqslant K)$阶连续偏导数, 证明:
    
        $$
        \left(\sum_{i=1}^n x_i \frac{\partial}{\partial x_i}\right)^k f(\boldsymbol{x})=K(K-1) \cdots(K-k+1) f(\boldsymbol{x})
        $$
    \end{problem}
    
    \begin{problem}
        28. 设函数 $x=r \cos \alpha-t \sin \alpha, y=r \sin \alpha+t \cos \alpha$, 其中 $\alpha \in \mathbb{R}$为常数. 证明:对任何可微函数 $f(x, y)$ ,成立
    
        $$
        \left(\frac{\partial f}{\partial x}\right)^2+\left(\frac{\partial f}{\partial y}\right)^2=\left(\frac{\partial f}{\partial r}\right)^2+\left(\frac{\partial f}{\partial t}\right)^2
        $$    
    \end{problem}
    \begin{solution}
        \begin{align*}
            \frac{\partial f}{\partial r}&=f_x\cos \alpha+f_y\sin \alpha\\
            \frac{\partial f}{\partial t}&=-f_x\sin \alpha+f_y\cos \alpha\\
            \Rightarrow& (\frac{\partial f}{\partial r})^2+(\frac{\partial f}{\partial t})^2=f_x^2+f_y^2
        \end{align*}
    \end{solution}
    
    \begin{problem}
        31. 求下列函数的二阶偏导数, 其中函数 $f$ 具有二阶连续导数:
    
        (1) $z=f\left(x^2+y^2, x y\right)$;
    
        (2) $z=f\left(x_1+x_2+\cdots+x_n\right)$.
    \end{problem}
    \begin{solution}
    (1)
    \begin{align*}
    \frac{\partial z}{\partial x}&=2xf_1+yf_2\\
    \frac{\partial z}{\partial y}&=2yf_1+xf_2\\
    \frac{\partial^2 z}{\partial x^2}&=2yf_1+xf_2\\
    \end{align*}
    \end{solution}
    
    \begin{problem}
    34. 设 $f(x)$ 是一个二次可微函数, 证明
        $F(x, t)=\frac{1}{2}[f(x-c t)+f(x+c t)] \quad$ (其中 $c$ 为常数)
        满足偏微分方程 $\frac{\partial^2 F}{\partial t^2}=c^2 \frac{\partial^2 F}{\partial x^2}$.
    \end{problem}
    \begin{solution}
        \begin{align*}
            \frac{\partial F}{\partial x}&=\frac{1}{2}[f^{\prime}(x-ct)+f^{\prime}(x+ct)]\\ 
            \frac{\partial F}{\partial t}&=\frac{c}{2}[-f^{\prime}(x-ct)+f^{\prime}(x+ct)]\\ 
            \frac{\partial^2 F}{\partial x^2}&=\frac{1}{2}[f^{\prime\prime}(x-ct)+f^{\prime\prime}(x+ct)]\\ 
            \frac{\partial^2 F}{\partial t^2}&=\frac{c^2}{2}[f^{\prime\prime}(x-ct)+f^{\prime\prime}(x+ct)]\\ 
            \Rightarrow& \frac{\partial^2 F}{\partial t^2}=c^2\frac{\partial^2 F}{\partial x^2}
        \end{align*}
    \end{solution}
    
    \begin{problem}
        37. 设 $x=2 r-s, y=r+2 s$, 求 $\frac{\partial^2 f(x, y)}{\partial r \partial s}$, 其中函数 $f(x, y)$ 具有二阶连续偏导数.
    \end{problem}
    \begin{solution}
        
    \end{solution}

\end{document}

