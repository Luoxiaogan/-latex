\documentclass[10pt, a4paper, oneside]{ctexart}
\usepackage{amsmath, amsthm, amssymb, bm, color, xcolor, framed, graphicx, hyperref, mathrsfs, etoolbox, wrapfig}
\usepackage[thicklines]{cancel}
\usepackage{enumitem} % 用于更灵活的列表环境
\usepackage{geometry} % 调整页面边距
\usepackage{fancyhdr} % 页眉页脚定制
\hypersetup{
    colorlinks=true,            %链接颜色
    linkcolor=black,             %内部链接
    filecolor=magenta,          %本地文档
    urlcolor=cyan,              %网址链接
    pdftitle={Overleaf Example},
    pdfpagemode=FullScreen,
    }

% 页边距设置
\geometry{left=3cm, right=3cm, top=2.5cm, bottom=2.5cm}

% 页眉页脚设置
\pagestyle{fancy}
\fancyhf{}
\fancyhead[L]{\leftmark} % 左页眉显示章节标题
\fancyhead[R]{\thepage} % 右页眉显示页码

\newcommand{\norm}[1]{\| #1 \|}
\newcommand{\ip}[1]{\left\langle#1\right\rangle}

% 标题设置
\title{\textbf{数分三 HW 2}}
\author{罗淦 2200013522}
\date{\today}

% 行距设置
\linespread{1.2}

% 定义黑色边框的 problem 环境
\newenvironment{problem}{\begin{framed}\par\noindent\textbf{\textit{题目. }}}{\end{framed}\par}
\newenvironment{solution}{%
  \par\noindent\textbf{\textit{解答. }}\ignorespaces
}{%
  \hfill\ensuremath{\square}\par % 在结尾添加正方形
}
\newenvironment{note}{\par\noindent\textbf{\textit{题目的注记. }}\ignorespaces}{\par}


% 允许公式在页面之间自动换行
\allowdisplaybreaks

\begin{document}

\maketitle

% 添加目录

\section{HW 2}


\begin{problem}
13. 求下列函数的定义域:

(1) $f(x, y, z)=\ln \left(y-x^2-z^2\right)$;

(2) $f(x, y, z)=\sqrt{x^2+y^2-z^2}$;

(3) $f(x, y, z)=\frac{\ln \left(x^2+y^2-z\right)}{\sqrt{z}}$.
\end{problem}
\begin{solution}
(1) $\{(x,y,z)| y\> x^2+z^2\}$\\
(2) $\{(x,y,z)| z^2\geq x^2+y^2\}$\\
(3) $\{(x,y,z)| x^2+y^2>z>0\}$
\end{solution}

\begin{problem}
14. 确定下列函数极限是否存在, 若存在则求出极限:

(1) $\lim _{E \ni(x, y) \rightarrow(0,0)} \frac{\sin \left(x^3+y^3\right)}{x^2+y}$, 其中 $E=\left\{(x, y): y>x^2\right\}$;

(2) $\lim _{(x, y) \rightarrow(0,0)} x \ln \left(x^2+y^2\right)$;

(3) $\lim _{|(x, y)| \rightarrow+\infty}\left(x^2+y^2\right) \mathrm{e}^{-(|x|+|y|)}$;

(4) $\lim _{|(x, y)| \rightarrow+\infty}\left(1+\frac{1}{|x|+|y|}\right)^{\frac{x^2}{|x|+|y|}}$;

(5) $\lim _{(x, y, z) \rightarrow(0,0,0)}\left(\frac{x y z}{x^2+y^2+z^2}\right)^{x+y}$;

(6) $\lim _{E \ni(x, y, z) \rightarrow(0,0,0)} x^{y z}$, 其中 $E=\{(x, y, z): x, y, z>0\}$;

(7) $\lim _{(x, y, z) \rightarrow(0,1,0)} \frac{\sin (x y z)}{x^2+z^2}$

(8) $\lim _{(x, y, z) \rightarrow(0,0,0)} \frac{\sin x y z}{\sqrt{x^2+y^2+z^2}}$

(9) $\lim _{\boldsymbol{x} \rightarrow \mathbf{0}} \frac{\left(\sum_{i=1}^n x_i\right)^2}{|\boldsymbol{x}|^2}$.
\end{problem}
\begin{solution}
(1) $\frac{\sin(x^3+y^3)}{x^2+y}=\frac{x^3+y^3+o(x^3+y^3)}{x^2+y}$. \\如果$\lim_{(x,y)\to (0,0)}\frac{x^3+y^3}{x^2+y}=0$, 那么当然有$\lim_{(x,y)\to (0,0)}\frac{o(x^3+y^3)}{x^2+y}=0$
首先考虑对分子配方, 使得最后留在分子的只有$x$.
\begin{align*}
    y^3=(x^2+y)y^2-x^2y^2=(x^2+y)(y^2+x^2y)-x^4y=(x^2+y)(y^2+x^2y+x^4)-x^6
\end{align*}
因此有
\begin{align*}
    |\frac{x^3+y^3}{x^2+y}|\leq |y^2+x^2y+x^4|+|\frac{x^3(1-x^3)}{x^2+y}|
\end{align*}
对$|x^2+y|\geq |x^2-|y||$, \textcolor{blue}{即使$y\to 0$, 我也不能取$|y|\leq \frac{x^2}{2}$, 因为这样就不是从各个方向来趋近于$(0,0)$了. 当然, 如果$x$是趋于一个非零的数, 我是可以这么做的.}\\
或许可以这样做: \textcolor{red}{如果$|y|>2x^2$, 那么$|x^2+y|\geq x^2$; 如果$|y|\leq \frac{x^2}{2}\leq 2x^2$, 那么$|x^2+y|\geq \frac{x^2}{2}$. 总之, $|x^2+y|\geq 2x^2$.}\\
因此有
\begin{align*}
    |\frac{x^3(1-x^3)}{x^2+y}|=\frac{|x^3(1-x^3)|}{|x^2+y|}\leq \frac{|x^3(1-x^3)|}{2x^2}=\frac{|x(1-x^3)|}{2}\to 0
\end{align*}
\textcolor{red}{这是在没有考虑题目给出的$y>x^2$的条件下做的, 如果有这个条件, 当然好做了}:
\begin{align*}
    |\frac{x^3(1-x^3)}{x^2+y}|\leq |\frac{x^3(1-x^3)}{2x^2}|=|x(1-x^3)|\to 0
\end{align*}
\end{solution}
\begin{note}
主要是因为分母是$x^2+y$, 非齐次导致不好操作. 否则可以极坐标换元\\
之所以对分子配方把分子上的$y$全部移除是为了后面对分母做完操作之后全部都是$x$就好办了.(之所以不去消去$x$是因为多出来的$xy$配方消不掉)\\
分类讨论来给出分母的下界这一点很有意思.
\end{note}
\begin{solution}
(2) \textcolor{blue}{看见$x^2+y^2$, 比较trivial地可以想到极坐标换元}. 
\begin{align*}
    x\ln(x^2+y^2)=2r\ln(r)\cdot \cos\theta\to 0
\end{align*}
(3) \textcolor{red}{考虑放缩之后\textbf{整体换元}, 这样就可以使用洛必达了(虽然换元之后就显然了)}
\begin{align*}
    \frac{x^2+y^2}{e^{|x|+|y|}}\leq \frac{(|x|+|y|)^2}{e^{|x|+|y|}} = \frac{t^2}{e^t}\to 0 , \quad t=|x|+|y|\to 0
\end{align*}
(4) 极限不存在, 首先取$x\equiv 0, y\to +\infty$的路径, 有极限为$1$(实际上恒等于$1$). 如果取$x=y\to +\infty$的路径, 那么
\begin{align*}
    (1+\frac{1}{2|x|})^{2|x|}=(1+\frac{1}{2|x|})^{2|x| \cdot \frac{1}{4}}\to e^{\frac{1}{4}}
\end{align*}
因此, 极限不存在\\
(5) 考虑点列$(\frac{1}{t},0,0), t\in \mathbb{N}^*$, 那么$\lim_{t\to\infty}o^{t}=0$; 点列$(\frac{1}{t},\frac{1}{t},\frac{1}{t})$\\那么$\lim_{t\to\infty}(\frac{3}{t})^{\frac{2}{t}}=\lim_{t\to\infty} e^{\frac{2}{t}\ln(\frac{3}{t})} \lim_{k\to 0^+}e^{2k\ln(3k)}=1$, 极限不存在.\\
(6) 点列$(0,\frac{1}{t},\frac{1}{t})$, 极限$\lim_{t\to\infty} 0^{\frac{1}{t^2}}=0$; 点列$(\frac{1}{t},\frac{1}{t},\frac{1}{t})$, 极限$\lim_{t\to\infty} (\frac{1}{t})^{\frac{1}{t^2}}=1$, 极限不存在\\
(7) 点列$(\frac{1}{t},\frac{t}{t+1},\frac{1}{t})$, 极限$\lim_{t\to\infty}\frac{\sin(\frac{1}{t(t+1)})}{\frac{2}{t^2}}=\frac{1}{2}$. 点列$(\frac{1}{t},\frac{t}{t+1},\frac{2}{t})$, 极限$\lim_{t\to\infty}\frac{\sin(\frac{2}{t(t+1)})}{\frac{5}{t^2}}=\frac{2}{5}$, 极限不存在.\\
(8) \textcolor{red}{极限存在, 注意和前几问的重大区别, 从渐进角度来看, 大概是$\frac{xyz}{\sqrt{x^2+y^2+z^2}}$, 分子的次数更大, 因此会趋于$0$!}\\
\textcolor{blue}{注意$|\sin t|\leq |t|$恒成立}
\begin{align*}
    |\frac{\sin(xyz)}{\sqrt{x^2+y^2+z^2}}|\leq |\frac{xyz}{\sqrt{x^2+y^2+z^2}}|
\end{align*}
考虑三维的球坐标换元$\begin{cases}
    x&=r\sin \theta\cos \phi\\
    y&=r \sin \theta \sin \phi\\
    z&= r\cos \phi
\end{cases}$, 那么
\begin{align*}
    |\frac{xyz}{\sqrt{x^2+y^2+z^2}}|=r^2\cdot|\sin\theta\cos \phi \sin \theta \sin \phi\cos \phi|\leq r^2 \to 0
\end{align*}
\textcolor{blue}{或者,使用基本不等式:
\begin{align*}
\frac{n}{\sum_{i=1}^n \frac{1}{x_i}}\leq \sqrt[n]{\Pi_{i=1}^n x_i}\leq \frac{\sum_{i=1}^n x_i}{n} \leq \sqrt[n]{\frac{\sum_{i=1}^n x_i}{n}}
\end{align*} }
那么有$\sqrt{x^2+y^2+z^2}\geq \sqrt{3}\cdot \sqrt[3]{xyz}$, 所以有
\begin{align*}
    |\frac{xyz}{\sqrt{x^2+y^2+z^2}}|\leq |\frac{xyz}{\sqrt{3}\cdot \sqrt[3]{xyz}}|=\frac{1}{\sqrt{3}}\cdot (xyz)^{\frac{2}{3}}\to 0
\end{align*}
(8) 点列$(\frac{1}{t},0,\cdots,0)$, 极限$\lim_{t\to \infty}\frac{\frac{1}{t^2}}{\frac{1}{t^2}}=1$\\点列$(\frac{1}{t},\frac{1}{t}, 0,\cdots,0)$, 极限$\lim_{t\to \infty}\frac{\frac{4}{t^2}}{\frac{2}{t^2}}=2$, 极限不存在.
\end{solution}


\begin{problem}
15. 试给出三元函数 $f(x, y, z)$ 累次极限 $\lim _{x \rightarrow x_0} \lim _{y \rightarrow y_0} \lim _{z \rightarrow z_0} f(x, y, z)$ 的定义, 并构造一个三元函数 $f(x, y, z)$, 使得它满足: $\lim _{(x, y, z) \rightarrow(0,0,0)} f(x, y, z)$存在, 但 $\lim _{x \rightarrow 0} \lim _{y \rightarrow 0} \lim _{z \rightarrow 0} f(x, y, z)$ 不存在.
\end{problem}
\begin{solution}
三元函数累次极限的定义: 设函数$w = f(x,y,z)$在$E\subseteq \mathbb{R}^3$上有定义\\
且邻域$U_0( (x_0,y_0,z_0), \delta  )\subseteq E$.\\
若在$U_0( (x_0,y_0,z_0), \delta )$内, 对每一个固定的$x\neq x_0, y\neq y_0$, 有$\lim_{z\to z_0}f(x,y,z)= \varphi(x,y)$存在\\
且(二元函数的累次极限已经定义了, 直接调用)$\lim_{x\to x_0}\lim_{y\to y_0}\varphi(x,y)=A$\\
则有$\lim_{x\to x_0}\lim_{y\to y_0}\lim_{z\to z_0}f(x,y,z)=A$.\\
\textcolor{blue}{构造:
$$f(x,y,z)=x+z+y\sin \frac{1}{z}$$
重极限$\lim_{(x,y,z)\to (0,0,0) }f(x,y,z)=0$存在, 但是累次极限不存在, 因为$z\to 0 $时就已经无穷了.}
\end{solution}

\begin{problem}
16. 设 $y=f(x)$ 在 $U_0\left(0, \delta_0\right) \subseteq \mathbb{R}$ 中有定义, 满足 $\lim _{x \rightarrow 0} f(x)=0$, 且对于 $\forall x \in U_0\left(0, \delta_0\right)$, 有 $f(x) \neq 0$. 记 $E=\{(x, y): x y \neq 0\}$, 证明:

(1) $\lim _{E \ni(x, y) \rightarrow(0,0)} \frac{f(x) f(y)}{f^2(x)+f^2(y)}$ 不存在;

(2) $\lim_{E \ni(x, y) \rightarrow(0,0)}  \frac{yf^2(x)}{f^4(x)+y^2}$不存在.
\end{problem}

\begin{solution}
\textcolor{blue}{核心的思路: 还是去取不同的子列$(x_k,y_k)$, 来使得极限趋于不同的值. 不过这里实际上需要控制的是$f(x_k),f(y_k)$, 所以更困难一点.}\\
(1) 首先取子列$(x_k , y_k)=(\frac{1}{k}, \frac{1}{k})$, 那么极限是$\frac{1}{2}$.\\
其次, 对任意的$k\in \mathbb{N}^{*}$, $\exists \delta_k > 0$, 使得$|x_k|<\delta_k$, 就有$|f(x_k)|<\frac{1}{k}$.\\
\textcolor{red}{接下来, 对于固定的$|f(x_k)|$, 存在$\delta^{\prime}_k>0$, 使得$|y_k|<\delta^{\prime}_k$, 就有$|f(y_k)|<\frac{|f(x_k)|}{k}<\frac{1}{k^2}$}\\
那么对于这样的子列$(x_k,y_k)$, 就有极限
\begin{align*}
|\frac{f(x_k)f(y_k)}{f(x_k)^2+f(y_k)^2}| = \frac{1}{\frac{|f(x_k)|}{|f(y_k)|}+\frac{|f(y_k)|}{|f(x_k)|}}
\end{align*}
又因为
\begin{align*}
    |\frac{f(x_k)^2+f(y_k)^2}{f(x_k)f(y_k)}|= \frac{|f(x_k)|}{|f(y_k)|}+\frac{|f(y_k)|}{|f(x_k)|}\geq \frac{|f(x_k)|}{|f(y_k)|} \geq k \to +\infty
\end{align*}
因此
\begin{align*}
    |\frac{f(x_k)f(y_k)}{f(x_k)^2+f(y_k)^2}|\to 0
\end{align*}
所以极限不存在

(2) 和第一问相同的套路, 甚至还要简单一些:\\
首先取$y=f(x)^2$, 极限是$\frac{1}{2}$.\\
对任意的$k\in \mathbb{N}^*$, 存在$\delta_k$, $|x_k|<\delta_k$, 有$|f(x_k)|<\frac{1}{k}$\\
对固定的$|f(x_k)|$, 存在$y_k$, 使得$|y_k|<\frac{|f(x_k)|^2}{k}$, 因此有:
\begin{align*}
    |\frac{y_k f(x_k)^2}{f(x_k)^4+y_k^2 }|=\frac{1}{ \frac{f(x_k)^2}{|y_k|}+\frac{|y_k|}{f(x_k)^2}}\leq \frac{1}{ \frac{f(x_k)^2}{|y_k|}}\leq \frac{1}{k}\to 0
\end{align*}
因此极限不存在
\end{solution}

\begin{problem}
    17. 构造二元函数$f(x,y)$, 使得对$k=1,2,\cdots,K$, 有$\lim_{x\to 0}f(x,x^k)=0$, 但是$\lim_{(x,y)\to(0,0)}f(x,y)$不存在.
\end{problem}
\begin{solution}
\textcolor{blue}{核心的思路: 重极限不存在, 但是方向导数存在的例子, 例如:$\frac{xy}{x^2+y^2} = \frac{\sin 2\theta}{2}$, 尝试构造类似这样的多项式的分式形式\\
并且注意到, $k=1,2,\cdots,K$, 分母的主导项应该是更低阶的无穷小.}\\
取
$$f(x,y)=\frac{x^{K+1}}{x^{K+1}+y} \Rightarrow \frac{x^{K+1}}{x^{K+1}+x^k} \sim \frac{x^{K+1}}{x^k} \to 0, \quad k=1,2\cdots,K$$
而$y=x^{K+1}$时, 极限为$\frac{1}{2}$, 重极限不存在
\end{solution}

\begin{problem}
18. 设函数 $f(x, y)$ 在 $\mathbb{R}^2$ 内除直线 $x=a$ 与 $y=b$ 外处处有定义,并且满足:

(a) $\lim _{y \rightarrow b} f(x, y)=g(x)$ 存在;

(b) $\lim _{x \rightarrow a} f(x, y)=h(y)$ 一致存在, 即对于 $\forall \varepsilon>0, \exists \delta>0$, 使得对于 $\forall(x, y) \in\{(x, y): 0<|x-a|<\delta\}$ ,有 $|f(x, y)-h(y)|<\varepsilon$.

证明: 存在 $c \in \mathbb{R}$, 使得有

(1) $\lim _{x \rightarrow a} \lim _{y \rightarrow b} f(x, y)=\lim _{x \rightarrow a} g(x)=c$;

(2) $\lim _{y \rightarrow b} \lim _{x \rightarrow a} f(x, y)=\lim _{y \rightarrow b} h(y)=c$;

(3) $\lim _{E \ni(x, y) \rightarrow(a, b)} f(x, y)=c$, 其中 $E=\mathbb{R}^2 \backslash\{(x, y): x=a$ 或 $y=b\}$.
\end{problem}
\begin{solution}
(1) \textcolor{blue}{思路: 证明极限存在, 考虑柯西收敛准则}\\
要证明: $\forall \epsilon>0, \exists \delta>0$, 使得$\forall x_1,x_2 \in U_0(a,\delta)$, 都有$|g(x_1)-g(x_2) |<\epsilon$.\\
因为$\lim_{y\to b}f(x,y)=g(x)$, 所以, $\forall \epsilon>0$, 存在$y_0\neq b$, 使得$|g(x_1)-f(x_1,y_0) |\leq \epsilon/4$, $|g(x_2)-f(x_2,y_0) |\leq \epsilon/4$. \textcolor{blue}{之所以是相同的$y_0$是为了使用一致存在的条件}\\
因为$\lim_{x\to a}f(x,y)=h(y)$一致存在, 对于上面的$\epsilon>0$, 存在$\delta>0$, $\forall x\in U_0(a,\delta)$, 都有$|f(x,y)-h(y) |<\epsilon/4$, 那么就取最初的$x_1,x_2\in U_0(a,\delta)$, 因此有$ |f(x_1,y_0)-h(y_0) |\leq \epsilon/4 $, $ |f(x_2,y_0)-h(y_0) |\leq \epsilon/4 $.\\
因此有, $\forall \epsilon>0, \exists \delta>0$, $\forall x_1,x_2\in U_0(a,\delta)$:
\begin{align*}
    &|g(x_1)-g(x_2) |\\\leq& |g(x_1)-f(x_1,y_0) |+|f(x_1,y_0)-h(y_0) |+|h(y_0)-f(x_2,y_0) |+|f(x_2,y_0)-g(x_2) |\leq \epsilon
\end{align*}
因此极限$\lim_{x\to a}g(x)$存在, 记为$c$.

(2) 要证明: $\forall \epsilon>0, \exists \delta_1>0$, 使得$\forall y_1,y_2 \in U_0(b,\delta_1)$, 都有$|h(y_1)-h(y_2) |<\epsilon $.\\
首先, 因为$\lim_{x\to a}f(x,y)=h(y)$一致存在, 因此对上述的$\epsilon>0$, $\exists \delta_0>0$, 使得$\forall x\in U_0(a,\delta_0)$, 都有$|f(x,y)-h(y) |<\epsilon/5$. \textcolor{blue}{注意, 因为一致性, 才可以对不同的$y_1,y_2$, 只要$x$和$a$够近, 就行}\\
又因为我们(1)证明了$\lim_{x\to a}g(x)=c$, 因为对取定的$\epsilon>0$, 存在$\delta_0^{\prime}$, $\forall x_1,x_2\in U_0(a,\delta_0^{\prime})$, 都有$|g(x_1)-g(x_2) |<\epsilon/5$.\\
现在取$\delta_2 = \min\{\delta_0, \delta_0^{\prime}\}$, 取$x_1,x_2\in U_0(a,\delta_2)$.\\
因此, $\forall \epsilon>0$, $\exists \delta_1>0$, $\forall y_1,y_2\in U_0(b,\delta_1)$. 以及取$x_1,x_2\in U_0(a,\delta_2)$
\begin{align*}
    &|h(y_1)-h(y_2) |\\
    \leq& |h(y_1)-f(x_1,y_1) |+|f(x_1,y_1)-g(x_1) |+|g(x_1)-g(x_2)| \\&+ |g(x_2)-f(x_2,y_2)| + |f(x_2,y_2)-h(y_2)|\\
    \leq &\epsilon
\end{align*} 
\textcolor{red}{樂, 这样只是证明了极限存在, 但是极限不一定等于$c$啊, 可以一步到位的:}\\
想证明: $\forall \epsilon>0$, $\exists \delta>0$, $\forall y\in U_0(b,\delta)$, $|h(y)-c|<\epsilon $\\
因为: $\lim_{x\to a}f(x,y)=h(y)$一致存在, 所以对$\forall y$, 只要$x\in U_0(a,\delta_1)$, 都有$|f(x,y)-h(y) |<\epsilon/3$, 这里取$x_0\in U_0(a,\delta_1)$, 那么$|f(x_0,y)-h(y) |<\epsilon/3$\\ 
因为$\lim_{y\to b}f(x,y)=g(x)$, 所以$\forall \epsilon>0$, $\exists \delta>0$, $\forall y\in U_0(b,\delta)$, $|f(x_0,y)-g(x_0)|<\epsilon/3 $, 这里的$x_0$是前面取定的$x_0$\\
可以取$x_0$充分接近$a$, 使得$|g(x_0)-c|<\epsilon/3$\\
因此有:
\begin{align*}
    |h(y)-c| \leq |h(y)-f(x_0,y)| + |f(x_0,y)-g(x_0)| +|g(x_0)-c|\leq \epsilon
\end{align*}

(3) 取$x$充分接近$a$, 那么$x\in U_0(a,\delta_0)$, 那么任意的$y$, 都有$|f(x,y)-h(y)|<\epsilon/2$.\\
$\lim_{y\to b}h(y)=c$, 那么$y\in U_0(b,\delta_1)$, 有$|h(y)-c |<\epsilon/2$.\\
结合在一起就是$|f(x,y)-c |<\epsilon$, $\forall (x,y)\in U_0((a,b), \delta^*), \delta^* = \min\{\delta, \delta_1\}$
\end{solution}

\begin{problem}
19. 设函数 $f(x)$ 在 $[0,1]$ 上连续, 函数 $g(y)$ 在 $[0,1]$ 上有唯一的第一类间断点 $y_0=\frac{1}{2}$, ($g(y)$ 在 $[0,1] \backslash\left\{\frac{1}{2}\right\}$ 上连续). 试求函数 $F(x, y)=$ $f(x) g(y)$ 在 $[0,1] \times[0,1]$ 上的全体间断点.
\end{problem}
\begin{solution}
全体间断点是: $\{(x,\frac{1}{2})| x\in [0,1], \textcolor{red}{f(x)\neq 0} \}$\\
只需要考虑$\{(x,\frac{1}{2})| x\in [0,1]\}$是否全部都是间断点.\\
如果$f(x_0)\neq 0$, 那么$(x_0,\frac{1}{2})$是间断点. 反证法, 假设$(x_0,\frac{1}{2})$是$f(x)g(y)$的连续点, 那么因为$f(x_0)\neq 0$, 那么$\frac{1}{2}$会是$\frac{f(x)g(x)}{f(x)}=g(x)$的连续点, 矛盾\\
如果$f(x_0)=0$, \textcolor{blue}{因为是第一类间断点, 所以$g(y)$在$\frac{1}{2}$附近有界}, 所以
$$ |f(x)g(y)|\leq M|f(x)|\to 0, (x,y)\to (x_0,\frac{1}{2})$$
因此是连续点
\end{solution}

\begin{problem}
22. 设 $U \subseteq \mathbb{R}^n$ 是一个非空开集, 证明: 向量函数 $\boldsymbol{f}: U \rightarrow \mathbb{R}^m$ 在 $U$ 内连续的充分必要条件是开集的原像是开集, 即对 $\mathbb{R}^m$ 中的任意开集 $E, \boldsymbol{f}^{-1}(E)$ 是 $\mathbb{R}^n$ 中的开集.
\end{problem}
\begin{solution}
\textcolor{red}{\textbf{方法一}}: 不妨假设在$\mathbb{R}^m$取的任意开集$E$属于$f(U)$, 那么$\forall y\in E, \exists x_0, f(x_0)=y, \exists \epsilon>0$, 使得$U(f(x_0),\epsilon)\subseteq E$\\
函数$\bm{f}$连续$\iff \forall \epsilon>0, \exists \delta>0, \forall x_0\in U$, 有$x\in U(x_0,\delta) \Rightarrow f(x)\in U(f(x_0),\epsilon)$\\
即\textcolor{blue}{$f(U(x_0,\delta))\subseteq U(f(x_0),\epsilon)\iff U(x_0,\delta)\subseteq f^{-1}(U(f(x_0),\epsilon))$}\\
开集$E$的原像$f^{-1}(E)$是开集$\iff \forall x_0\in f^{-1}(E), \exists \epsilon>0$, 使得$U(f(x_0),\epsilon)\subseteq E\Rightarrow$$\exists \delta>0, U(x_0,\delta) \subseteq f^{-1}(E)$\\
($\Leftarrow$) 已知开集的原像都是开集, 那么$U(f(x_0),\epsilon), \forall \epsilon>0$是开集, 那么原像$f^{-1}(U(f(x_0),\epsilon))$也是开始, 且$x_0\in f^{-1}(U(f(x_0),\epsilon))$, 所以$\exists \delta>0$, $U(x_0,\delta)\subseteq f^{-1}(U(f(x_0),\epsilon))$, 即$x\in U(x_0,\delta)\Rightarrow f(x)\in U(f(x_0),\epsilon)$. 得证.\\
($\Rightarrow$) 已知函数$\bm{f}$是连续函数. 任取开集$E\subseteq f(U)$, 对取定的$E$, 任取$x_0\in f^{-1}(E)$, 有$y=f(x_0)\in E$, 因为$E$开集, $\exists \epsilon_{y}>0$, $\forall 0<\epsilon\leq \epsilon_y$, 有$U(f(x_0),\epsilon)\subseteq E$. 因为$f$连续, 所以$\exists \delta>0$, $f(U(x_0,\delta))\subseteq U(f(x_0),\epsilon)\iff U(x_0,\delta)\subseteq f^{-1}(U(f(x_0),\epsilon))\subseteq f^{-1}(E)$, $f^{-1}(E)$是开集,得证.

\textcolor{red}{\textbf{方法二}}: 反证法:\\
($\Rightarrow$): 已知函数连续, 假设开集$E\subseteq f(U)$的原像$f^{-1}(E)$不是开集, 存在$x_0\in f^{-1}(E)$是孤立点. 但由于$f(x_0)$是开集$E$的内点, 因此$\exists \epsilon>0, U(f(x_0),\epsilon)\subseteq E$, 因为$f$连续, $\exists \delta>0$, $f(U(x_0,\delta))\subseteq U(f(x_0),\epsilon) \iff U(x_0,\delta)\subseteq f^{-1}(U(f(x_0),\epsilon))\subseteq f^{-1}(E)$, 因此矛盾.\\
($\Leftarrow$) 已知开集的原像是开集, 不妨假设$f$有间断点$x_0$, 即$\exists \epsilon_0>0$, $\forall k\in \mathbb{N}^*$, $\exists x_k$, 使得$|f(x_k)-f(x_0)|>\epsilon_0$, 但是开集$U(f(x_0),\epsilon_0)$的原像$f^{-1}(U(f(x_0),\epsilon_0))$是开集, 且$x_0\in f^{-1}(U(f(x_0),\epsilon_0))$, 而$\{x_k\}\not\subseteq f^{-1}(U(f(x_0),\epsilon_0))$, 这与开集的定义矛盾.
\end{solution}

\begin{problem}
    25. 设函数 $f(x, y)$ 在 $D=[0,1] \times[0,1]$ 上连续, 它的最大值为 $M$,最小值为 $m$. 证明:对于 $\forall c \in(m, M)$, 存在无限多个 $(\xi, \eta) \in D$, 使得
    $$
    f(\xi, \eta)=c .
    $$
\end{problem}
\begin{solution}
\textcolor{red}{因为是连续的, 那么任意不同的两点, 有无数条连通的道路, 且这些道路的交集只有端点}.\\
对这些连续道路使用介值定理, 可以得到无限多个$(\xi,\eta)$. 得证.
\end{solution}

\begin{problem}
    28. 证明: 函数 $f(x, y)=\sqrt{x y}$ 在闭区域 $D=\{(x, y): x \geqslant 0, y \geqslant 0\}$上不一致连续.
\end{problem}
\begin{solution}
$f(x,y)$不一致连续$\iff \exists \epsilon_0>0, \forall \delta>0,s.t., \exists (x_1,y_1),(x_2,y_2)\in D, \norm{(x_1,y_1)-(x_2,y_2)}<\delta, |f(x_1,y_1)-f(x_2,y_2)|>\epsilon_0$.
\textcolor{red}{考虑取两个子列, 这两个子列在$\mathbb{R}^2$上距离趋于零, 但函数值不是}:
\begin{align*}
    ||(k,\frac{1}{k} )-(k,0)||&=\frac{1}{k}\to 0\\
    | f(k,\frac{1}{k})-f(k,0) |&=1
\end{align*}
因此不一致连续
\end{solution}

\begin{problem}
31. 设$E\in \mathbb{R}^n$是开集, $D\subseteq E$称为$E$的一个分支, 若$D$是区域, 并且对任意区域$D^{\prime}\subseteq E$, 只要$D\cap D^{\prime}\neq \varnothing$, 总有$D^{\prime}\subseteq D$. 证明: $\mathbb{R}^n$任意开集都是可数个分支的并
\end{problem}
\begin{solution}
取$\mathbb{Q}^n\cap D$, 因为$\mathbb{Q}^n$在$\mathbb{R}^n$中稠密, 且属于任意开集$D$的任意一个分支都是开区域, 因此, 必然有至少一个有理点落入这个分支中, 因此可以用这个有理点来标记这个分支, 又因为分支之间是无交的, 因此任意开集是可数个分支的并
\end{solution}


\begin{problem}
1. 设函数 $u=f(\boldsymbol{x})$ 在 $U\left(\boldsymbol{x}_0, \delta_0\right) \subset \mathbb{R}^n\left(\delta_0>0\right)$ 内存在各个偏导数, 并且所有的偏导数在该邻域内有界, 证明 $f(\boldsymbol{x})$ 在 $\boldsymbol{x}_0$ 处连续; 举例说明存在函数 $u=g(\boldsymbol{x})$, 它在 $\boldsymbol{x}_0$ 的某个邻域内存在无界的各个偏导数, 但它在 $x_0$ 处连续.
\end{problem}
\begin{solution}
\textcolor{blue}{常见的思路: 拆添项, 构造成关于某个分量的拉格朗日中值定理}\\
记$x=(x_1,\cdots,x_n), x_0=(u_1,\cdots,u_n), \Delta x_i=x_i-u_i$, 那么
\begin{align*}
f(u_1+\Delta x_1, u_2,\cdots,u_n)&=f(u_1,\cdots,u_n)+\Delta x_1f_{1}(u_1+\theta_1 \Delta x_1,u_2,\cdots,u_n)\\
f(u_1+\Delta x_1, u_2+\Delta x_2,u_3,\cdots,u_n)&=f(u_1+\Delta x_1,u_2,\cdots,u_n)+\Delta x_2f_{2}(u_1+\Delta x_1,u_2+\theta_2 \Delta x_2,\cdots,u_n)\\
\cdots &\\
f(x_1,\cdots,x_{n-1},x_n)&=f(x_1,\cdots,x_{n-1},u_n)+\Delta x_n f_{n}(x_1,\cdots,x_{n-1},u_n+\theta_n \Delta x_n)
\end{align*}
因此, 根据导函数的有界性, 得到
\begin{align*}
    |f(\bm{x})-f(\bm{x}_0) |\leq M(|\Delta x_1|+\cdots+|\Delta x_n|)\to 0
\end{align*}
构造反例, \textcolor{red}{想到在一元函数中, $t\sin (t)$补充在零点取$0$的定义后, 满足在$t=0$连续, 但导函数无界, 因此考虑$t=x^2+y^2$的情况}, 构造:
\begin{align*}
    g(x,y)=\begin{cases}
        (x^2+y^2)\sin(\frac{1}{x^2+y^2}), & (x,y)\neq (0,0)\\
        0, & (x,y)=(0,0)
    \end{cases}
\end{align*}
那么计算偏导数, 有
\begin{align*}
\frac{\partial f}{\partial x}&=2x\sin(\frac{1}{x^2+y^2})-2\frac{x}{x^2+y^2}\cos(\frac{1}{x^2+y^2})\\
\frac{\partial f}{\partial y}&=2y\sin(\frac{1}{x^2+y^2})-2\frac{y}{x^2+y^2}\cos(\frac{1}{x^2+y^2})\\
\end{align*}
上面的两个偏导数在$(x,y)\to(0,0)$的时候无界. 第一项能收敛到$0$, 但第二项无法控制:
\begin{align*}
    \frac{x}{x^2+y^2}\cos(\frac{1}{x^2+y^2})=\frac{\cos \theta}{r}\cos(\frac{1}{r^2})\to \infty
\end{align*}
\end{solution}

\begin{problem}
4. 求下列函数的各个偏导数:

(1) $z=\frac{x}{2 x^2+y^3+x y}$;

(2) $z=x \sqrt{x^2-y^2}$;

(3) $z=\tan \left(x^2+2 y^3\right)$;

(4) $u=(x+y+z) \mathrm{e}^{x y z}$;

(5) $u=\sin \left(y e^{x z}\right)$;

(6) $u=\ln \left(x y+x^4+z^2\right)$

(7) $u=\sqrt[3]{1-z \sin ^2(x+y)}$

(8) $u=\frac{\sin x z}{\cos x^2+y}$

(9) $u=\ln (\sec \sqrt{x+y-z})$;

(10) $u=\mathrm{e}^{-x z} \tan y$

(11) $u=\mathrm{e}^z\left(x^2+y^2+z^2\right)$;

(12) $u=\left(\frac{x}{y}\right)^z$;

(13) $u=\ln \left(1+\sqrt{\sum_{i=1}^n x_i^2}\right)$;

(14) $u=x_1 x_2 \cdots x_n+\left(x_1+x_2+\cdots+x_n\right)^n$.
\end{problem}
\begin{solution}
(1)
$$\frac{\partial z}{\partial x}=\frac{1}{2x^2+y^3+xy}-\frac{x(4x+y)}{(2x^2+y^3+xy)^2}, \;\frac{\partial z}{\partial y}=-\frac{x(3y^2+x)}{(2x^2+y^3+xy)^2}$$
(4)
\begin{align*}
\frac{\partial u}{\partial x}&=e^{xyz}+(x+y+z)yze^{xyz}\\
\frac{\partial u}{\partial y}&=e^{xyz}+(x+y+z)xze^{xyz}\\
\frac{\partial u}{\partial z}&=e^{xyz}+(x+y+z)xye^{xyz}\\
\end{align*}
(7)
\begin{align*}
    \frac{\partial u}{\partial x}&=\frac{1}{3}(1-z\sin^2 (x+y))^{-2/3}\cdot(-2z)\sin(x+y)\cos(x+y)\\
    \frac{\partial u}{\partial y}&=\frac{1}{3}(1-z\sin^2 (x+y))^{-2/3}\cdot(-2z)\sin(x+y)\cos(x+y)\\
    \frac{\partial u}{\partial z}&=\frac{1}{3}(1-z\sin^2 (x+y))^{-2/3}\cdot(-1)\sin^2(x+y)
\end{align*}
(10)
\begin{align*}
    \frac{\partial u}{\partial x}&=-ze^{-xz}\tan y\\
    \frac{\partial u}{\partial y}&=e^{-xz}\frac{1}{\cos^2 y}\\
    \frac{\partial u}{\partial z}&=-xe^{-xz}\tan y\\
\end{align*}
(13)
\begin{align*}
    \frac{\partial u}{\partial x_i}=\frac{x_i}{(\sum_{i=1}^n x_i^2)+(\sum_{i=1}^n x_i^2)^{\frac{1}{2}}}
\end{align*}
\end{solution}

\begin{problem}
7. 设函数 $f(x, y, z)=x^2-x y+y^2+z^2$, 求它在 $(1,1,1)$ 处的沿各个方向的方向导数, 并求出方向导数的最大值、最小值以及方向导数为零的所有方向.
\end{problem}
\begin{solution}
    \textcolor{red}{对于梯度存在的函数, 求方向导数, 可以先求梯度, 然后方向导数就是梯度在这个方向的单位向量上的投影. 但如果是不可微的函数, 那么就需要按照定义来计算了}. 注意: 这个单位向量一般用方向余弦表示, 在$\mathbb{R}^2$可以简化为$(\cos \theta,\sin \theta)$\\
    计算在$(1,1,1)$处梯度$\nabla f(1,1,1)=(1,1,2)$
    \\考虑方向单位向量: $u=(u_1,u_2,u_3)=(\cos \theta_1,\cos \theta_2,\cos \theta_3)$, 得到$\frac{\partial f}{\partial u}=u_1+u_2+2u_3$.\\
    方向导数的最大值和最小值: 梯度方向的同方向和反方向, 即$\max \frac{\partial f}{\partial u}=\sqrt{6},\min \frac{\partial f}{\partial u}=-\sqrt{6}$\\
    方向导数为$0$: 满足$\begin{cases}
        u_1+u_2+2u_3&=0\\
        u_1^2+u_2^2+u_3^2&=1
    \end{cases}$, 因此是圆和平面相交的圆环上的向量, 也即所有平行于平面$u_1+u_2+2u_3=0$的向量方向
\end{solution}

\begin{problem}
10. 设定义在 $\mathbb{R}^n$ 上的函数由下式给出:
$$
f(x)= \begin{cases}|x|^2 \sin \frac{1}{|x|^2}, & |x| \neq 0 \\ 0, & |x|=0\end{cases}
$$
证明: $\frac{{\partial} f(\boldsymbol{x})}{\partial x_i}(i=1,2, \cdots, n)$ 在 $\boldsymbol{x}=\mathbf{0}$ 处不连续, 但 $f(\boldsymbol{x})$ 在 $\mathbb{R}^n$ 上处处可微.
\end{problem}
\begin{solution}
$\norm{x}^2=\sum_{i=1}^n x_i^2$\\
首先计算在$\bm{0}$处的偏导数:
$$\lim_{x_i\to 0}\frac{x_i^2 \sin \frac{1}{x_i^2}}{x_i}=\lim_{x_i\to 0}x_i\sin \frac{1}{x_i^2}=0$$
然后计算$\bm{x}\neq\bm{0}$的偏导数:
\begin{align*}
    \frac{\partial f}{\partial x_i}=2x_i\sin \frac{1}{\sum_{i=1}^n x_i^2}-\frac{2x_i}{\sum_{i=1}^n x_i^2}\cos\frac{1}{\sum_{i=1}^n x_i^2}
\end{align*}
在$\bm{x}\to\bm{0}$时, 第一项趋于$0$, 但第二项无法控制, 例如在$x_1=x_2=\cdots=x_n$时, 有
\begin{align*}
    |\frac{2x_i}{\sum_{i=1}^n x_i^2}\cos\frac{1}{\sum_{i=1}^n x_i^2}|=|\frac{2}{nx_i}\cos\frac{1}{nx_i^2}|\to\infty
\end{align*}
因此偏导数在$\bm{0}$不连续\\
讨论$f(\bm{x})$的可微性, 因为偏导数在$\mathbb{R}^n\backslash\{0\}$连续, 因此函数在非零点可微, 只需要说明在$\bm{0}$处可微\\
根据可微性的定义, 要证明存在向量$\bm{h}$使得:
\begin{align*}
    \lim_{\norm{\Delta\bm{ x}}\to 0}\frac{f(\Delta \bm{x})-f(0)-\bm{h}^T\Delta \bm{x}}{\norm{\Delta\bm{ x}}}=\lim_{\norm{\Delta\bm{ x}}\to 0}\frac{f(\Delta \bm{x})-\bm{h}^T\Delta \bm{x}}{\norm{\Delta\bm{ x}}}=0
\end{align*}
因为
\begin{align*}
    \lim_{\norm{\Delta\bm{ x}}\to 0}\frac{f(\Delta \bm{x})-\bm{h}^T\Delta \bm{x}}{\norm{\Delta\bm{ x}}}&=\lim_{\norm{\Delta\bm{ x}}\to 0}\frac{f(\Delta \bm{x})}{\norm{\Delta\bm{ x}}}-\lim_{\norm{\Delta\bm{ x}}\to 0}\frac{\bm{h}^T\Delta \bm{x}}{\norm{\Delta\bm{ x}}}\\
    &=\lim_{\norm{\Delta\bm{ x}}\to 0}\norm{\Delta\bm{ x}}\sin\frac{1}{\Delta\bm{ x}}- \lim_{\norm{\Delta\bm{ x}}\to 0}\bm{h}^T\frac{\Delta\bm{ x}}{\norm{\Delta\bm{ x}}}=0
\end{align*}
因此可微(\textcolor{red}{实际上只需要验证$\frac{f(x_0+\delta x)-f(x_0)}{\norm{\delta x}}\to 0$即可})
\end{solution}

\begin{problem}
13. 设函数 $f(x, y)=x^2 y-3 y$, 求 $f(x, y)$ 的微分, 并求 $f(5.12,6.85)$的近似值
\end{problem}
\begin{solution}
考虑:
\begin{align*}
    f(5.12,6,85)=f(5,7)+f_x(5,7)*0.12+f_y(5,7)*(-0.15)=159.1
\end{align*}
精确值:
\begin{align*}
    159.01
\end{align*}
很精确!
\end{solution}

\begin{problem}
16. 求下列函数的梯度:

(1) $f(x, y, z)=x^2 \sin y z+y^2 \mathrm{e}^{x z}+z^2$;

(2) $f(\boldsymbol{x})=|\boldsymbol{x}| \mathrm{e}^{-|\boldsymbol{x}|}, \boldsymbol{x} \in \mathbb{R}^n \backslash\{\boldsymbol{0}\}(n \geqslant 2)$.
\end{problem}
\begin{solution}
(1) 
\begin{align*}
    \nabla f = (2x\sin yz+y^2ze^{xz},x^2z\cos yz+2ye^{xz},x^2u\cos yz+xy^2e^{xz}+2z)
\end{align*}
(2)
\begin{align*}
    &\frac{\partial f}{\partial x_i}=\frac{x_i}{\sqrt{\sum_{i=1}^n x_i^2}}e^{-\sqrt{\sum_{i=1}^n x_i^2}}-x_ie^{-\sqrt{\sum_{i=1}^n x_i^2}}\\
    \Rightarrow&\nabla f=(\frac{\partial f}{\partial x_1},\cdots,\frac{\partial f}{\partial x_n})
\end{align*}
\end{solution}

\begin{problem}
19. 求函数 $f(x, y, z)=2 x^3 y-3 y^2 z$ 在 $(1,2,-1)$ 处所有的方向导数构成的集合
\end{problem}

\begin{solution}
因为函数处处可微, 可以直接计算梯度:
$$\nabla f(1,2,-1)=(12,14,-12)$$
方向余弦$v=(\cos \theta_1,\cos \theta_2,\cos \theta_3)$和$\nabla f$的内积就是所有方向导数构成的集合:
$$\{v^T\nabla f=12\cos \theta_1+14\cos \theta_2-12\cos \theta_3|\cos^2 \theta_1+\cos^2\theta_2+\cos^2 \theta_3=1\}$$
\end{solution}

\end{document}

