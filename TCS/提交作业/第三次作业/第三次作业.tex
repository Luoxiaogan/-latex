\documentclass[10pt, a4paper, oneside]{ctexart}
\usepackage{amsmath, amsthm, amssymb, bm, color, xcolor, framed, graphicx, hyperref, mathrsfs, etoolbox, wrapfig, xurl}
\usepackage[thicklines]{cancel}
\usepackage{enumitem} % 用于更灵活的列表环境
\usepackage{geometry} % 调整页面边距
\usepackage{fancyhdr} % 页眉页脚定制
\hypersetup{
    colorlinks=true,            %链接颜色
    linkcolor=black,             %内部链接
    filecolor=magenta,          %本地文档
    urlcolor=cyan,              %网址链接
    pdftitle={Overleaf Example},
    pdfpagemode=FullScreen,
    }

% 页边距设置
\geometry{left=3cm, right=3cm, top=2.5cm, bottom=2.5cm}

% 页眉页脚设置
\pagestyle{fancy}
\fancyhf{}
\fancyhead[L]{\leftmark} % 左页眉显示章节标题
\fancyhead[R]{\thepage} % 右页眉显示页码

% 标题设置
\title{\textbf{TCS HW3}}
\author{罗淦 2200013522}
\date{\today}

% 行距设置
\linespread{1.2}

% 定义黑色边框的 problem 环境
\newenvironment{problem}{\begin{framed}\par\noindent\textbf{\textit{题目. }}}{\end{framed}\par}
\newenvironment{solution}{%
  \par\noindent\textbf{\textit{解答. }}\ignorespaces
}{%
  \hfill\ensuremath{\square}\par % 在结尾添加正方形
}
\newenvironment{note}{\par\noindent\textbf{\textit{题目的注记. }}\ignorespaces}{\par}


% 允许公式在页面之间自动换行
\allowdisplaybreaks

\begin{document}

\maketitle

% 添加目录
%\tableofcontents
%\newpage

\begin{problem}
    计算$\text{STP}(x_1,\cdots,x_n,y,0)=$?
    \end{problem}
    \begin{solution}
        $$\text{STP}(x_1,\cdots,x_n,y,0)=\begin{cases}
            1, & \text{if}\quad y=0\\
            0, &\text{否则} 
        \end{cases}$$
    \end{solution}
    
    \begin{problem}
        3.1 证明HALT$(0,x)$是不可计算的
    \end{problem}
    \begin{solution}
    反证法, 假设HALT$(0,x)$可计算, 构造:
    $$[A] \text{IF} \quad  \text{HALT}(0,x) \quad \text{GOTO} \quad A $$
    这是可计算的, 有程序$P$可以计算, 程序$P$的编号是$k$.\\
    那么HALT$(0,k) \iff P(k)=0 \iff \neg $HALT$(0,k)$, 矛盾.
    \end{solution}
    
    \begin{problem}
        3.3 证明: 不存在可计算函数$f(x)$, 使得当$\Phi(x,x)\downarrow$时$f(x) = \Phi(x,x)+1$.
    \end{problem}
    \begin{solution}
        反证法, 如果是可计算的, 那么函数$f(x)$有编号$k$, 那么\textcolor{red}{$f(x)=\Phi(x,k)$}, 因此有$f(k)=\Phi(k,k)=\Phi(k,k)+1$, 矛盾.
    \end{solution}
    
    \begin{problem}
    设$f$是一个全函数, $B=\{f(n)| n\in N\}$, 证明:\\
    (1) 若$f$是可计算的, 那么$B$是$r.e.$\\
    (2) 若$f$是可计算的, 严格增加的($f(n)<f(n+1), \forall n\in N$), 则$B$是递归的.
    \end{problem}
    \begin{solution}
        (1) \textcolor{blue}{构造一个可计算函数, 使得输入的$x$如果在$B$里面, 就停机(输出什么都行), 否则不停机(无定义)}
        \begin{align*}
            h(x) = \begin{cases}
                0, &\text{if} \quad \exists n (f(n)=x)\\
                \uparrow, &\text{else}
            \end{cases}
        \end{align*}
        可计算函数:
        \begin{align*}
            [A] &\text{IF} \quad f(N)=X\quad \text{GOTO} \quad E\\
            & N\leftarrow N+1\\
            & \text{GOTO} \quad A
        \end{align*}
        (2)\textcolor{blue}{构造一个可计算函数, 使得输入的$x$如果在$B$里面, 就输出$1$, 如果不在里面, 就输出$0$}\\
        \textcolor{red}{又因为函数是严格单调的, 因此我只需要遍历$N$, 直到$f(N)$大于$X$或者$f(N)=X$即可.}
        \begin{align*}
            [A] &\text{IF} \quad f(N)=X\quad \text{GOTO} \quad C\\
            [B] &\text{IF} \quad f(N)>X\quad \text{GOTO} \quad E\\
            & N\leftarrow N+1\\
            & \text{GOTO} \quad A\\
            [C]& Y\leftarrow Y+1\\
            & \text{GOTO} \quad E
        \end{align*}
    \end{solution}
    \begin{note}
        \begin{itemize}
            \item 存在不是 可计算 的 全函数, 例子(这是一个不可计算的谓词): HALT(x,x)
            \item 存在不是 部分可计算 的 部分函数, 例子: $f(x)= \begin{cases}
                1, & \text{if } \neg \text{HALT}(x,x)\\
                0, & \text{if } \text{HALT}(x,x)
            \end{cases}$
        \end{itemize}
    \end{note}
    
    \begin{problem}
    3.6 设$A,B$是$N$的非空子集, 定义:
    \begin{align*}
        A\odot B & = \{2x| x\in A\}\cup \{2x+1 | x\in B\}\\
        A \otimes B & = \{ <x,y>| x\in A,y\in B  \}
    \end{align*}
    证明:\\
    (1) $A\odot B$是递归的当且仅当$A$和$B$是递归的\\
    (2) $A\otimes B$是递归的当且仅当$A$和$B$是递归的
    \end{problem}
    \begin{note}
    实际上, 上述两个集合的构造方式都是"双射"的编码, 因此只要知道$x$\textcolor{red}{是否属于}$A\odot B$, 那么就一定知道$x$\textcolor{red}{是否属于}$A$或者$B$, 反之亦然(显然)
    \end{note}
    \begin{solution}
    (1) 分析谓词: $(x\in A\odot B)$, $(x\in A)$, $(x\in B)$:
    \begin{align*}
        (x\in A\odot B) \iff ( (2|x) \wedge (\lfloor x/2 \rfloor\in  A ))
        \vee ( \neg(2|x) \wedge  ( \lfloor (x-1)/2 \rfloor  ) \in B)
    \end{align*}
    (2) 对于配对函数$z=<x,y>$, 有原始递归函数$l(z),r(z)$.\\
    分析谓词: $(x\in A\otimes B)$, $(x\in A)$, $(x\in B)$:
    \begin{align*}
        (z\in A\otimes B) \iff ( (x = l(z)) \wedge (x \in  A ))
        \vee ( (x = r(z)) \wedge  ( x \in B))
    \end{align*}
    \end{solution}
    
    \begin{problem}
    3.7 证明集合$B = \{x\in N |   a\in \text{ran}\Phi_x \}$是递归可枚举的, $a$是常数.
    \end{problem}
    \begin{solution}
    \textcolor{blue}{这里的$\text{ran}\Phi_x$就是值域\\注意, 这里的函数$\Phi_x$是部分可计算的, 而不是像3.5的全函数\\我需要构造一个可计算函数来判断$x$}\textcolor{red}{是属于}\textcolor{blue}{$B$的, 即, 如果属于, 我知道, 但是不知道不属于}\\
    需要再添加一个程序来执行对$\Phi(n_0,x)$是否有定义的判定.
    \begin{align*}
        [A] &\text{IF} \quad \neg STP(N,X,T)  \quad \text{GOTO} \quad B\\
        & \text{IF} \quad \Phi(N,X)=a  \quad \text{GOTO} \quad E\quad \quad \textcolor{blue}{\text{需要对输入N停机之后才能有输出来判断是不是a}}\\
        [B]& N\leftarrow N+1\\
        & \text{IF} \quad N\leq T \quad \text{GOTO} \quad A\\
        &T\leftarrow T+1 \\
        &N\leftarrow \\
        &\text{GOTO} \quad A
    \end{align*}
    \end{solution}
\end{document}