\documentclass[10pt, a4paper, oneside]{ctexart}
\usepackage{amsmath, amsthm, amssymb, bm, color, xcolor, framed, graphicx, hyperref, mathrsfs, etoolbox, wrapfig, fdsymbol}
\usepackage[thicklines]{cancel}
\usepackage{enumitem}
\usepackage{geometry}
\usepackage{fancyhdr}

\hypersetup{
    colorlinks=true,
    linkcolor=black,
    filecolor=magenta,
    urlcolor=cyan,
    pdftitle={Overleaf Example},
    pdfpagemode=FullScreen,
}

\geometry{left=3cm, right=3cm, top=2.5cm, bottom=2.5cm}
\pagestyle{fancy}
\fancyhf{}
\fancyhead[L]{\leftmark}
\fancyhead[R]{\thepage}

\title{\textbf{TCS HW2}}
\author{罗淦 2200013522}
\date{\today}

\linespread{1.2}

\newenvironment{problem}{\begin{framed}\par\noindent\textbf{\textit{题目. }}}{\end{framed}\par}
\newenvironment{solution}{%
  \par\noindent\textbf{\textit{解答. }}\ignorespaces
}{%
  \hfill\ensuremath{\square}\par
}
\newenvironment{note}{\par\noindent\textbf{\textit{题目的注记. }}\ignorespaces}{\par}

\allowdisplaybreaks

\begin{document}
\maketitle

\begin{problem}
$P_{12}$ 1.3 证明下面的函数是部分可计算的:\\
(1) $x_1+x_2$;  (2) $x_1-x_2$;  (3) $x_1x_2$;  (4) 空函数
\end{problem}
\begin{solution}
\textcolor{red}{要证明一个函数是部分可计算的, 实际上就是可以用$S$函数把它写出来, "部分"指的是可以在某些点上没有定义}\\
(1) 思路:$x_2$一直减$1$, $x_1$一直加$1$, 直到减到零
\begin{align*}
    &Y\leftarrow X_1\\
    &Z\leftarrow X_2\\
    [A]& \text{IF} \quad Z\neq 0 \quad \text{GOTO} \quad B\\
    & \text{GOTO}\quad E\\
    [B]& Z\leftarrow Z-1\\
    &Y\leftarrow Y+1\\
    &\text{GOTO}\quad  A
\end{align*} 
(2) 思路: 如果$x_1<x_2$, 那么无定义, 因此看谁先减到$0$.
\begin{align*}
    &Z_1\leftarrow X_1\\
    &Z_2\leftarrow X_2\\
    [A]& \text{IF} \quad Z_1\neq 0 \quad \text{GOTO} \quad B\\
    &\text{GOTO}\quad  D_1\\
    [B]& \text{IF} \quad Z_2\neq 0 \quad \text{GOTO} \quad C\\
    &\text{GOTO}\quad  D_2\\
    [C]& Z_1\leftarrow Z_1-1\\
    & Z_2\leftarrow Z_2-1\\
    &\text{GOTO}\quad  A\\
    [D_1]& Y\leftarrow Z_2\\
    &\text{GOTO}\quad  E\\
    [D_2]& Y\leftarrow Z_1\\
    &\text{GOTO}\quad  E
\end{align*}
(3) 思路: 如果有$0$, 返回$0$; 如果都不是$0$, 一直减$x_2$, 同时对$x_1$做加法(已经证明是部分可计算函数)
\begin{align*}
    &Z_1\leftarrow X_1\\
    &Z_2\leftarrow X_2\\
    [A]& \text{IF} \quad Z_1\neq 0 \quad \text{GOTO} \quad B\\
    &\text{GOTO}\quad  E\\
    [B]& \text{IF} \quad Z_2\neq 0 \quad \text{GOTO} \quad C\\
    &\text{GOTO}\quad  E\\
    [C]& Z_2\leftarrow Z_2-1\\
    &Z_1\leftarrow Z_1+Z_1\\
    &\text{GOTO}\quad  B
\end{align*}
(4) 思路: 空函数就是处处无定义, 直接进入死循环即可.
\begin{align*}
    [A]&Z\leftarrow Z+1\\
    & \text{IF} \quad Z\neq 0 \quad \text{GOTO} \quad A\\
\end{align*}
\end{solution}

\begin{problem}
    $P_{13}$ 1.4 证明下述谓词是可计算的\\
(1) $x\geq a$, $a$是正整数\\
(2) $x_1\leq x_2$\\
(3) $x_1=x_2$
\end{problem}
\begin{solution}
\textcolor{red}{要证明一个谓词是可计算的,实际上就是证明这个谓词(判断过程)可以用$S$语言表示}\\
(1) 思路: 两边一直减$1$, 看谁先减到$0$, \textcolor{blue}{又因为是大于等于, 所以可以先验证$Z_2$}
\begin{align*}
    &Z_1\leftarrow X\\
    &Z_2\leftarrow a\\
    [A]& \text{IF} \quad Z_2\neq 0 \quad \text{GOTO} \quad B\\
    &\text{GOTO}\quad  D_2\\
    [B]& \text{IF} \quad Z_1\neq 0 \quad \text{GOTO} \quad C\\
    &\text{GOTO}\quad  D_1\\
    [C]&Z_1\leftarrow Z_1-1\\
    &Z_2\leftarrow Z_2-1\\
    &\text{GOTO}\quad  A\\
    [D_1]&\text{GOTO}\quad  E\\
    [D_2]&Y\leftarrow Y+1\\
    &\text{GOTO}\quad  E
\end{align*}
(2) 思路: 和第一问思路类似, 注意取等条件对判定顺序的影响
\begin{align*}
    &Z_1\leftarrow X_1\\
    &Z_2\leftarrow X_2\\
    [A]& \text{IF} \quad Z_1\neq 0 \quad \text{GOTO} \quad B\\
    &\text{GOTO}\quad  D_2\\
    [B]& \text{IF} \quad Z_2\neq 0 \quad \text{GOTO} \quad C\\
    &\text{GOTO}\quad  D_1\\
    [C]&Z_1\leftarrow Z_1-1\\
    &Z_2\leftarrow Z_2-1\\
    &\text{GOTO}\quad  A\\
    [D_1]&\text{GOTO}\quad  E\\
    [D_2]&Y\leftarrow Y+1\\
    &\text{GOTO}\quad  E
\end{align*}
(3) 思路: 可以使用(1)和(2)的判定了
\begin{align*}
    &Z_1\leftarrow X_1\\
    &Z_2\leftarrow X_2\\
    [A]&\text{IF} \quad Z_1\leq Z_2 \quad \text{GOTO} \quad B\\
    &\text{GOTO}\quad  E\\
    [B]&\text{IF} \quad Z_2\leq Z_1 \quad \text{GOTO} \quad C\\
    &\text{GOTO}\quad  E\\
    [C]&Y\leftarrow Y+1\\
    &\text{GOTO}\quad  E
\end{align*}
\end{solution}

\begin{problem}
$P_{16}\; 2.1.5$用基本的原始递归函数来表示下面的函数, 从而它们也是原始递归函数\\
(1) $E(x)=\begin{cases}
    1, &x\text{偶数}\\
    0, &x\text{奇数}
\end{cases}$\\
(3) $\max(x,y)=\begin{cases}
    x&,x\geq y\\
    y,&x<y
\end{cases}$
\end{problem}
\begin{solution}
(1) $E(0)=1, E(x+1)=\alpha(E(x))$.\\
(2) $\max(x,y)=x\alpha(y\dotminus x)+y\alpha(x \dotminus p(y))$\\
这是因为, 当$x\geq y$时, $y\dotminus x=0$, 因此$\alpha(y\dotminus x)=1$. 但为了防止$x=y$且取非零值($x=y=0$不影响)的时候, 得到$2x$, 因此要保证$x=y$的时候, $y$的系数不能是$1$, 因此取$x\dotminus p(y)$ 
\end{solution}

\begin{problem}
$P_{22}\;2.3.4$利用极小化给出下述函数$f(x)$:\\
(1) $f(x)=\begin{cases}
    \sqrt{x}, &x\text{是完全平方数}\\
    \uparrow, &\text{否则}
\end{cases}$\\
(2) $g(x)$是全函数, 若存在$t$, 使得$g(t)=x$, 则$f(x)$等于使得$g(t)=x$成立的最小的$t$, 否则$f(x)\uparrow$.
\end{problem}
\begin{solution}
(1) $f(x)=\min_{t}(t^2=x)$\\
(2) $f(x)=\min_{t}(g(t)=x)$
\end{solution}
\begin{note}
\textbf{\textcolor{blue}{应该是对的吧}}: \textcolor{red}{使用极小化定义的函数中,加入我想要定义的函数不是全函数(即在某些情况下无定义), 那么一定是用无界极小化定义的. 因为有界极小化+原始递归,定义出的都是全函数}
\end{note}

\begin{problem}
$P_{36}\; 2.1$ 证明: 仅在有穷个点处非零值, 在其余点取零的函数一定是原始递归函数.
\end{problem}
\begin{solution}
 思路:因为只在有限个地方取非零值, 那么只需要对这有限个点做有限递归\\
设$f(x)$在$x_1,\cdots,x_k$处取非零值$c_1,\cdots,c_k$, 其余点取零.\\
定义:
$$\chi_{x_i}(x)=\begin{cases}
    1,&x=x_i\\
    0,&x\neq x_i
\end{cases}=(x=x_i)$$
是原始递归的谓词, 那么定义:
\begin{align*}
    f(x)=\sum_{i=1}^{k}c_i\cdot\chi_{x_i}(x)
\end{align*}
(有限求和,常数乘法原始递归)因此也是原始递归的.
\end{solution}
\begin{note}
    \textcolor{blue}{也可以写成:\begin{align*}f(x)=\sum_{i=1}^{k}c_i \alpha(x-x_i)\end{align*} }
\end{note}

\begin{problem}
$P_{36}\;2.3$ 设$f(0)=1,f(1)=1,f(2)=2^2,f(3)=3^{3^3}=3^{27}$等. 一般地, $f(n)$等于高度为$n$的一叠$n$, 都是指数, 试证明$f$是原始递归的. 
\end{problem}
\begin{solution}
\textcolor{red}{自己没有想出来, 看了答案才想出来的, 一直纠结于单变量的情况, 却没有想到\textbf{构建多变量的函数, 之后给参数赋值来变回单变量}}\\
设$h(n,m)$是$m+1$个$n$的指数堆叠, 那么$h(n,0)=n,h(n,t+1)=n^{h(n,t)}$, 所以$h(n,m)$原始递归, 因此$h(n,n-1)=f(n)$原始递归.\\
\textcolor{blue}{虽然最后一步更自然的说法应该是取$n=m+1$, 那么$h(n,m)=h(m+1,m)=h(n,n-1)=f(n)$. 因为上面的$n$作为参数, 我是证明了$h(n,m)$这个二元函数关于$m$是原始递归的.}
\end{solution}

\begin{problem}
$P_{36}\;2.5$ 设$\sigma(0)=0$; 当$x\neq 0$时, $\sigma(x)$是$x$的所有因子的和. 证明$\sigma(x)$是原始递归函数.
\end{problem}
\begin{solution}
写成原始递归的形式:
\begin{align*}
    \sigma(x)=\sum_{i=1}^x \min_{i\leq t\leq n} \left(t|x\right)=\sum_{i=1}^x \min_{t\leq n} \left((t|x) \wedge (i\leq t) \right),\; x\neq 0
\end{align*}
谓词$(t|x)$和$(i\leq t)$是原始递归的, 原始递归谓词的"且"$\wedge$是原始递归的, 有界极小化$\min_{t\leq n} \left((t|x) \wedge (i\leq t) \right)$是原始递归的, 因此$\sigma(x)$也是原始递归的
\end{solution}

\begin{problem}
$P_{36}\; 2.7$ 设$\phi(x)$是小于等于$x$且与$x$互素的正整数的个数, 证明Euler函数$\phi(x)$是原始递归函数.
\end{problem}
\begin{solution}
先考虑判断$x,y$是否互素的量词$P(x,y)$是原始递归的. 
$$P(x,y)=(\forall)_{t\leq x}( (t=1) \vee \neg ((t|x) \wedge (t|y)) )=(\forall)_{t\leq x}( (t=1) \vee \neg (t|x) \vee \neg (t|y)) $$
因此:
$$\phi(x)=\sum_{t=1}^{x}P(x,t)$$
\end{solution}
\begin{note}
奇怪的是, 书上的答案是这么写的:
\begin{align*}
    P(x,y)= \textcolor{red}{(x>0 \vee y>0)} \wedge (\forall)_{t\leq x}( (t=1) \vee \neg (t|x) \vee \neg (t|y)) 
\end{align*}
即$x,y$不可以同时为$0$, \textcolor{red}{\textbf{这是必要的吗?}}
\end{note}

\begin{problem}
$P_{24}\; 2.4.3$ 验证: 
$$(x)_i=\min_{t\leq x} \{\neg (p_i^{t+1} | x )\}$$
对于$(0)_i$和$(x)_0$成立, 其中$x$和$i$是任意的自然数.
\end{problem}
\begin{solution}
\textcolor{blue}{实际含义: 第$i$个素数$p_i$作为$x$的因子的次数}
\begin{align*}
    (0)_i=\min_{t\leq 0} \{\neg (p_i^{t+1} | 0 )\}=\neg (p_i|0)=\neg 1 =0\\
    (x)_0=\min_{t\leq x} \{\neg (0^{t+1} | x )\}=\min_{t\leq x} \{\neg (0 | x )\}=\min_{t\leq x} \{1\}=0
\end{align*}
解释: $(0|x)=(\exists)_{t\leq x}(t\cdot 0=x)=0$
\end{solution}

\begin{problem}
$P_{37}\; 2.11$ 设$R(x,t)$是原始递归谓词, 定义有界极大化:
$$g(x,y)=\max_{t\leq y}R(x,t)$$
当存在$t\leq x$使得$R(x,t)$为真时, $g(x,y)$等于这样的$t$的最大值; 当不存在这样的$t$的时候, $g(x,y)=0$. 证明$g(x,y)$原始递归.
\end{problem}
\begin{solution}
\textcolor{blue}{第一反应是, 已经知道了有界极小化原始递归, 可以尝试用有界极小化来表示有界极大化.}\\
那么可以这么想: 使得$R(x,t)$成立的最大的$t$就是: 使得$k=t$到$y$的$R(x,k)$只有$R(x,t)$成立的最小的$t$.
$$g(x,y)=\min_{t\leq y} \left\{  R(x,t) \wedge\left(\textcolor{red} { \left( (t<y) \wedge \alpha \left(  \sum_{k=t+1}^{y} R(x,k) \right) \right) }\vee  \textcolor{blue}{(t=y)}    \right) \right\}$$
当不存在这样的$t$的时候, $R(x,t)$恒为$0$, 因此$g(x,y)=0$, 满足题设
\end{solution}
\begin{note}
书上给的两种解法也很有意思:\\
(1) 方法一: 和我的思路是一样的, 即首先$R(x,t)$要成立, 且要么$s\leq t$, 要么$R(x,s)$不成立. 但是他比我聪明的地方在于对 $\forall$的量词和$\vee$的使用
\begin{align*}
    g(x,y)=\min_{t\leq y}\left\{ R(x,t)\wedge (\forall s)_{\leq y} [(s\leq t) \vee \neg R(x,s) ] \right\}
\end{align*}
(2) 方法二: 倒着开始用有界极小化, 找到了再减回去得到正着数的下标
\begin{align*}
    g(x,y)=\begin{cases}
        y-\min_{t\leq x}R(x,y-t),& (\exists z)_{\leq y} R(x,z)\\
        0, & \text{否则}
    \end{cases}
\end{align*}
\end{note}

\begin{problem}
$P_{37}\;2.13$\\
(1) Cantor编码$\pi(x,y)$的定义如下所示\\
(2) 若$\pi(x,y)=z$, 则$\sigma_1(z)=x, \sigma_2(z)=y$\\
(3) $\sigma(z)=\sigma_1(z)+\sigma_2(z)$\\
证明$\pi(x,y),\sigma_1(z),\sigma_2(z),\sigma(z)$都是原始递归的.
\end{problem}

\begin{solution}
(1) $\pi(x,y)$是$x$行$y$列, 首先, 第$0$行, 第$y$列是$\frac{y(y+1)}{2}$, 所以$(x,y)$元素是:
$$\frac{y(y+1)}{2}+(y+2)+(y+3)+\cdots+(x+y+1)=\frac{(x+y)(x+y+1)}{2}+x$$
因此是原始递归函数\\
(2) \textcolor{red}{答案是这么写的, 但是还有点疑惑}
\begin{align*}
    \sigma_1(z)=\min_{x\leq z}[(\exists y)_{\leq z} \pi(x,y)=z]\\
    \sigma_2(z)=\min_{y\leq z}[(\exists x)_{\leq z} \pi(x,y)=z]
\end{align*}
(3) $\sigma_1(z),\sigma_2(z)$原始递归, 所以加和原始递归
\end{solution}
\end{document}
