\documentclass[10pt, a4paper, oneside]{ctexart}
\usepackage{amsmath, amsthm, amssymb, bm, color, xcolor, framed, graphicx, hyperref, mathrsfs, etoolbox, wrapfig, fdsymbol}
\usepackage[thicklines]{cancel}
\usepackage{enumitem}
\usepackage{geometry}
\usepackage{fancyhdr}

\hypersetup{
    colorlinks=true,
    linkcolor=black,
    filecolor=magenta,
    urlcolor=cyan,
    pdftitle={Overleaf Example},
    pdfpagemode=FullScreen,
}

\geometry{left=3cm, right=3cm, top=2.5cm, bottom=2.5cm}
\pagestyle{fancy}
\fancyhf{}
\fancyhead[L]{\leftmark}
\fancyhead[R]{\thepage}

\title{\textbf{理论计算机基础}}
\author{Little Wolf}
\date{\today}

\linespread{1.2}

\newenvironment{problem}{\begin{framed}\par\noindent\textbf{\textit{题目. }}}{\end{framed}\par}

\newenvironment{problemb}{\begin{framed}\par\noindent\textbf{\textcolor{blue}{\textit{非作业题目. }}}}{\end{framed}\par}

\newenvironment{solution}{%
  \par\noindent\textbf{\textit{解答. }}\ignorespaces
}{%
  \hfill\ensuremath{\square}\par
}
\newenvironment{note}{\par\noindent\textbf{\textit{题目的注记. }}\ignorespaces}{\par}

\allowdisplaybreaks

\begin{document}

\maketitle

\tableofcontents
\newpage

\section{\texorpdfstring{$\mathscr{S}$ 程序和可计算函数}{S 程序和可计算函数}}

\begin{problem}
    $P_9$ 1.3.6 对程序:
    \begin{align*}
        \mathscr{P}_2:& \\
        &\text{IF}\quad X\neq 0 \quad\text{GOTO}\quad A \\
        &Y\leftarrow Y+1\\
        &Z\leftarrow Z+1 \\
        &\text{IF} \quad Z\neq 0 \quad\text{GOTO}\quad E\\
        [A]& X\leftarrow X-1\\
        [B]& \text{IF} \quad X\neq 0 \quad \text{GOTO} \quad B
    \end{align*}
    给出它从输入变量$X$分别等于$0,1,5$的初始状态开始的计算.
    \end{problem}
    
    \begin{solution}
    (1) 思路: $X=0$, 不跳转到$[A]$, 之后$Y=1,Z=1$, 此时$Z\neq 0$, 跳转到$[E]$, 结束程序. 程序结束时, 输出变量$Y=1$.\\
    计算: 
    \begin{align*}
    &(1,\quad\{X=0,Y=0,Z=0\})\\
    &(2,\quad\{X=0,Y=0,Z=0\})\\
    &(3,\quad\{X=0,Y=1,Z=0\})\\
    &(4,\quad\{X=0,Y=1,Z=1\})\\
    &\textcolor{blue}{(7,\quad\{X=0,Y=1,Z=1\})}\quad(\text{终点快相})
    \end{align*}
    \\(2) 思路: $X=1$, 跳转到$[A]$, 执行$X\leftarrow X-1$后, $X=0$, 不执行$[B]$, 结束程序. 程序结束时, 输出变量$Y=0$.\\
    计算: 
    \begin{align*}
    &(1,\quad\{X=1,Y=0,Z=0\})\\   
    &(5,\quad\{X=1,Y=0,Z=0\})\\   
    &(6,\quad\{X=0,Y=0,Z=0\})\\   
    &(7,\quad\{X=0,Y=0,Z=0\}) 
    \end{align*}
    \\(3) 思路: $X=5$, 跳转到$[A]$, 执行$X\leftarrow X-1$后, $X=4$, 执行$[B]$, 进入死循环. 程序结束时, 输出变量$Y=0$.\\
    计算: 
    \begin{align*}
    &(1,\quad\{X=5,Y=0,Z=0\})\\
    &(5,\quad\{X=5,Y=0,Z=0\})\\
    &(6,\quad\{X=4,Y=0,Z=0\})\\
    &(6,\quad\{X=4,Y=0,Z=0\})\\
    &\cdots\quad(\text{死循环})
    \end{align*}
    \end{solution}
    
    \begin{problem}
    $P_9$ 1.3.7 对程序
    \begin{align*}
        \mathscr{P}_3:& \\
        &X_1\leftarrow X_1+1\\
        &X_1\leftarrow X_1+1\\
        [A]&X_1 \leftarrow X_1-1\\
        &\text{IF}\quad X_1\neq 0 \quad\text{GOTO}\quad C\\
        [B]& Z\leftarrow Z+1\\
        &\text{IF}\quad Z\neq 0\quad \text{GOTO}\quad B\\
        [C]& X_1\leftarrow X_1-1\\
        &\text{IF}\quad X_1\neq 0\quad \text{GOTO}\quad A\\
        &\text{IF}\quad X_2\neq 0\quad \text{GOTO}\quad D\\
        &Y\leftarrow Y+1\\
        [D]& Y\leftarrow Y
    \end{align*}
    设输入变量的初始状态的值如下: \\
    (1) $X_1=2, X_2=0$\\
    (1) $X_1=4, X_2=3$\\
    (1) $X_1=1, X_2=4$\\
    写出计算
    \end{problem}
    \begin{solution}
    (1) 分析: 执行了$[A]$后, $X_1=3$, 跳转到$C$, 之后$X_1=2$, 跳转回$[A]$, $X_1=1$, 再跳转到$[C]$, $X_1=0$, 而$X_2=0$, 执行$Y\leftarrow Y+1$, 之后进入\textcolor{red}{空指令}$[D]$. 最后输出变量$Y=1$.\\
    计算: 
    \begin{align*}
    &(1,\quad\{X_1=2,X_2=0,Y=0,Z=0\})\\
    &(2,\quad\{X_1=3,X_2=0,Y=0,Z=0\})\\
    &(3,\quad\{X_1=4,X_2=0,Y=0,Z=0\})\\
    &(4,\quad\{X_1=3,X_2=0,Y=0,Z=0\})\\
    &(7,\quad\{X_1=3,X_2=0,Y=0,Z=0\})\\
    &(8,\quad\{X_1=2,X_2=0,Y=0,Z=0\})\\
    &(3,\quad\{X_1=2,X_2=0,Y=0,Z=0\})\\
    &(4,\quad\{X_1=1,X_2=0,Y=0,Z=0\})\\
    &(7,\quad\{X_1=1,X_2=0,Y=0,Z=0\})\\
    &(8,\quad\{X_1=0,X_2=0,Y=0,Z=0\})\\
    &(9,\quad\{X_1=0,X_2=0,Y=0,Z=0\})\\
    &(10,\quad\{X_1=0,X_2=0,Y=0,Z=0\})\\
    &(11,\quad\{X_1=0,X_2=0,Y=1,Z=0\})\\
    &(12,\quad\{X_1=0,X_2=0,Y=1,Z=0\})
    \end{align*}
    \\(2) 思路: 执行$[A]$之后, $X_1=5$, 跳转$[C]$, 之后$X_1=4$, 跳转回$[A]$, 在$[C],[A]$间来回跳转, 根据$X_1$的奇偶性, 最后在执行$[C]$的第一步之后$X_1=0$, $X_2=3\neq 0$, 跳转到\textcolor{red}{空指令}$D$. 最后输出变量$Y=0$.\\
    计算: 
    \begin{align*}
    &(1, \quad \{X_1=4,X_2=3,Y=0,Z=0\})\\
    &(2, \quad \{X_1=5,X_2=3,Y=0,Z=0\})\\
    &(3, \quad \{X_1=6,X_2=3,Y=0,Z=0\})\\
    &(4, \quad \{X_1=5,X_2=3,Y=0,Z=0\})\\
    &(7, \quad \{X_1=5,X_2=3,Y=0,Z=0\})\\
    &(8, \quad \{X_1=4,X_2=3,Y=0,Z=0\})\\
    &(3, \quad \{X_1=4,X_2=3,Y=0,Z=0\})\\
    &(4, \quad \{X_1=3,X_2=3,Y=0,Z=0\})\\
    &(7, \quad \{X_1=3,X_2=3,Y=0,Z=0\})\\
    &(8, \quad \{X_1=2,X_2=3,Y=0,Z=0\})\\
    &(3, \quad \{X_1=2,X_2=3,Y=0,Z=0\})\\
    &(4, \quad \{X_1=1,X_2=3,Y=0,Z=0\})\\
    &(7, \quad \{X_1=1,X_2=3,Y=0,Z=0\})\\
    &(8, \quad \{X_1=0,X_2=3,Y=0,Z=0\})\\
    &(9, \quad \{X_1=0,X_2=3,Y=0,Z=0\})\\
    &(11, \quad \{X_1=0,X_2=3,Y=0,Z=0\})\\
    &(12, \quad \{X_1=0,X_2=3,Y=0,Z=0\})
    \end{align*}
    \\(3) 思路: 执行$[A]$之后, $X_1=2$, 跳转$[C]$, 之后$X_1=1$, 跳转回$[A]$, $X_1=0$, 执行$[B]$, $Z=1$, 在$[B]$中进入\textcolor{red}{死循环}. 最后输出变量$Y=0$.\\
    计算: 
    \begin{align*}
    &(1, \quad \{X_1=1,X_2=4,Y=0,Z=0\})\\
    &(2, \quad \{X_1=2,X_2=4,Y=0,Z=0\})\\
    &(3, \quad \{X_1=3,X_2=4,Y=0,Z=0\})\\
    &(4, \quad \{X_1=2,X_2=4,Y=0,Z=0\})\\
    &(7, \quad \{X_1=2,X_2=4,Y=0,Z=0\})\\
    &(8, \quad \{X_1=1,X_2=4,Y=0,Z=0\})\\
    &(3, \quad \{X_1=1,X_2=4,Y=0,Z=0\})\\
    &(4, \quad \{X_1=0,X_2=4,Y=0,Z=0\})\\
    &(5, \quad \{X_1=0,X_2=4,Y=0,Z=0\})\\
    &(6, \quad \{X_1=0,X_2=4,Y=0,Z=1\})\\
    &(5, \quad \{X_1=0,X_2=4,Y=0,Z=1\})\\
    &(6, \quad \{X_1=0,X_2=4,Y=0,Z=2\})\\
    &\cdots \quad (\text{进入死循环})
    \end{align*}
    \end{solution}
    
    \begin{problem}
    $P_{12}$ 1.1 写出计算下述函数的$\mathscr{S}$程序(允许使用宏指令):\\
    (1) $f(x)=\left\lfloor x/2 \right\rfloor$(向下取整)\\
    (2) $x$偶数, $f(x)=1$; $x$奇数, $f(x)$无定义.
    \end{problem}
    
    \begin{solution}
        (1) 思路: \textcolor{blue}{除以$2$可以用一直减$2$表示.}
        \begin{align*}
            \mathscr{P}_1:&\\
            &Z\leftarrow Z+1\\
            &X\leftarrow X+1\quad(\text{$+1$的目的是为了保证$2$的输出是$1$, 以此类推})\\
            [A]&X \leftarrow X-1\\
            &X \leftarrow X-1\\
            &\text{IF} \quad X\neq 0 \quad \text{GOTO} \quad B\\
            &\text{IF} \quad Z\neq 0 \quad \text{GOTO} \quad E\\
            [B]&Y\leftarrow Y+1\\
            &\text{IF} \quad Y\neq 0 \quad \text{GOTO} \quad A
        \end{align*}
        使用宏指令的版本:
        \begin{align*}
            \mathscr{P}_1^*:&\\
            &X\leftarrow X+1\\
            [A]&X \leftarrow X-2\\
            &\text{IF} \quad X\neq 0 \quad \text{GOTO} \quad B\\
            & \text{GOTO}\quad E\\
            [B]&Y\leftarrow Y+1\\
            &\text{GOTO}\quad A
        \end{align*}
        (2) 思路: 对输入的$X$, 循环减两次$1$, 但每次都检查$X$是否是$0$, 来判断奇偶性, 为了兼容$0$, 首先加上$1$. \textcolor{red}{简单来说, 就是看减去的是奇数个还是偶数个$1$来进行出口的分类.}
        \begin{align*}
            \mathscr{P}_2^*:&\\
            &X\leftarrow X+1\\
            [A]&X\leftarrow X-1\\
            &\text{IF}\quad X=0 \quad \text{GOTO}\quad  B\\
            &X\leftarrow X-1\\
            &\text{IF}\quad X\neq 0 \quad \text{GOTO}\quad  A\\
            & \text{GOTO} \quad C\\
            [B]&Y\leftarrow Y+1\\
            &\text{GOTO}\quad E\\
            [C]&Z\leftarrow Z+1\\
            &\text{IF}\quad Z\neq 0 \quad \text{GOTO}\quad  C\\
        \end{align*} 
        如果不允许判断$X=0$, 可以这么写:
        \begin{align*}
            \mathscr{P}_2^*:&\\
            &X\leftarrow X+1\\
            [A]&X\leftarrow X-1\\
            &\text{IF}\quad X\neq 0 \quad \text{GOTO}\quad  B\\
            &\text{GOTO} \quad C\quad(\text{偶数出口})\\
            [B]&X\leftarrow X-1\\
            &\text{IF}\quad X\neq 0 \quad \text{GOTO}\quad  A\\
            &\text{GOTO} \quad D\quad(\text{奇数出口})\\
            [C]&Y\leftarrow Y+1\\
            &\text{GOTO}\quad E\\
            [D]&Z\leftarrow Z+1\\
            &\text{IF}\quad Z\neq 0 \quad \text{GOTO}\quad  D\quad(\text{死循环})
        \end{align*} 
    \end{solution}
    
    \begin{note}
    可供使用的宏指令:\begin{itemize}
        \item GOTO $A$
        \item $V\leftarrow V^{\prime}$ 
        \item 判断 $X=0$ 和跳转
    \end{itemize}
    \end{note}
    
    \begin{problem}
    $P_{12}$ 1.2 给出下列程序$\mathscr{P}$计算的函数$\psi_{\mathscr{P}}^{(1)}(x)$:
    \begin{align*}
        (1) [A]&X\leftarrow X+1\\
        &X\leftarrow X-1\\
        &\text{IF}\quad X\neq 0 \quad \text{GOTO}\quad  A\\
        (2) [A]&X\leftarrow X-1\\
        &\text{IF}\quad X=0 \quad \text{GOTO}\quad  A\\
        &X\leftarrow X-1\\
        &\text{IF}\quad X\neq 0 \quad \text{GOTO}\quad  A\\
        (3) \text{空程序}&
    \end{align*}
    \end{problem}
    \begin{solution}
    (1) $\psi_{\mathscr{P}_1}^{(1)}(x) =
    \begin{cases}
        \uparrow (\text{未定义})  & \text{if } x \in \mathbb{N}^*, \\
      0 & \text{if } x = 0.
    \end{cases}$\\
    (2) $\psi_{\mathscr{P}_1}^{(1)}(x) =\begin{cases}0,&x\text{是正偶数}\\ \uparrow , &x=0 \text{或}x\text{是奇数}\end{cases} $\\
    (3) $\psi_{\mathscr{P}_1}^{(1)}(x)=0, \forall x\in \mathbb{N}$
    \end{solution}

\begin{problem}
$P_{12}$ 1.3 证明下面的函数是部分可计算的:\\
(1) $x_1+x_2$;  (2) $x_1-x_2$;  (3) $x_1x_2$;  (4) 空函数
\end{problem}
\begin{solution}
\textcolor{red}{要证明一个函数是部分可计算的, 实际上就是可以用$S$函数把它写出来, "部分"指的是可以在某些点上没有定义}\\
(1) 思路:$x_2$一直减$1$, $x_1$一直加$1$, 直到减到零
\begin{align*}
    &Y\leftarrow X_1\\
    &Z\leftarrow X_2\\
    [A]& \text{IF} \quad Z\neq 0 \quad \text{GOTO} \quad B\\
    & \text{GOTO}\quad E\\
    [B]& Z\leftarrow Z-1\\
    &Y\leftarrow Y+1\\
    &\text{GOTO}\quad  A
\end{align*} 
(2) 思路: 如果$x_1<x_2$, 那么无定义, 因此看谁先减到$0$.
\begin{align*}
    &Z_1\leftarrow X_1\\
    &Z_2\leftarrow X_2\\
    [A]& \text{IF} \quad Z_1\neq 0 \quad \text{GOTO} \quad B\\
    &\text{GOTO}\quad  D_1\\
    [B]& \text{IF} \quad Z_2\neq 0 \quad \text{GOTO} \quad C\\
    &\text{GOTO}\quad  D_2\\
    [C]& Z_1\leftarrow Z_1-1\\
    & Z_2\leftarrow Z_2-1\\
    &\text{GOTO}\quad  A\\
    [D_1]& Y\leftarrow Z_2\\
    &\text{GOTO}\quad  E\\
    [D_2]& Y\leftarrow Z_1\\
    &\text{GOTO}\quad  E
\end{align*}
(3) 思路: 如果有$0$, 返回$0$; 如果都不是$0$, 一直减$x_2$, 同时对$x_1$做加法(已经证明是部分可计算函数)
\begin{align*}
    &Z_1\leftarrow X_1\\
    &Z_2\leftarrow X_2\\
    [A]& \text{IF} \quad Z_1\neq 0 \quad \text{GOTO} \quad B\\
    &\text{GOTO}\quad  E\\
    [B]& \text{IF} \quad Z_2\neq 0 \quad \text{GOTO} \quad C\\
    &\text{GOTO}\quad  E\\
    [C]& Z_2\leftarrow Z_2-1\\
    &Z_1\leftarrow Z_1+Z_1\\
    &\text{GOTO}\quad  B
\end{align*}
(4) 思路: 空函数就是处处无定义, 直接进入死循环即可.
\begin{align*}
    [A]&Z\leftarrow Z+1\\
    & \text{IF} \quad Z\neq 0 \quad \text{GOTO} \quad A\\
\end{align*}
\end{solution}

\begin{problem}
    $P_{13}$ 1.4 证明下述谓词是可计算的\\
(1) $x\geq a$, $a$是正整数\\
(2) $x_1\leq x_2$\\
(3) $x_1=x_2$
\end{problem}
\begin{solution}
\textcolor{red}{要证明一个谓词是可计算的,实际上就是证明这个谓词(判断过程)可以用$S$语言表示}\\
(1) 思路: 两边一直减$1$, 看谁先减到$0$, \textcolor{blue}{又因为是大于等于, 所以可以先验证$Z_2$}
\begin{align*}
    &Z_1\leftarrow X\\
    &Z_2\leftarrow a\\
    [A]& \text{IF} \quad Z_2\neq 0 \quad \text{GOTO} \quad B\\
    &\text{GOTO}\quad  D_2\\
    [B]& \text{IF} \quad Z_1\neq 0 \quad \text{GOTO} \quad C\\
    &\text{GOTO}\quad  D_1\\
    [C]&Z_1\leftarrow Z_1-1\\
    &Z_2\leftarrow Z_2-1\\
    &\text{GOTO}\quad  A\\
    [D_1]&\text{GOTO}\quad  E\\
    [D_2]&Y\leftarrow Y+1\\
    &\text{GOTO}\quad  E
\end{align*}
(2) 思路: 和第一问思路类似, 注意取等条件对判定顺序的影响
\begin{align*}
    &Z_1\leftarrow X_1\\
    &Z_2\leftarrow X_2\\
    [A]& \text{IF} \quad Z_1\neq 0 \quad \text{GOTO} \quad B\\
    &\text{GOTO}\quad  D_2\\
    [B]& \text{IF} \quad Z_2\neq 0 \quad \text{GOTO} \quad C\\
    &\text{GOTO}\quad  D_1\\
    [C]&Z_1\leftarrow Z_1-1\\
    &Z_2\leftarrow Z_2-1\\
    &\text{GOTO}\quad  A\\
    [D_1]&\text{GOTO}\quad  E\\
    [D_2]&Y\leftarrow Y+1\\
    &\text{GOTO}\quad  E
\end{align*}
(3) 思路: 可以使用(1)和(2)的判定了
\begin{align*}
    &Z_1\leftarrow X_1\\
    &Z_2\leftarrow X_2\\
    [A]&\text{IF} \quad Z_1\leq Z_2 \quad \text{GOTO} \quad B\\
    &\text{GOTO}\quad  E\\
    [B]&\text{IF} \quad Z_2\leq Z_1 \quad \text{GOTO} \quad C\\
    &\text{GOTO}\quad  E\\
    [C]&Y\leftarrow Y+1\\
    &\text{GOTO}\quad  E
\end{align*}
\end{solution}

\section{原始递归函数}

\begin{problem}
$P_{16}\; 2.1.5$用基本的原始递归函数来表示下面的函数, 从而它们也是原始递归函数\\
(1) $E(x)=\begin{cases}
    1, &x\text{偶数}\\
    0, &x\text{奇数}
\end{cases}$\\
(3) $\max(x,y)=\begin{cases}
    x&,x\geq y\\
    y,&x<y
\end{cases}$
\end{problem}
\begin{solution}
(1) $E(0)=1, E(x+1)=\alpha(E(x))$.\\
(2) $\max(x,y)=x\alpha(y\dotminus x)+y\alpha(x \dotminus p(y))$\\
这是因为, 当$x\geq y$时, $y\dotminus x=0$, 因此$\alpha(y\dotminus x)=1$. 但为了防止$x=y$且取非零值($x=y=0$不影响)的时候, 得到$2x$, 因此要保证$x=y$的时候, $y$的系数不能是$1$, 因此取$x\dotminus p(y)$ 
\end{solution}

\begin{problem}
$P_{22}\;2.3.4$利用极小化给出下述函数$f(x)$:\\
(1) $f(x)=\begin{cases}
    \sqrt{x}, &x\text{是完全平方数}\\
    \uparrow, &\text{否则}
\end{cases}$\\
(2) $g(x)$是全函数, 若存在$t$, 使得$g(t)=x$, 则$f(x)$等于使得$g(t)=x$成立的最小的$t$, 否则$f(x)\uparrow$.
\end{problem}
\begin{solution}
(1) $f(x)=\min_{t}(t^2=x)$\\
(2) $f(x)=\min_{t}(g(t)=x)$
\end{solution}
\begin{note}
\textbf{\textcolor{blue}{应该是对的吧}}: \textcolor{red}{使用极小化定义的函数中,加入我想要定义的函数不是全函数(即在某些情况下无定义), 那么一定是用无界极小化定义的. 因为有界极小化+原始递归,定义出的都是全函数}
\end{note}

\begin{problem}
$P_{36}\; 2.1$ 证明: 仅在有穷个点处非零值, 在其余点取零的函数一定是原始递归函数.
\end{problem}
\begin{solution}
 思路:因为只在有限个地方取非零值, 那么只需要对这有限个点做有限递归\\
设$f(x)$在$x_1,\cdots,x_k$处取非零值$c_1,\cdots,c_k$, 其余点取零.\\
定义:
$$\chi_{x_i}(x)=\begin{cases}
    1,&x=x_i\\
    0,&x\neq x_i
\end{cases}=(x=x_i)$$
是原始递归的谓词, 那么定义:
\begin{align*}
    f(x)=\sum_{i=1}^{k}c_i\cdot\chi_{x_i}(x)
\end{align*}
(有限求和,常数乘法原始递归)因此也是原始递归的.
\end{solution}
\begin{note}
    \textcolor{blue}{也可以写成:\begin{align*}f(x)=\sum_{i=1}^{k}c_i \alpha(x-x_i)\end{align*} }
\end{note}

\begin{problem}
$P_{36}\;2.3$ 设$f(0)=1,f(1)=1,f(2)=2^2,f(3)=3^{3^3}=3^{27}$等. 一般地, $f(n)$等于高度为$n$的一叠$n$, 都是指数, 试证明$f$是原始递归的. 
\end{problem}
\begin{solution}
\textcolor{red}{自己没有想出来, 看了答案才想出来的, 一直纠结于单变量的情况, 却没有想到\textbf{构建多变量的函数, 之后给参数赋值来变回单变量}}\\
设$h(n,m)$是$m+1$个$n$的指数堆叠, 那么$h(n,0)=n,h(n,t+1)=n^{h(n,t)}$, 所以$h(n,m)$原始递归, 因此$h(n,n-1)=f(n)$原始递归.\\
\textcolor{blue}{虽然最后一步更自然的说法应该是取$n=m+1$, 那么$h(n,m)=h(m+1,m)=h(n,n-1)=f(n)$. 因为上面的$n$作为参数, 我是证明了$h(n,m)$这个二元函数关于$m$是原始递归的.}
\end{solution}

\begin{problem}
$P_{36}\;2.5$ 设$\sigma(0)=0$; 当$x\neq 0$时, $\sigma(x)$是$x$的所有因子的和. 证明$\sigma(x)$是原始递归函数.
\end{problem}
\begin{solution}
写成原始递归的形式:
\begin{align*}
    \sigma(x)=\sum_{i=1}^x \min_{i\leq t\leq n} \left(t|x\right)=\sum_{i=1}^x \min_{t\leq n} \left((t|x) \wedge (i\leq t) \right),\; x\neq 0
\end{align*}
谓词$(t|x)$和$(i\leq t)$是原始递归的, 原始递归谓词的"且"$\wedge$是原始递归的, 有界极小化$\min_{t\leq n} \left((t|x) \wedge (i\leq t) \right)$是原始递归的, 因此$\sigma(x)$也是原始递归的
\end{solution}

\begin{problem}
$P_{36}\; 2.7$ 设$\phi(x)$是小于等于$x$且与$x$互素的正整数的个数, 证明Euler函数$\phi(x)$是原始递归函数.
\end{problem}
\begin{solution}
先考虑判断$x,y$是否互素的量词$P(x,y)$是原始递归的. 
$$P(x,y)=(\forall)_{t\leq x}( (t=1) \vee \neg ((t|x) \wedge (t|y)) )=(\forall)_{t\leq x}( (t=1) \vee \neg (t|x) \vee \neg (t|y)) $$
因此:
$$\phi(x)=\sum_{t=1}^{x}P(x,t)$$
\end{solution}
\begin{note}
奇怪的是, 书上的答案是这么写的:
\begin{align*}
    P(x,y)= \textcolor{red}{(x>0 \vee y>0)} \wedge (\forall)_{t\leq x}( (t=1) \vee \neg (t|x) \vee \neg (t|y)) 
\end{align*}
即$x,y$不可以同时为$0$, \textcolor{red}{\textbf{这是必要的吗?}}
\end{note}

\begin{problem}
$P_{24}\; 2.4.3$ 验证: 
$$(x)_i=\min_{t\leq x} \{\neg (p_i^{t+1} | x )\}$$
对于$(0)_i$和$(x)_0$成立, 其中$x$和$i$是任意的自然数.
\end{problem}
\begin{solution}
\textcolor{blue}{实际含义: 第$i$个素数$p_i$作为$x$的因子的次数}
\begin{align*}
    (0)_i=\min_{t\leq 0} \{\neg (p_i^{t+1} | 0 )\}=\neg (p_i|0)=\neg 1 =0\\
    (x)_0=\min_{t\leq x} \{\neg (0^{t+1} | x )\}=\min_{t\leq x} \{\neg (0 | x )\}=\min_{t\leq x} \{1\}=0
\end{align*}
解释: $(0|x)=(\exists)_{t\leq x}(t\cdot 0=x)=0$
\end{solution}

\begin{problem}
$P_{37}\; 2.11$ 设$R(x,t)$是原始递归谓词, 定义有界极大化:
$$g(x,y)=\max_{t\leq y}R(x,t)$$
当存在$t\leq x$使得$R(x,t)$为真时, $g(x,y)$等于这样的$t$的最大值; 当不存在这样的$t$的时候, $g(x,y)=0$. 证明$g(x,y)$原始递归.
\end{problem}
\begin{solution}
\textcolor{blue}{第一反应是, 已经知道了有界极小化原始递归, 可以尝试用有界极小化来表示有界极大化.}\\
那么可以这么想: 使得$R(x,t)$成立的最大的$t$就是: 使得$k=t$到$y$的$R(x,k)$只有$R(x,t)$成立的最小的$t$.
$$g(x,y)=\min_{t\leq y} \left\{  R(x,t) \wedge\left(\textcolor{red} { \left( (t<y) \wedge \alpha \left(  \sum_{k=t+1}^{y} R(x,k) \right) \right) }\vee  \textcolor{blue}{(t=y)}    \right) \right\}$$
当不存在这样的$t$的时候, $R(x,t)$恒为$0$, 因此$g(x,y)=0$, 满足题设
\end{solution}
\begin{note}
书上给的两种解法也很有意思:\\
(1) 方法一: 和我的思路是一样的, 即首先$R(x,t)$要成立, 且要么$s\leq t$, 要么$R(x,s)$不成立. 但是他比我聪明的地方在于对 $\forall$的量词和$\vee$的使用
\begin{align*}
    g(x,y)=\min_{t\leq y}\left\{ R(x,t)\wedge (\forall s)_{\leq y} [(s\leq t) \vee \neg R(x,s) ] \right\}
\end{align*}
(2) 方法二: 倒着开始用有界极小化, 找到了再减回去得到正着数的下标
\begin{align*}
    g(x,y)=\begin{cases}
        y-\min_{t\leq x}R(x,y-t),& (\exists z)_{\leq y} R(x,z)\\
        0, & \text{否则}
    \end{cases}
\end{align*}
\end{note}

\begin{problem}
$P_{37}\;2.13$\\
(1) Cantor编码$\pi(x,y)$的定义如下所示\\
(2) 若$\pi(x,y)=z$, 则$\sigma_1(z)=x, \sigma_2(z)=y$\\
(3) $\sigma(z)=\sigma_1(z)+\sigma_2(z)$\\
证明$\pi(x,y),\sigma_1(z),\sigma_2(z),\sigma(z)$都是原始递归的.
\end{problem}

\begin{solution}
(1) $\pi(x,y)$是$x$行$y$列, 首先, 第$0$行, 第$y$列是$\frac{y(y+1)}{2}$, 所以$(x,y)$元素是:
$$\frac{y(y+1)}{2}+(y+2)+(y+3)+\cdots+(x+y+1)=\frac{(x+y)(x+y+1)}{2}+x$$
因此是原始递归函数\\
(2) \textcolor{red}{答案是这么写的, 但是还有点疑惑}
\begin{align*}
    \sigma_1(z)=\min_{x\leq z}[(\exists y)_{\leq z} \pi(x,y)=z]\\
    \sigma_2(z)=\min_{y\leq z}[(\exists x)_{\leq z} \pi(x,y)=z]
\end{align*}
(3) $\sigma_1(z),\sigma_2(z)$原始递归, 所以加和原始递归
\end{solution}

\begin{problem}
    $P_{38}$ 2.21 证明下述字函数是原始递归的:

(1) 长度函数 $|w|$

$|w|$ 等于 $w$ 的长度,即 $w$ 中的字符个数;

(2) 连接函数 $\operatorname{CONCAT}^{(m)}\left(w_1, w_2, \cdots, w_m\right)$ ,其中 $m$ 是正整数,
$$
\operatorname{CONCAT}^{(m)}\left(w_1, w_2, \cdots, w_m\right)=w_1 w_2 \cdots w_m
$$

(3) 字尾函数 RTEND(w)

当 $w=\varepsilon$ 时, RTEND $(w)=\varepsilon$ ;当 $w \neq \varepsilon$ 时,RTEND $(w)$ 等于 $w$ 最右端的字符;

(4) 字头函数 $\operatorname{LTEND}(w)$

当 $w=\varepsilon$ 时, $\operatorname{LTEND}(w)=\varepsilon$ ;当 $w \neq \varepsilon$ 时,LTEND $(w)$ 等于 $w$ 最左端的字符;

(5) 截尾函数 $w^{-}$

当 $w=\varepsilon$ 时, $w^{-}=\varepsilon$ ;当 $w \neq \varepsilon$ 时, $w^{-}$等于 $w$ 删去最右端的字符后的子串;

(6)截头函数 $^-w$

当 $w=\varepsilon$ 时, $^-w=\varepsilon$ ;当 $w \neq \varepsilon$ 时, $^-w$ 等于 $w$ 删去最左端的字符后的子串。
\end{problem}
\begin{solution}
\textcolor{red}{要证明字函数是原始递归函数, 需要证明这个字函数对应的数论函数是原始递归的; 或者是由其他的原始递归的字函数构造得到的}

指定字母表$A=\{a_1,\cdots,a_n\}$

(1) \textcolor{blue}{从直观上说, 给定一个字符串$w$, 获取$w$的长度是不需要知道$w$的字符所属的字母表的, 但是, 这里为了分析这个过程的\textbf{可计算性}, 需要指定一个字母表.}

那么在$n$进制下, 长度为$k+1$的最小字符串是$a_1a_1\cdots a_1=\sum_{t=0}^k n^{t}$, 那么长度是(控制$k\leq x$, 可以是有界极小化, 这里的$x$在比较的时候, 就是数), 根据有界极小化和原始递归谓词定义的函数是原始递归的:
$$\min_{ k\leq x } \{\sum_{t=0}^k n^k > x \} $$

(2) 连接函数, $w_1$向前挪动一格, 值乘上$n$, 而向前挪动的格子数可以由第一问的长度函数得到, 不妨设长度函数是$l(w)$, 那么: 
\begin{align*}
    &\text{CONCAT}^{(m)}(w_1,\cdots,w_m)\\
    =& w_m + w_{m-1}\cdot n^{l(w_m)}+ w_{m-1}\cdot n^{l(w_m)+l(w_{m-1})}+\cdots+w_1\cdot n^{l_(w_m)+\cdots+l(w_2)}\\
    =& \sum_{i=0}^{m-1} w_{m-i}\cdot n^{l_{w_m}+\cdots+l_{w_{m-i+1}}}
\end{align*}

(3) 首先, 判断$w=\epsilon$当然是原始递归的谓词, 因为这就是$w=0$的判断, 其次, 获取$w$最右端字符就是$R^{+}(w,n)$, 因此是原始递归的 
\begin{align*}
    \text{RTEND}(w)=\begin{cases}
        \epsilon, &\text{if} \quad w=\epsilon\\
        R^+(w,n), &\text{else} 
    \end{cases}
\end{align*}

(4) 长度为$l(w)$的字符串的首位是$n^{l(w)-1}$的阶, 因此用$Q^+(w,n^{l(w)-1})$获取最左端的字符(即$w$用 $n^{l(w)-1}$来除之后得到的整数部分), 因此是原始递归的 
\begin{align*}
    \text{LTEND}(w)=\begin{cases}
        \epsilon, &\text{if} \quad w=\epsilon\\
        Q^+(w,n^{l(w)-1}), &\text{else} 
    \end{cases}
\end{align*}
当然, 也可以用书上讲过的, 可以直接拿来用的函数, 更加安全, \textcolor{blue}{$h(m,n,x)$表示字符串$x$在$n$进制表示下第$m$位的字符}, 因此可以写成:
\begin{align*}
    \text{LTEND}(w)=\begin{cases}
        \epsilon, &\text{if} \quad w=\epsilon\\
        h(l(w)-1,n,x), &\text{else} 
    \end{cases}
\end{align*}

(5) 截尾函数$w^-$, 除$n$后留下来的整数部分就是向右挪动一格的结果, 很容易用原始递归函数来表示, 注意$R^+(w,n)$是余数部分, $Q^+(w,n)$才是整数部分:
\begin{align*}
    w^{-}=\begin{cases}
        \epsilon,& \text{if}\quad w=\epsilon\\
        Q^{+}(w,n), &\text{else}
    \end{cases}
\end{align*}

(6) 可以获得长度$l(w)$, 最左端字符的的阶是$n^{l(w)-1}$, 可以根据LTEND$(w)$获得首位字符是什么, 那么减去即可:
\begin{align*}
    ^-w=\begin{cases}
        \epsilon, &\text{if}\quad w=\epsilon\\
        w-\text{LTEND}(w)\cdot n^{l(w)-1}, &\text{else}
    \end{cases}
\end{align*}
\end{solution}

\begin{problem}
    $P_{38}$ 2.22 设两个字母表 $A=\left\{s_1, s_2, \cdots, s_n\right\}$ 和 $\tilde{A}=\left\{s_1, s_2, \cdots, s_m\right\}$, 其中 $n<m$ 。任给一个数 $x$, 设 $w \in A^*$ 是$x$的$n$进制表示, $w$也是$\tilde{A}^*$上的字符串, 把以$m$为底所表示的数记为$\text{UPCHANGE}_{n,m}(x)$.

    反之,任给一个数 $x$ ,设 $w \in \tilde{A}^*$ 是 $x$ 的 $m$ 进制表示,删去 $w$ 中不属于 $A$ 的符号后得到 $w^{\prime} \in A^*$ ,把以 $n$ 为底 $w^{\prime}$ 所表示的数记作DOWNCHANGE ${ }_{n, m}(x)$ 。例如, DOWNCHANGE ${ }_{2,5}(36)=9$. 又如, $s_1 s_4 s_2 s_3$ 以 5 为底表示 238 ,删去 $s_4$ 和 $s_3$ 得到 $s_1 s_2, s_1 s_2$, 以 2 为底表示 4 , 故DOWNCHANGE ${ }_{2,5}(238)=4$.
    
    试证明: DOWNCHANGE$_{n, m}(x)$ 和UPCHANGE$_{n, m}(x)$ 是原始递归的.
\end{problem}

\begin{solution}
(1) 对UPCHANGE$_{n,m}(x)$, 用原始递归的函数来表示它. 通过$l_{n}(x)$来获取$x$在$n$进制表示下的长度, 用$h(k,n,x), 0\leq k\leq l_n(w)-1$来获取每一位在字母表$A$中的序数, 因为$A\subseteq \tilde{A}$, 因此也是在字母表$\tilde{A}$中的序数, 然后通过$m$进制表示即可:
\begin{align*}
    \text{UPCHANGE}_{n,m}(x)=\sum_{k=0}^{l_n(x)-1} h(k,n,x)\cdot m^{k}
\end{align*}

(2) 对于$\text{DOWNCHANGE}_{n,m}(x)$, 用$l_{m}(x)$来获取在$m$进制下的长度, 用$h(k,m,x), 0\leq k\leq l_m(x)-1$来获取每一位的字符. 删去不属于$A$的字符的操作只需要考察$h(k,m,x)$是否小于等于$n$即可, \textcolor{red}{而对于删去一些字符后的位数, 可以用累加计数的方法来实现}, 函数$P(x)=\begin{cases}
    1,&\text{if}\quad x\leq n\\
    0,&\text{else}
\end{cases}$:
\begin{align*}
    \text{DOWNCHANGE}_{n,m}(x)=\sum_{k=0}^{l_m(x)-1} h(k,m,x)\cdot P(h(k,m,x)) \cdot n^{\sum_{i=0}^{k} P(h(k,m,x))}
\end{align*}
\end{solution}

\begin{note}
(1)的书上的答案中写的是:
\begin{align*}
    \text{UPCHANGE}_{n,m}(x)=\sum_{k=0}^{\textcolor{red}{l_n(x)}} h(k,n,x)\cdot m^{k}
\end{align*}

(1)的书上的答案中写的是:
\begin{align*}
    \text{DOWNCHANGE}_{n,m}(x)=\sum_{k=0}^{ \textcolor{red}{l_m(x)} } h(k,m,x)\cdot P(h(k,m,x)) \cdot n^{\sum_{i=0}^{ \textcolor{red}{k-1} } P(h(k,m,x))}
\end{align*}

\textcolor{blue}{我仔细检查了, 应该是答案写错了?}
\end{note}

\section{通用程序}

\begin{problem}
计算$\text{STP}(x_1,\cdots,x_n,y,0)=$?
\end{problem}
\begin{solution}
    $$\text{STP}(x_1,\cdots,x_n,y,0)=\begin{cases}
        1, & \text{if}\quad y=0\\
        0, &\text{否则} 
    \end{cases}$$
\end{solution}

\begin{problem}
    3.1 证明HALT$(0,x)$是不可计算的
\end{problem}
\begin{solution}
反证法, 假设HALT$(0,x)$可计算, 构造:
$$[A] \text{IF} \quad  \text{HALT}(0,x) \quad \text{GOTO} \quad A $$
这是可计算的, 有程序$P$可以计算, 程序$P$的编号是$k$.\\
那么HALT$(0,k) \iff P(k)=0 \iff \neg $HALT$(0,k)$, 矛盾.
\end{solution}

\begin{problem}
    3.3 证明: 不存在可计算函数$f(x)$, 使得当$\Phi(x,x)\downarrow$时$f(x) = \Phi(x,x)+1$.
\end{problem}
\begin{solution}
    反证法, 如果是可计算的, 那么函数$f(x)$有编号$k$, 那么\textcolor{red}{$f(x)=\Phi(x,k)$}, 因此有$f(k)=\Phi(k,k)=\Phi(k,k)+1$, 矛盾.
\end{solution}

\begin{problem}
设$f$是一个全函数, $B=\{f(n)| n\in N\}$, 证明:\\
(1) 若$f$是可计算的, 那么$B$是$r.e.$\\
(2) 若$f$是可计算的, 严格增加的($f(n)<f(n+1), \forall n\in N$), 则$B$是递归的.
\end{problem}
\begin{solution}
    (1) \textcolor{blue}{构造一个可计算函数, 使得输入的$x$如果在$B$里面, 就停机(输出什么都行), 否则不停机(无定义)}
    \begin{align*}
        h(x) = \begin{cases}
            0, &\text{if} \quad \exists n (f(n)=x)\\
            \uparrow, &\text{else}
        \end{cases}
    \end{align*}
    可计算函数:
    \begin{align*}
        [A] &\text{IF} \quad f(N)=X\quad \text{GOTO} \quad E\\
        & N\leftarrow N+1\\
        & \text{GOTO} \quad A
    \end{align*}
    (2)\textcolor{blue}{构造一个可计算函数, 使得输入的$x$如果在$B$里面, 就输出$1$, 如果不在里面, 就输出$0$}\\
    \textcolor{red}{又因为函数是严格单调的, 因此我只需要遍历$N$, 直到$f(N)$大于$X$或者$f(N)=X$即可.}
    \begin{align*}
        [A] &\text{IF} \quad f(N)=X\quad \text{GOTO} \quad C\\
        [B] &\text{IF} \quad f(N)>X\quad \text{GOTO} \quad E\\
        & N\leftarrow N+1\\
        & \text{GOTO} \quad A\\
        [C]& Y\leftarrow Y+1\\
        & \text{GOTO} \quad E
    \end{align*}
\end{solution}
\begin{note}
    \begin{itemize}
        \item 存在不是 可计算 的 全函数, 例子(这是一个不可计算的谓词): HALT(x,x)
        \item 存在不是 部分可计算 的 部分函数, 例子: $f(x)= \begin{cases}
            1, & \text{if } \neg \text{HALT}(x,x)\\
            0, & \text{if } \text{HALT}(x,x)
        \end{cases}$
    \end{itemize}
\end{note}

\begin{problem}
3.6 设$A,B$是$N$的非空子集, 定义:
\begin{align*}
    A\odot B & = \{2x| x\in A\}\cup \{2x+1 | x\in B\}\\
    A \otimes B & = \{ <x,y>| x\in A,y\in B  \}
\end{align*}
证明:\\
(1) $A\odot B$是递归的当且仅当$A$和$B$是递归的\\
(2) $A\otimes B$是递归的当且仅当$A$和$B$是递归的
\end{problem}
\begin{note}
实际上, 上述两个集合的构造方式都是"双射"的编码, 因此只要知道$x$\textcolor{red}{是否属于}$A\odot B$, 那么就一定知道$x$\textcolor{red}{是否属于}$A$或者$B$, 反之亦然(显然)
\end{note}
\begin{solution}
(1) 分析谓词: $(x\in A\odot B)$, $(x\in A)$, $(x\in B)$:
\begin{align*}
    (x\in A\odot B) \iff ( (2|x) \wedge (\lfloor x/2 \rfloor\in  A ))
    \vee ( \neg(2|x) \wedge  ( \lfloor (x-1)/2 \rfloor  ) \in B)
\end{align*}
(2) 对于配对函数$z=<x,y>$, 有原始递归函数$l(z),r(z)$.\\
分析谓词: $(x\in A\otimes B)$, $(x\in A)$, $(x\in B)$:
\begin{align*}
    (z\in A\otimes B) \iff ( (x = l(z)) \wedge (x \in  A ))
    \vee ( (x = r(z)) \wedge  ( x \in B))
\end{align*}
\end{solution}

\begin{problem}
3.7 证明集合$B = \{x\in N |   a\in \text{ran}\Phi_x \}$是递归可枚举的, $a$是常数.
\end{problem}
\begin{solution}
\textcolor{blue}{这里的$\text{ran}\Phi_x$就是值域\\注意, 这里的函数$\Phi_x$是部分可计算的, 而不是像3.5的全函数\\我需要构造一个可计算函数来判断$x$}\textcolor{red}{是属于}\textcolor{blue}{$B$的, 即, 如果属于, 我知道, 但是不知道不属于}\\
需要再添加一个程序来执行对$\Phi(n_0,x)$是否有定义的判定.
\begin{align*}
    [A] &\text{IF} \quad \neg STP(N,X,T)  \quad \text{GOTO} \quad B\\
    & \text{IF} \quad \Phi(N,X)=a  \quad \text{GOTO} \quad E\quad \quad \textcolor{blue}{\text{需要对输入N停机之后才能有输出来判断是不是a}}\\
    [B]& N\leftarrow N+1\\
    & \text{IF} \quad N\leq T \quad \text{GOTO} \quad A\\
    &T\leftarrow T+1 \\
    &N\leftarrow \\
    &\text{GOTO} \quad A
\end{align*}
\end{solution}

\newpage

\section{Turing机}

\begin{problem}
$P_{55}$ 4.1.4 设对$w\in \{0,1\}^*$, 有$f(w)=\overline{w}$, 其中$\overline{w}$是$w$的反码, 即$0\to 1, 1\to 0$. 构造一台TM计算$\overline{w}$.
\end{problem}

\begin{solution}

\begin{figure}[h]
    \centering
    \includegraphics[width=0.6\textwidth]{image/1.png}
\end{figure}
\end{solution}

\begin{problem}
$P_{67}$ 4.1 设字母表$A=\{a,b\}$, $f(x)=x^R$, $x^R$是$x$的反转, 即颠倒$x$的符号的排列顺序所得到的字符串. 例如, $(abab)^R=baba$, 试构造一台TM计算$f(x)$.
\end{problem}
\begin{solution}
思路: 倒着读(从右往左读), 然后在字符串最右端的空白符的右边开始写(这样只需要把记录过的原字符串的字符改写成x, 而不用最后还跑回去改写会空白了吗?)
\begin{figure}[h]
    \centering
    \includegraphics[width=0.9\textwidth]{image/2.png}
\end{figure}
这样的TM输出的会是$x^R$, 之所以前面没有附带了相同长度的$x$字符串, 是因为\textcolor{red}{图灵机在接收状态只会输出输出带上在$A$中的字符}.
\newpage
当然, 如果想把$x$改为$B$也可以做到. 可以把状态$q_8$再添加一部分: 如果遇到的是$x$, 那么把$x$改写为$B$之后向右移动一格, 那么就可以把所有的$x$都变为$B$:
\begin{figure}[h]
    \centering
    \includegraphics[width=0.9\textwidth]{image/3.png}
\end{figure}
\end{solution}

\end{document}
