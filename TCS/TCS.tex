\documentclass[10pt, a4paper, oneside]{ctexart}
\usepackage{amsmath, amsthm, amssymb, bm, color, xcolor, framed, graphicx, hyperref, mathrsfs, etoolbox, wrapfig, fdsymbol}
\usepackage[thicklines]{cancel}
\usepackage{enumitem}
\usepackage{geometry}
\usepackage{fancyhdr}

\hypersetup{
    colorlinks=true,
    linkcolor=black,
    filecolor=magenta,
    urlcolor=cyan,
    pdftitle={Overleaf Example},
    pdfpagemode=FullScreen,
}

\geometry{left=3cm, right=3cm, top=2.5cm, bottom=2.5cm}
\pagestyle{fancy}
\fancyhf{}
\fancyhead[L]{\leftmark}
\fancyhead[R]{\thepage}

\title{\textbf{理论计算机基础}}
\author{Little Wolf}
\date{\today}

\linespread{1.2}

\newenvironment{problem}{\begin{framed}\par\noindent\textbf{\textit{题目. }}}{\end{framed}\par}
\newenvironment{solution}{%
  \par\noindent\textbf{\textit{解答. }}\ignorespaces
}{%
  \hfill\ensuremath{\square}\par
}
\newenvironment{note}{\par\noindent\textbf{\textit{题目的注记. }}\ignorespaces}{\par}

\allowdisplaybreaks

\begin{document}

\maketitle

\tableofcontents
\newpage

\section{\texorpdfstring{$\mathscr{S}$ 程序和可计算函数}{S 程序和可计算函数}}

\begin{problem}
    $P_9$ 1.3.6 对程序:
    \begin{align*}
        \mathscr{P}_2:& \\
        &\text{IF}\quad X\neq 0 \quad\text{GOTO}\quad A \\
        &Y\leftarrow Y+1\\
        &Z\leftarrow Z+1 \\
        &\text{IF} \quad Z\neq 0 \quad\text{GOTO}\quad E\\
        [A]& X\leftarrow X-1\\
        [B]& \text{IF} \quad X\neq 0 \quad \text{GOTO} \quad B
    \end{align*}
    给出它从输入变量$X$分别等于$0,1,5$的初始状态开始的计算.
    \end{problem}
    
    \begin{solution}
    (1) 思路: $X=0$, 不跳转到$[A]$, 之后$Y=1,Z=1$, 此时$Z\neq 0$, 跳转到$[E]$, 结束程序. 程序结束时, 输出变量$Y=1$.\\
    计算: 
    \begin{align*}
    &(1,\quad\{X=0,Y=0,Z=0\})\\
    &(2,\quad\{X=0,Y=0,Z=0\})\\
    &(3,\quad\{X=0,Y=1,Z=0\})\\
    &(4,\quad\{X=0,Y=1,Z=1\})\\
    &\textcolor{blue}{(7,\quad\{X=0,Y=1,Z=1\})}\quad(\text{终点快相})
    \end{align*}
    \\(2) 思路: $X=1$, 跳转到$[A]$, 执行$X\leftarrow X-1$后, $X=0$, 不执行$[B]$, 结束程序. 程序结束时, 输出变量$Y=0$.\\
    计算: 
    \begin{align*}
    &(1,\quad\{X=1,Y=0,Z=0\})\\   
    &(5,\quad\{X=1,Y=0,Z=0\})\\   
    &(6,\quad\{X=0,Y=0,Z=0\})\\   
    &(7,\quad\{X=0,Y=0,Z=0\}) 
    \end{align*}
    \\(3) 思路: $X=5$, 跳转到$[A]$, 执行$X\leftarrow X-1$后, $X=4$, 执行$[B]$, 进入死循环. 程序结束时, 输出变量$Y=0$.\\
    计算: 
    \begin{align*}
    &(1,\quad\{X=5,Y=0,Z=0\})\\
    &(5,\quad\{X=5,Y=0,Z=0\})\\
    &(6,\quad\{X=4,Y=0,Z=0\})\\
    &(6,\quad\{X=4,Y=0,Z=0\})\\
    &\cdots\quad(\text{死循环})
    \end{align*}
    \end{solution}
    
    \begin{problem}
    $P_9$ 1.3.7 对程序
    \begin{align*}
        \mathscr{P}_3:& \\
        &X_1\leftarrow X_1+1\\
        &X_1\leftarrow X_1+1\\
        [A]&X_1 \leftarrow X_1-1\\
        &\text{IF}\quad X_1\neq 0 \quad\text{GOTO}\quad C\\
        [B]& Z\leftarrow Z+1\\
        &\text{IF}\quad Z\neq 0\quad \text{GOTO}\quad B\\
        [C]& X_1\leftarrow X_1-1\\
        &\text{IF}\quad X_1\neq 0\quad \text{GOTO}\quad A\\
        &\text{IF}\quad X_2\neq 0\quad \text{GOTO}\quad D\\
        &Y\leftarrow Y+1\\
        [D]& Y\leftarrow Y
    \end{align*}
    设输入变量的初始状态的值如下: \\
    (1) $X_1=2, X_2=0$\\
    (1) $X_1=4, X_2=3$\\
    (1) $X_1=1, X_2=4$\\
    写出计算
    \end{problem}
    \begin{solution}
    (1) 分析: 执行了$[A]$后, $X_1=3$, 跳转到$C$, 之后$X_1=2$, 跳转回$[A]$, $X_1=1$, 再跳转到$[C]$, $X_1=0$, 而$X_2=0$, 执行$Y\leftarrow Y+1$, 之后进入\textcolor{red}{空指令}$[D]$. 最后输出变量$Y=1$.\\
    计算: 
    \begin{align*}
    &(1,\quad\{X_1=2,X_2=0,Y=0,Z=0\})\\
    &(2,\quad\{X_1=3,X_2=0,Y=0,Z=0\})\\
    &(3,\quad\{X_1=4,X_2=0,Y=0,Z=0\})\\
    &(4,\quad\{X_1=3,X_2=0,Y=0,Z=0\})\\
    &(7,\quad\{X_1=3,X_2=0,Y=0,Z=0\})\\
    &(8,\quad\{X_1=2,X_2=0,Y=0,Z=0\})\\
    &(3,\quad\{X_1=2,X_2=0,Y=0,Z=0\})\\
    &(4,\quad\{X_1=1,X_2=0,Y=0,Z=0\})\\
    &(7,\quad\{X_1=1,X_2=0,Y=0,Z=0\})\\
    &(8,\quad\{X_1=0,X_2=0,Y=0,Z=0\})\\
    &(9,\quad\{X_1=0,X_2=0,Y=0,Z=0\})\\
    &(10,\quad\{X_1=0,X_2=0,Y=0,Z=0\})\\
    &(11,\quad\{X_1=0,X_2=0,Y=1,Z=0\})\\
    &(12,\quad\{X_1=0,X_2=0,Y=1,Z=0\})
    \end{align*}
    \\(2) 思路: 执行$[A]$之后, $X_1=5$, 跳转$[C]$, 之后$X_1=4$, 跳转回$[A]$, 在$[C],[A]$间来回跳转, 根据$X_1$的奇偶性, 最后在执行$[C]$的第一步之后$X_1=0$, $X_2=3\neq 0$, 跳转到\textcolor{red}{空指令}$D$. 最后输出变量$Y=0$.\\
    计算: 
    \begin{align*}
    &(1, \quad \{X_1=4,X_2=3,Y=0,Z=0\})\\
    &(2, \quad \{X_1=5,X_2=3,Y=0,Z=0\})\\
    &(3, \quad \{X_1=6,X_2=3,Y=0,Z=0\})\\
    &(4, \quad \{X_1=5,X_2=3,Y=0,Z=0\})\\
    &(7, \quad \{X_1=5,X_2=3,Y=0,Z=0\})\\
    &(8, \quad \{X_1=4,X_2=3,Y=0,Z=0\})\\
    &(3, \quad \{X_1=4,X_2=3,Y=0,Z=0\})\\
    &(4, \quad \{X_1=3,X_2=3,Y=0,Z=0\})\\
    &(7, \quad \{X_1=3,X_2=3,Y=0,Z=0\})\\
    &(8, \quad \{X_1=2,X_2=3,Y=0,Z=0\})\\
    &(3, \quad \{X_1=2,X_2=3,Y=0,Z=0\})\\
    &(4, \quad \{X_1=1,X_2=3,Y=0,Z=0\})\\
    &(7, \quad \{X_1=1,X_2=3,Y=0,Z=0\})\\
    &(8, \quad \{X_1=0,X_2=3,Y=0,Z=0\})\\
    &(9, \quad \{X_1=0,X_2=3,Y=0,Z=0\})\\
    &(11, \quad \{X_1=0,X_2=3,Y=0,Z=0\})\\
    &(12, \quad \{X_1=0,X_2=3,Y=0,Z=0\})
    \end{align*}
    \\(3) 思路: 执行$[A]$之后, $X_1=2$, 跳转$[C]$, 之后$X_1=1$, 跳转回$[A]$, $X_1=0$, 执行$[B]$, $Z=1$, 在$[B]$中进入\textcolor{red}{死循环}. 最后输出变量$Y=0$.\\
    计算: 
    \begin{align*}
    &(1, \quad \{X_1=1,X_2=4,Y=0,Z=0\})\\
    &(2, \quad \{X_1=2,X_2=4,Y=0,Z=0\})\\
    &(3, \quad \{X_1=3,X_2=4,Y=0,Z=0\})\\
    &(4, \quad \{X_1=2,X_2=4,Y=0,Z=0\})\\
    &(7, \quad \{X_1=2,X_2=4,Y=0,Z=0\})\\
    &(8, \quad \{X_1=1,X_2=4,Y=0,Z=0\})\\
    &(3, \quad \{X_1=1,X_2=4,Y=0,Z=0\})\\
    &(4, \quad \{X_1=0,X_2=4,Y=0,Z=0\})\\
    &(5, \quad \{X_1=0,X_2=4,Y=0,Z=0\})\\
    &(6, \quad \{X_1=0,X_2=4,Y=0,Z=1\})\\
    &(5, \quad \{X_1=0,X_2=4,Y=0,Z=1\})\\
    &(6, \quad \{X_1=0,X_2=4,Y=0,Z=2\})\\
    &\cdots \quad (\text{进入死循环})
    \end{align*}
    \end{solution}
    
    \begin{problem}
    $P_{12}$ 1.1 写出计算下述函数的$\mathscr{S}$程序(允许使用宏指令):\\
    (1) $f(x)=\left\lfloor x/2 \right\rfloor$(向下取整)\\
    (2) $x$偶数, $f(x)=1$; $x$奇数, $f(x)$无定义.
    \end{problem}
    
    \begin{solution}
        (1) 思路: \textcolor{blue}{除以$2$可以用一直减$2$表示.}
        \begin{align*}
            \mathscr{P}_1:&\\
            &Z\leftarrow Z+1\\
            &X\leftarrow X+1\quad(\text{$+1$的目的是为了保证$2$的输出是$1$, 以此类推})\\
            [A]&X \leftarrow X-1\\
            &X \leftarrow X-1\\
            &\text{IF} \quad X\neq 0 \quad \text{GOTO} \quad B\\
            &\text{IF} \quad Z\neq 0 \quad \text{GOTO} \quad E\\
            [B]&Y\leftarrow Y+1\\
            &\text{IF} \quad Y\neq 0 \quad \text{GOTO} \quad A
        \end{align*}
        使用宏指令的版本:
        \begin{align*}
            \mathscr{P}_1^*:&\\
            &X\leftarrow X+1\\
            [A]&X \leftarrow X-2\\
            &\text{IF} \quad X\neq 0 \quad \text{GOTO} \quad B\\
            & \text{GOTO}\quad E\\
            [B]&Y\leftarrow Y+1\\
            &\text{GOTO}\quad A
        \end{align*}
        (2) 思路: 对输入的$X$, 循环减两次$1$, 但每次都检查$X$是否是$0$, 来判断奇偶性, 为了兼容$0$, 首先加上$1$. \textcolor{red}{简单来说, 就是看减去的是奇数个还是偶数个$1$来进行出口的分类.}
        \begin{align*}
            \mathscr{P}_2^*:&\\
            &X\leftarrow X+1\\
            [A]&X\leftarrow X-1\\
            &\text{IF}\quad X=0 \quad \text{GOTO}\quad  B\\
            &X\leftarrow X-1\\
            &\text{IF}\quad X\neq 0 \quad \text{GOTO}\quad  A\\
            & \text{GOTO} \quad C\\
            [B]&Y\leftarrow Y+1\\
            &\text{GOTO}\quad E\\
            [C]&Z\leftarrow Z+1\\
            &\text{IF}\quad Z\neq 0 \quad \text{GOTO}\quad  C\\
        \end{align*} 
        如果不允许判断$X=0$, 可以这么写:
        \begin{align*}
            \mathscr{P}_2^*:&\\
            &X\leftarrow X+1\\
            [A]&X\leftarrow X-1\\
            &\text{IF}\quad X\neq 0 \quad \text{GOTO}\quad  B\\
            &\text{GOTO} \quad C\quad(\text{偶数出口})\\
            [B]&X\leftarrow X-1\\
            &\text{IF}\quad X\neq 0 \quad \text{GOTO}\quad  A\\
            &\text{GOTO} \quad D\quad(\text{奇数出口})\\
            [C]&Y\leftarrow Y+1\\
            &\text{GOTO}\quad E\\
            [D]&Z\leftarrow Z+1\\
            &\text{IF}\quad Z\neq 0 \quad \text{GOTO}\quad  D\quad(\text{死循环})
        \end{align*} 
    \end{solution}
    
    \begin{note}
    可供使用的宏指令:\begin{itemize}
        \item GOTO $A$
        \item $V\leftarrow V^{\prime}$ 
        \item 判断 $X=0$ 和跳转
    \end{itemize}
    \end{note}
    
    \begin{problem}
    $P_{12}$ 1.2 给出下列程序$\mathscr{P}$计算的函数$\psi_{\mathscr{P}}^{(1)}(x)$:
    \begin{align*}
        (1) [A]&X\leftarrow X+1\\
        &X\leftarrow X-1\\
        &\text{IF}\quad X\neq 0 \quad \text{GOTO}\quad  A\\
        (2) [A]&X\leftarrow X-1\\
        &\text{IF}\quad X=0 \quad \text{GOTO}\quad  A\\
        &X\leftarrow X-1\\
        &\text{IF}\quad X\neq 0 \quad \text{GOTO}\quad  A\\
        (3) \text{空程序}&
    \end{align*}
    \end{problem}
    \begin{solution}
    (1) $\psi_{\mathscr{P}_1}^{(1)}(x) =
    \begin{cases}
        \uparrow (\text{未定义})  & \text{if } x \in \mathbb{N}^*, \\
      0 & \text{if } x = 0.
    \end{cases}$\\
    (2) $\psi_{\mathscr{P}_1}^{(1)}(x) =\begin{cases}0,&x\text{是正偶数}\\ \uparrow , &x=0 \text{或}x\text{是奇数}\end{cases} $\\
    (3) $\psi_{\mathscr{P}_1}^{(1)}(x)=0, \forall x\in \mathbb{N}$
    \end{solution}

\begin{problem}
$P_{12}$ 1.3 证明下面的函数是部分可计算的:\\
(1) $x_1+x_2$;  (2) $x_1-x_2$;  (3) $x_1x_2$;  (4) 空函数
\end{problem}
\begin{solution}
\textcolor{red}{要证明一个函数是部分可计算的, 实际上就是可以用$S$函数把它写出来, "部分"指的是可以在某些点上没有定义}\\
(1) 思路:$x_2$一直减$1$, $x_1$一直加$1$, 直到减到零
\begin{align*}
    &Y\leftarrow X_1\\
    &Z\leftarrow X_2\\
    [A]& \text{IF} \quad Z\neq 0 \quad \text{GOTO} \quad B\\
    & \text{GOTO}\quad E\\
    [B]& Z\leftarrow Z-1\\
    &Y\leftarrow Y+1\\
    &\text{GOTO}\quad  A
\end{align*} 
(2) 思路: 如果$x_1<x_2$, 那么无定义, 因此看谁先减到$0$.
\begin{align*}
    &Z_1\leftarrow X_1\\
    &Z_2\leftarrow X_2\\
    [A]& \text{IF} \quad Z_1\neq 0 \quad \text{GOTO} \quad B\\
    &\text{GOTO}\quad  D_1\\
    [B]& \text{IF} \quad Z_2\neq 0 \quad \text{GOTO} \quad C\\
    &\text{GOTO}\quad  D_2\\
    [C]& Z_1\leftarrow Z_1-1\\
    & Z_2\leftarrow Z_2-1\\
    &\text{GOTO}\quad  A\\
    [D_1]& Y\leftarrow Z_2\\
    &\text{GOTO}\quad  E\\
    [D_2]& Y\leftarrow Z_1\\
    &\text{GOTO}\quad  E
\end{align*}
(3) 思路: 如果有$0$, 返回$0$; 如果都不是$0$, 一直减$x_2$, 同时对$x_1$做加法(已经证明是部分可计算函数)
\begin{align*}
    &Z_1\leftarrow X_1\\
    &Z_2\leftarrow X_2\\
    [A]& \text{IF} \quad Z_1\neq 0 \quad \text{GOTO} \quad B\\
    &\text{GOTO}\quad  E\\
    [B]& \text{IF} \quad Z_2\neq 0 \quad \text{GOTO} \quad C\\
    &\text{GOTO}\quad  E\\
    [C]& Z_2\leftarrow Z_2-1\\
    &Z_1\leftarrow Z_1+Z_1\\
    &\text{GOTO}\quad  B
\end{align*}
(4) 思路: 空函数就是处处无定义, 直接进入死循环即可.
\begin{align*}
    [A]&Z\leftarrow Z+1\\
    & \text{IF} \quad Z\neq 0 \quad \text{GOTO} \quad A\\
\end{align*}
\end{solution}

\begin{problem}
    $P_{13}$ 1.4 证明下述谓词是可计算的\\
(1) $x\geq a$, $a$是正整数\\
(2) $x_1\leq x_2$\\
(3) $x_1=x_2$
\end{problem}
\begin{solution}
\textcolor{red}{要证明一个谓词是可计算的,实际上就是证明这个谓词(判断过程)可以用$S$语言表示}\\
(1) 思路: 两边一直减$1$, 看谁先减到$0$, \textcolor{blue}{又因为是大于等于, 所以可以先验证$Z_2$}
\begin{align*}
    &Z_1\leftarrow X\\
    &Z_2\leftarrow a\\
    [A]& \text{IF} \quad Z_2\neq 0 \quad \text{GOTO} \quad B\\
    &\text{GOTO}\quad  D_2\\
    [B]& \text{IF} \quad Z_1\neq 0 \quad \text{GOTO} \quad C\\
    &\text{GOTO}\quad  D_1\\
    [C]&Z_1\leftarrow Z_1-1\\
    &Z_2\leftarrow Z_2-1\\
    &\text{GOTO}\quad  A\\
    [D_1]&\text{GOTO}\quad  E\\
    [D_2]&Y\leftarrow Y+1\\
    &\text{GOTO}\quad  E
\end{align*}
(2) 思路: 和第一问思路类似, 注意取等条件对判定顺序的影响
\begin{align*}
    &Z_1\leftarrow X_1\\
    &Z_2\leftarrow X_2\\
    [A]& \text{IF} \quad Z_1\neq 0 \quad \text{GOTO} \quad B\\
    &\text{GOTO}\quad  D_2\\
    [B]& \text{IF} \quad Z_2\neq 0 \quad \text{GOTO} \quad C\\
    &\text{GOTO}\quad  D_1\\
    [C]&Z_1\leftarrow Z_1-1\\
    &Z_2\leftarrow Z_2-1\\
    &\text{GOTO}\quad  A\\
    [D_1]&\text{GOTO}\quad  E\\
    [D_2]&Y\leftarrow Y+1\\
    &\text{GOTO}\quad  E
\end{align*}
(3) 思路: 可以使用(1)和(2)的判定了
\begin{align*}
    &Z_1\leftarrow X_1\\
    &Z_2\leftarrow X_2\\
    [A]&\text{IF} \quad Z_1\leq Z_2 \quad \text{GOTO} \quad B\\
    &\text{GOTO}\quad  E\\
    [B]&\text{IF} \quad Z_2\leq Z_1 \quad \text{GOTO} \quad C\\
    &\text{GOTO}\quad  E\\
    [C]&Y\leftarrow Y+1\\
    &\text{GOTO}\quad  E
\end{align*}
\end{solution}

\begin{problem}
$P_{16}\; 2.1.5$用基本的原始递归函数来表示下面的函数, 从而它们也是原始递归函数\\
(1) $E(x)=\begin{cases}
    1, &x\text{偶数}\\
    0, &x\text{奇数}
\end{cases}$\\
(3) $\max(x,y)=\begin{cases}
    x&,x\geq y\\
    y,&x<y
\end{cases}$
\end{problem}
\begin{solution}
(1) $E(0)=1, E(x+1)=\alpha(E(x))$.\\
(2) $\max(x,y)=x\alpha(y\dotminus x)+y\alpha(x \dotminus p(y))$\\
这是因为, 当$x\geq y$时, $y\dotminus x=0$, 因此$\alpha(y\dotminus x)=1$. 但为了防止$x=y$且取非零值($x=y=0$不影响)的时候, 得到$2x$, 因此要保证$x=y$的时候, $y$的系数不能是$1$, 因此取$x\dotminus p(y)$ 
\end{solution}

\begin{problem}
$P_{22}\;2.3.4$利用极小化给出下述函数$f(x)$:\\
(1) 
\end{problem}


\end{document}
