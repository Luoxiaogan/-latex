\documentclass[10pt, a4paper, oneside]{ctexart}
\usepackage{amsmath, amsthm, amssymb, bm, color, xcolor, framed, graphicx, hyperref, mathrsfs, etoolbox, wrapfig, xurl}
\usepackage[thicklines]{cancel}
\usepackage{enumitem} % 用于更灵活的列表环境
\usepackage{geometry} % 调整页面边距
\usepackage{fancyhdr} % 页眉页脚定制
\hypersetup{
    colorlinks=true,            %链接颜色
    linkcolor=black,             %内部链接
    filecolor=magenta,          %本地文档
    urlcolor=cyan,              %网址链接
    pdftitle={Overleaf Example},
    pdfpagemode=FullScreen,
    }

% 页边距设置
\geometry{left=3cm, right=3cm, top=2.5cm, bottom=2.5cm}

% 页眉页脚设置
\pagestyle{fancy}
\fancyhf{}
\fancyhead[L]{\leftmark} % 左页眉显示章节标题
\fancyhead[R]{\thepage} % 右页眉显示页码

% 标题设置
\title{\textbf{应随 HW1}}
\author{罗淦 2200013522}
\date{\today}

% 行距设置
\linespread{1.2}

% 定义黑色边框的 problem 环境
\newenvironment{problem}{\begin{framed}\par\noindent\textbf{\textit{题目. }}}{\end{framed}\par}
\newenvironment{solution}{%
  \par\noindent\textbf{\textit{解答. }}\ignorespaces
}{%
  \hfill\ensuremath{\square}\par % 在结尾添加正方形
}
\newenvironment{note}{\par\noindent\textbf{\textit{题目的注记. }}\ignorespaces}{\par}


% 允许公式在页面之间自动换行
\allowdisplaybreaks

\begin{document}

\maketitle

% 添加目录
%\tableofcontents
%\newpage

\section{HW1}

\begin{problem}
    2. $\{S_n\}$一维简单随机游动. $\forall n\geq 0$, $X_n=\max_{0\leq k\leq n}S_k$. $\{X_n\}$是马氏链吗? 说明之. 
    \end{problem}
    \begin{solution}
    理解: $X_n$理解为前$n$步到达过的最大坐标$\max_{0\leq k\leq n}S_k$, 考虑用$S_n$表示$X_n$. \\$X_{n+1}=\begin{cases}
        X_n, & S_{n+1}\leq X_n\\
        S_{n+1}, &S_{n+1}>X_n
    \end{cases}$, $X_{n+1}$的状态只取决于$S_n$和$X_n$的状态(但还是很难分析马氏性啊)\\ 
    考虑条件概率:
    \begin{align*}
        &P(X_{n+1}=i_{n+1}|X_{n}=i_n,X_{n-1}=i_{n-1},\cdots,X_1=i_1,X_0=i_0)\\
        =&P(S_{n+1}\leq X_n)\textcolor{blue}{P(X_n=i_{n+1}|X_{n}=i_n,X_{n-1}=i_{n-1},\cdots,X_1=i_1,X_0=i_0)}\\&+P(S_{n+1}>X_n)P(S_{n+1}=i_{n+1}|X_{n}=i_n,X_{n-1}=i_{n-1},\cdots,X_1=i_1,X_0=i_0)\\
        =&P(S_{n+1}\leq X_n=i_n)\textcolor{blue}{P(X_n=i_{n+1}|X_{n}=i_n)}\\&+P(S_{n+1}>X_n=i_n)P(S_{n+1}=i_{n+1}|X_{n}=i_n,X_{n-1}=i_{n-1},\cdots,X_1=i_1,X_0=i_0)
    \end{align*}
    参考:\\
    1. \url{https://math.stackexchange.com/questions/683123/the-maximum-of-a-simple-random-walk}\\
    2. \url{https://math.stackexchange.com/questions/683060/let-s-n-be-a-simple-random-walk-m-n-is-maxs-1-s-2-ldots-s-n-is-m-n}\\
    \textcolor{red}{$\{X_n\}$不是马氏链, 反例如下}.\\
    因为是简单马氏链, 因此$S_0=X_0=0$. \\下面考虑$\{X_3=1\}$, 那么之前${(S_0,S_1,S_2,S_3)}$可能的状态集合有:$(0,1,0,1),(0,1,0,-1),(0,-1,0,1)$, 且都是等概率的($p=\frac{1}{16}$). \\那么条件概率$P(X_4=1|X_3=1)=\frac{4}{6}=\frac{2}{3}$. \\但对于$P(X_4=1|X_3=1,X_2=0)$, 那么对应$(0,-1,0,1)$的情况, 此时$P(X_4=1|X_3=1,X_2=0)=\frac{1}{2}$, 不符合马氏性的定义.
    \end{solution}
    \begin{note}
        \textbf{\textcolor{red}{有没有什么深层次的原因呢?}}
    \end{note}
    
    \begin{problem}
    3. 某数据通信系统由 $n$ 个中继站组成, 从上一站向下一站传送信号 0 或 1 时, 接收的正确率为 $p$. 现用 $X_0$ 表示初始站发出的数字, 用 $X_k$ 表示第 $k$ 个中继站接收到的数字。
    
    (1) 写出 $\left\{X_k: 0 \leqslant k \leqslant n\right\}$ 的转移概率.
    (2) 求
    $$
    P\left(X_0=1 \mid X_n=1\right)=\frac{\alpha+\alpha(p-q)^n}{1+(2 \alpha-1)(p-q)^n}
    $$
    其中 $\alpha=P\left(X_0=1\right), q=1-p$. 并解释上述条件概率的实际意义.
    \end{problem}
    
    \begin{solution}
    (1) $\mathbf{P}=\begin{pmatrix}
        p&1-p\\
        1-p&p
    \end{pmatrix}=\begin{pmatrix}
        p_{00}&p_{01}\\p_{10}&p_{11}
    \end{pmatrix}$.\\
    (2) 对角化$\mathbf{P}$. $|\lambda \mathbf{I}-\mathbf{P}|=(\lambda-1)(\lambda -(p-q))$\\$\lambda_1=1$对应特征向量$(1,1)^T$, $\lambda_2=p-q$对应特征向量$(1,-1)^T$\\因此$\mathbf{P}^n=\begin{pmatrix}
        1&1\\1&-1
    \end{pmatrix}^{-1}\begin{pmatrix}
        1&0&\\0&(p-q)^n
    \end{pmatrix}\begin{pmatrix}
        1&1\\1&-1
    \end{pmatrix}=\frac{1}{2}\begin{pmatrix}
        1+(p-q)^n&1-(p-q)^n\\1-(p-q)^n&1+(p-q)^n
    \end{pmatrix}$\\
    下面计算:
    \begin{align*}
        P(X_0=1|X_n=1)&=\frac{P(X_0=1,X_n=1)}{P(X_n=1)}=\frac{P(X_0=1)P(X_n=1|X_0=1)}{P(X_n=1)}\\
        &=\frac{\alpha\cdot (0,1)^T\mathbf{P}^n[2]}{(1-\alpha,\alpha)^T\mathbf{P}^n[2]}=\frac{\alpha+\alpha(p-q)^n}{1+(2\alpha-1)(p-q)^n}
    \end{align*}
    实际意义: 已知第$n$个中继站接收到$1$的情况下, 最开始真实发出的也是$1$的"后验概率".
    \end{solution}
    
    \begin{problem}
    5. 某篮球运动员投球成功的概率取决于他前两次的投球成绩. 如果两次都成功,则下次投球成功的概率为$\frac{3}{4}$;如果两次都失败,下次投球成功的概率为$\frac{1}{2}$;如果两次一次成功一次失败,下次投球成功的概率为$\frac{2}{3}$。用马氏链来刻画连续投球,求出投球成功的概率近似值。
    \end{problem}
    \begin{solution}
    设成功是$W$,失败是$L$,那么设状态空间为$S=\{S_1,S_2,S_3,S_4\}$,对应$S_1=WW,S_2=WL,S_3=LW,S_4=LL$,转移矩阵是(考虑前两次投球所属的状态空间,根据这一次投球的结果,得到前一次加上这一次所处的状态空间)
        $$P=\begin{pmatrix}
            \frac{3}{4}&\frac{1}{4}&0&0\\0&0&\frac{2}{3}&\frac{1}{3}\\ \frac{2}{3}&\frac{1}{3}&0&0\\ 0&0&\frac{1}{2}&\frac{1}{2}
        \end{pmatrix}$$
        之后,计算perron vector,$\pi=(\pi_1,\cdots,\pi_4)^T=(\frac{1}{2},\frac{3}{16},\frac{3}{16},\frac{1}{8})^T$,那么得到了平衡状态下在各个状态的概率,因此可以计算成功率为
        $$p=\pi_1\cdot \frac{3}{4}+\pi_2\cdot \frac{2}{3}+\pi_3\cdot \frac{2}{3}+\pi_4\cdot \frac{1}{2}=\frac{11}{16}$$
    \end{solution}
    \begin{note}
    把连续两次的成绩看成一个状态,来找转移概率.
    \end{note}
    
    
    \begin{problem}
    7. 假设某加油站给一辆车加油需要一个单位时间 (比如, 5 分钟). 令 \( {\xi }_{n} \) 是第 \( n \) 个单位时间来加油的汽车数. 假设 \( {\xi }_{1},{\xi }_{2},\cdots \) 独立同分布, 取值非负整数, \( P\left( {{\xi }_{1} = k}\right) = {p}_{k},k \geq 0 \) . 在任意时刻 \( n \) ,如果加油站有车, 那么加油站为其中一辆车加油 (耗时一个单位时间, 然后该汽车在时刻 \( n + 1 \) 离开加油站); 否则,加油站什么都不做. 将 \( n \) 时刻加油站中的汽车数记为 \( {X}_{n} \) . 写出 \( \left\{ {X}_{n}\right\} \) 的状态空间与转移概率.
    \end{problem}
    \begin{solution}
    $S=\{0,1,2,\cdots\}=\mathbb{Z}$,$P(X_{n+1}=j|X_n=i)=p_{j-i+1}, \forall j\geq i-1$.
    \end{solution}

\end{document}