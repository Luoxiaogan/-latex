\documentclass[10pt, a4paper, oneside]{ctexart}
\usepackage{amsmath, amsthm, amssymb, bm, color, xcolor, framed, graphicx, hyperref, mathrsfs, etoolbox, wrapfig}
\usepackage[thicklines]{cancel}
\usepackage{enumitem} % 用于更灵活的列表环境
\usepackage{geometry} % 调整页面边距
\usepackage{fancyhdr} % 页眉页脚定制
\hypersetup{
    colorlinks=true,            %链接颜色
    linkcolor=black,             %内部链接
    filecolor=magenta,          %本地文档
    urlcolor=cyan,              %网址链接
    pdftitle={Overleaf Example},
    pdfpagemode=FullScreen,
    }

% 页边距设置
\geometry{left=3cm, right=3cm, top=2.5cm, bottom=2.5cm}

% 页眉页脚设置
\pagestyle{fancy}
\fancyhf{}
\fancyhead[L]{\leftmark} % 左页眉显示章节标题
\fancyhead[R]{\thepage} % 右页眉显示页码

% 标题设置
\title{\textbf{博弈论 HW1}}
\author{罗淦 2200013522}
\date{\today}

% 行距设置
\linespread{1.2}

% 定义黑色边框的 problem 环境
\newenvironment{problem}{\begin{framed}\par\noindent\textbf{\textit{题目. }}}{\end{framed}\par}
\newenvironment{solution}{%
  \par\noindent\textbf{\textit{解答. }}\ignorespaces
}{%
  \hfill\ensuremath{\square}\par % 在结尾添加正方形
}
\newenvironment{note}{\par\noindent\textbf{\textit{题目的注记. }}\ignorespaces}{\par}


% 允许公式在页面之间自动换行
\allowdisplaybreaks

\begin{document}

\maketitle

% 添加目录
%\tableofcontents
%\newpage

\section{作业1}


\subsection{\texorpdfstring{$n$个参与者的一价拍卖和$n$个参与者的二价拍卖}{n个参与者的一价拍卖和n个参与者的二价拍卖}}


\begin{solution}
(a) 学习讲义上的书写, 需要阐明参与者, 策略空间, 收益函数.\\
参与者$N=\{1,2,\cdots,n\}$\\
策略空间$S_i=[0,+\infty), \forall i\in N$.\\
每个玩家的收益函数:
\begin{align*}
    v_i(p_1,\cdots,p_n)=\begin{cases}
        v_i-p_i& \text{if}\quad  p_i>\max_{k\neq i,k\in N}\{p_k\}\\
        \frac{v_i-p_i}{|\{k|p_k=\max\{p_1,\cdots,p_n\}\}|}& \text{if} \quad p_i=\max\{p_1,\cdots,p_n\} \quad \text{and}\quad |\{k|p_k=\max\{p_1,\cdots,p_n\}\}|>1\\
        0&\text{if} \quad p_i<\max\{p_1,\cdots,p_n\} 
    \end{cases}
\end{align*}
\textcolor{blue}{注意, 右边的$v_i$是 willingness to pay, 左边的$v_i(p_1,\cdots,p_n)$是收益函数(消费者剩余)}\\
(b) $n$个参与者的二价拍卖: \\
参与者$N=\{1,\cdots,n\}$\\
策略空间$S_i=[0,+\infty), \forall i\in N$\\
每个玩家的收益函数:
\begin{align*}
    v_i(p_1,\cdots,p_n)=\begin{cases}
        v_i-\max_{k\neq i,k\in N}\{p_k\}& \text{if}\quad  p_i>\max_{k\neq i,k\in N}\{p_k\}\\
        \frac{v_i-\textcolor{red}{\max_{k\notin J, k\in N}\{p_k\}}}{|\{j|p_j=\max\{p_1,\cdots,p_n\}\}|}& \text{if} \quad p_i=\max\{p_1,\cdots,p_n\} \quad \text{and}\quad |\textcolor{red}{\underbrace{\{j|p_j=\max\{p_1,\cdots,p_n\}\}}_{\text{这个集合定义为$J$}}}|>1\\
        0&\text{if} \quad p_i<\max\{p_1,\cdots,p_n\} 
    \end{cases}
\end{align*}
\textcolor{blue}{注意是支付次高的价格, 在出最高价的人数大于一的时候, 要把他们都剔除掉.}
\end{solution}


\subsection{4.5, Iterated Elimination}

\begin{solution}
第一轮, 对行玩家, $U$被$M$完全占优, 删去$U$. 对列玩家, 没有完全被占优策略.\\
第二轮, 对行玩家, 没有完全被占优策略. 对列玩家, $C$被$R$完全占优, 删去$C$.\\
此时没有策略被占优, 留下来的是$\{M,D\}\times\{L,R\}$. 因此存活下来的策略组合有: $(M,L),(M,R),(D,L),(D,R)$.
\end{solution}


\subsection{4.6, Roommates}

\begin{solution}
(a) $v_i=(10-t_j-t_i)t_i$. 固定$t_j\geq 0$, $\max_{t_i\geq 0}v_i$得到最优反应.
\begin{align*}
    BR_i(t_j)=\begin{cases}
        \frac{10-t_j}{2}&\text{if}\quad 0\leq t_j<10 \\
        0 &\text{if}\quad 10\leq t_j
    \end{cases}=\max\{\frac{10-t_j}{2},0\}
\end{align*}
根据对称性, 有
\begin{align*}
    BR_j(t_i)=\begin{cases}
        \frac{10-t_i}{2}&\text{if}\quad 0\leq t_i<10 \\
        0 &\text{if}\quad 10\leq t_i
    \end{cases}=\max\{\frac{10-t_i}{2},0\}
\end{align*}
(b) \textcolor{red}{按照约定, 单轮删除, 对于每一个玩家, 一直删除完全被占有策略直到无法删除为止}.\\
每个人的策略空间$S_i=S_j=[0,+\infty)$.\\
对于$[0,5]$, 根据(a)中对最优反应的分析, \textcolor{blue}{存在一个对手策略, 使得$x\in [0,5]$是最优反应, 那么它一定不是严格被占优的. 因此$[0,5]$不会被删掉.}\\
对于$\forall t_i\in(5,+\infty)$的策略, 因为收益函数$v_i=(10-t_j-t_i)t_i$关于$t_i$的导函数$10-t_j-2t_i<0$, 因此$v_i(5)>v_i(t_i)$, 被策略$5$严格占优, 因此被删除.\\
因此对于两个玩家, 第一轮删除之后, 留下来的策略都是$[0,5]$.\\
(c) 直接求解$x=\frac{10-x}{2}\Rightarrow x=\frac{10}{3}$.
\end{solution}


\subsection{4.7, Campaigning}

\begin{solution}
(a) 参与者$N={1,2}$\\
策略空间$S_1=S_2=\{P,B,N\}$\\
payoff function(收益函数):
\begin{align*}
    &v_1(P,P)=0.5,v_1(P,B)=0,v_1(P,N)=0.3\\
    &v_1(B,P)=1,v_1(B,B)=0.5,v_1(B,N)=0.4\\
    &v_1(N,P)=0.7,v_1(N,B)=0.6,v_1(N,N)=0.5\\\\
    &v_2(P,P)=0.5,v_2(B,P)=0,v_2(N,P)=0.3\\
    &v_2(P,B)=1,v_2(B,B)=0.5,v_2(N,B)=0.4\\
    &v_2(P,N)=0.7,v_2(B,N)=0.6,v_2(N,N)=0.5
\end{align*}
\\(b) 博弈矩阵如下:
\begin{center}
\begin{tabular}{|c|c|c|c|}
    \hline
      & P & B & N \\ \hline
    P & (0.5,0.5) & (0,1) & (0.3,0.7) \\ \hline
    B & (1,0) & (0.5,0.5) & (0.4,0.6) \\ \hline
    N & (0.7,0.3) & (0.6,0.4) & (0.5,0.5) \\ \hline
\end{tabular}
\end{center}
(c) 第一轮, 对行玩家, $P$被$B$严格占优, 删去$P$, $B,N$之间没有严格占优关系; 对列玩家, $P$被$B$严格占优, 删去$P$, $B,N$之间没有严格占优关系\\
第二轮, 对行玩家, $B$被$N$严格占优, 删去$N$; 对列玩家, $B$被$N$严格占优, 删去$N$\\
剩下策略组合$(N,N)$; this procedure will lead to a clear prediction.
\end{solution}


\subsection{4.8 Beauty Contest Best Responses}

\begin{solution}
(a) 平均数的$\frac{3}{4}$是$\frac{20(n-1)+x}{n}\cdot \frac{3}{4}=15+\frac{3(x-20)}{4n}$, 不妨取$x=16$, 那么$15+\frac{3(x-20)}{4n}=15-\frac{3}{n}<15<16<19$, 因此$i$号玩家获胜, $19$当然不是唯一的最优反应.\\
(b) 设$a=\frac{20(n-1)+x}{n}\cdot \frac{3}{4}$\\
如果$a\leq x$, 那么只需要$x<20$即可($x\leq 19$), 即此时$x\in [15+\frac{3(x-20)}{4n},20)\cap \mathbb{Z}$.\\
如果$x<a$, 那么需要$0\leq a-x<20-a\iff x\in (30+\frac{3x-60-40n}{2n}, 15+\frac{3(x-20)}{4n})\cap \mathbb{Z}$.\\
因此, 如果$i$玩家是最终胜者, 策略取值范围是$(30+\frac{3x-60-40n}{2n}, 20)\cap \mathbb{Z}$, 当然与$n$有关.
\end{solution}

\subsection{Tow-player game}
\begin{solution}
(a) 取$x_i>0$, 那么$v_i(x_i,x_j)=\arctan x_i$, 那么任意比$x_i$大的策略都严格占优$x_i$. 因此所有的大于$0$的策略都是被严格占优的.\\
(b) 取$x_i=0$, 那么$v_i(0,x_j)=\begin{cases}
    2& \text{if} \quad x_j=1\\
    0 & \text{if otherwise}
\end{cases}$, $x_j=1$的时候, $2>\frac{\pi}{2}$, 是最优反应, 因此一定不是被严格占优的.\\
(c) 不是互为最佳反应(\textcolor{red}{mutual best responses}). 因为对方取$0$的时候, 我取非零值有正的收益函数, 不是最优反应.
\end{solution}

\subsection{\texorpdfstring{$n$-firm Cournot competition}{n-firm Cournot competition}}


\begin{solution}
    (a) Write down its normal form game(写出标准形式博弈, \textcolor{blue}{其实就是写出:玩家, 策略空间, 收益函数})\\
    玩家: $N=\{1,\cdots,n\}$\\
    策略空间: $S_i=[0,+\infty)$\\
    收益函数: 
    $$v_i(q_1,\cdots,q_n)=q_i\cdot\max\{100-\sum_{j=1}^nq_j,0\}-10q_i=\max\{q_i(90-\sum_{j\neq i}q_j -q_i), -10q_i\}$$
    (b) 计算最优反应: 
    $$BR_i(q_{-i})=\begin{cases}
        \frac{90-\sum_{j\neq i}q_j}{2}&\text{if} \quad 0\leq \sum_{j=1}^n q_j \leq 90\\
        0&\text{if} \quad  90\leq \sum_{j=1}^n q_j 
    \end{cases}=\max\{\frac{90-\sum_{j\neq i}q_j}{2},0\}$$
    因为$q_j\in [0,+\infty)$, 所以$BR_i(q_{-i})\in [0,45]$, 即在某个对手策略下是最优反应, 因此一定不是被严格占优的. 而对$(45,+\infty)$的策略, 都被$45$严格占优. 第一次第一轮删除之后, 所有玩家的策略组合都是$[0,45]$.\\
    \textcolor{red}{不能直接联立最优反应来计算最后存活的策略, 下面需要对$n$进行分类讨论.}\\
    根据$[0,45]$的策略空间, 当$n=2$时, $BR_i(q_{-i})=BR_i(q_j)=\frac{90-q_j}{2}\in [22.5,45]$, 这种情况下, 还可以迭代进行不断删除被占优策略(实际上就是在二维的$BR$图上联立计算交点), 得到$x=\frac{90-x}{2}\iff x=30$.\\
    当$n\geq 3$时, $BR_i(q_{-i})=\max\{\frac{90-\sum_{j\neq i}q_j}{2},0\}\in \textcolor{red}{[0,45]=S_{i}}$, 也即限制了新的策略空间之后, 最优反应的取值范围没有变化, 等于策略空间. 那么, 策略空间中的每一个策略, 都存在一个对手的策略, 使得它是最优反应, 因此不是被占优的, 所以, 达到这一步之后就不能进行不断删除被占优策略了! 即, 最后剩下的策略空间就是$[0,45]$.
    \end{solution}

\end{document}

